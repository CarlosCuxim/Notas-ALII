\section{Matrices triangularizables y diagonalizables}

Uno de los objetivos de la semejanza de matrices es encontrar la matriz más ``simple'' asociada a una transformación lineal, para realizar operaciones con las matrices de manera más sencilla. Un especial interés se da a las matrices semejantes a matrices triangulares o diagonales.

\begin{defi}
  Sea $M$ una matriz, decimos que es \emph{triangularizable} (\emph{diagonalizable}) si existe una matriz $T$ triangular (diagonal) tal que $M \sim T$.
\end{defi}

A priori, no tenemos una forma de saber si una matriz es triangularizable o diagonalizable. Hasta ahora, solo sabemos que dentro de una misma clase se preserva la determinante y traza, pero la vuelta no es cierta, aunque dos matrices compartan traza y determinante no necesariamente son semejantes. Aun necesitamos herramientas más sofisticadas para saber cuando una matriz es triangularizable o diagonalizable.

\subsection{Clasificación de matrices de $2\times 2$}

Con los conocimientos que tenemos hasta este punto podemos clasificar las clases de equivalencia de $\M_2(\C)$. Por lo que en lo que resta de esta sección, nos dedicaremos a determinar cuando una matriz compleja de $2\times 2$ es triangularizable o diagonalizable.

Para ello, calculemos algunas conjugaciones. La primera conjugación nos permite ``invertir'' los elementos de una matriz.
\begin{equation}
  \begin{pmatrix} 0 & 1 \\ 1 & 0 \end{pmatrix}
  \begin{pmatrix} a & b \\ c & d \end{pmatrix}
  \begin{pmatrix} 0 & 1 \\ 1 & 0 \end{pmatrix}
    = \begin{pmatrix} d & c \\ b & a \end{pmatrix}. \label{eq:ConjI}
\end{equation}
Notemos que si $c = 0$ entonces la matriz es triangular superior, y aplicando esta conjugación obtenemos que es semejante a una matriz triangular inferior. Aplicando una conjugación similar es posible demostrar que toda matriz semejante a una matriz triangular superior es semejante a una matriz triangular inferior y viceversa. De este modo, para que una matriz sea triangularizable no importa si es semejante a una matriz triangular inferior o superior.

Para la siguiente conjugación, tomemos a cualquier $s \in \C$ y veamos que
\begin{equation}
  \begin{pmatrix} 1 & 0 \\ s & 1 \end{pmatrix}
  \begin{pmatrix} a & b \\ c & d \end{pmatrix}
  \begin{pmatrix} 1 & 0 \\ -s & 1 \end{pmatrix}
    = \begin{pmatrix} -bs+a & b \\ -bs^2 + (a-d)s + c  & bs+d \end{pmatrix}. \label{eq:ConjII}
\end{equation}
Ya que estamos trabajando sobre $\C$, notemos que si $b \neq 0$ entonces existe $s \in \C$ tal que $-bs^2 + (a-d)s + c = 0$, por lo que la matriz sería semejante a una triangular superior. En el caso que $b = 0$, la matriz ya es triangularizable y por lo ya comentando en la conjugación anterior podemos concluir que toda matriz de $\M_2(\C)$ es semejante a una matriz triangular superior. En otras palabras
\[ 
    \begin{pmatrix} a & b \\ c & d \end{pmatrix} \sim \begin{pmatrix} u & v \\ 0 & w \end{pmatrix}.
\]

Para la siguiente conjugación, consideremos a $s \in \C$ y para cualquier matriz triangular superior veamos que
\begin{equation}
  \begin{pmatrix} 1 & s \\ 0 & 1 \end{pmatrix}
  \begin{pmatrix} u & v \\ 0 & w \end{pmatrix}
  \begin{pmatrix} 1 & -s \\ 0 & 1 \end{pmatrix}
    = \begin{pmatrix} u & (w-u)s + v \\ 0 & w \end{pmatrix}.
\end{equation}
Notemos que si $u \neq w$  entonces existe $s \in \C$ tal que $(w-u)s + v = 0$, de este modo tenemos que las matrices pueden ser semejantes a una matriz diagonal con elementos distintos en la diagonal, o una matriz triangular con elementos en la diagonal iguales. Es decir, si $u,w \in \C$ entonces
\[
  \begin{pmatrix} a & b \\ c & d \end{pmatrix} \sim \begin{pmatrix} u & 0 \\ 0 & w \end{pmatrix}
    \Eqor
    \begin{pmatrix} a & b \\ c & d \end{pmatrix} \sim \begin{pmatrix} u & w \\ 0 & u \end{pmatrix}.
\]
Ahora, en la segunda equivalencia, si $w \neq 0$ entonces notemos que

\begin{equation}
  \begin{pmatrix} 1 & 0 \\ 0 & w \end{pmatrix}
  \begin{pmatrix} u & w \\ 0 & u \end{pmatrix}
  \begin{pmatrix} 1 & 0 \\ 0 & w^{-1} \end{pmatrix}
    = \begin{pmatrix} u & 1 \\ 0 & u \end{pmatrix}.
\end{equation}

De esta forma, tenemos que toda matriz compleja de $2 \times 2$ es semejante a alguna de estas matrices
\[
  \begin{pmatrix} a & b \\ c & d \end{pmatrix} \sim \begin{pmatrix} u & 0 \\ 0 & w \end{pmatrix}
    \Eqor
    \begin{pmatrix} a & b \\ c & d \end{pmatrix} \sim \begin{pmatrix} u & 1 \\ 0 & u \end{pmatrix}.
\]
con $u$ y $w$ arbitrarios. Aunque aun queda determinar si estas matrices son los representantes de las clases de equivalencia o si hay repeticiones.

Primero notemos que si $M = P^{-1} N P$ entonces $M^2 = (P^{-1} N P)(P^{-1} N P) = P^{-1}N^2 P$, de igual forma $M^3 = (P^{-1}N^2 P) (P^{-1} N P) = P^{-1}N^3 P$, de manera inductiva se puede ver que para todo $k \in \N$ se cumple que
\[
  M^k = P^{-1} N^k P
\]
Usando esta propiedad, notemos que si $c_k M^k + \cdots + c_2 M^2 + c_1 M + c_0 I = \bec 0$ entonces
\begin{align*}
  \bec 0 &= c_k M^k + \cdots + c_2 M^2 + c_1 M + c_0 I \\
    &= c_k P^{-1}N^k P + \cdots + c_2 P^{-1}N^2 P + c_1 P^{-1}N P + c_0 I \\
    &= P^{-1} (c_k N^k + \cdots + c_2 N^2 + c_1 N + c_0 I) P.
\end{align*}
De aquí es claro que $c_k N^k + \cdots + c_2 N^2 + c_1 N + c_0 I$, lo que nos deja con la siguiente proposición.

\begin{prop}
  Sean $M, N \in \M_n(\F) - \{\bec 0\}$ si $M \sim N$ entonces $\sum_{i=0}^k c_k M^k = \bec 0$ si y solo si $\sum_{i=0}^k c_k N^k = \bec 0$, donde $M^0 = I_n$. \qed
\end{prop}

