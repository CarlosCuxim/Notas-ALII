\section{Matrices triangularizables y diagonalizables} \label{sec:TyDdeMat2x2}

Uno de los objetivos de la semejanza de matrices es encontrar la matriz más ``simple'' asociada a una transformación lineal, para realizar operaciones con las matrices de manera más sencilla. Un especial interés se da a las matrices semejantes a matrices triangulares o diagonales.

\begin{defi}
  Sea $M$ una matriz, decimos que es \emph{triangularizable} (\emph{diagonalizable}) si existe una matriz $T$ triangular (diagonal) tal que $M \sim T$.
\end{defi}

A \emph{priori}, no tenemos una forma de saber si una matriz es triangularizable o diagonalizable. Hasta ahora, solo sabemos que dentro de una misma clase se preserva la determinante y traza, pero la vuelta no es cierta, aunque dos matrices compartan traza y determinante no necesariamente son semejantes. Aun necesitamos herramientas más sofisticadas para saber cuando una matriz es triangularizable o diagonalizable.

\subsection{Clasificación de matrices de \texorpdfstring{$2\times 2$}{2x2}}

Con los conocimientos que tenemos hasta este punto podemos clasificar las clases de equivalencia de $\M_2(\C)$. Por lo que en lo que resta de esta sección, nos dedicaremos a determinar cuando una matriz compleja de $2\times 2$ es triangularizable o diagonalizable.

Para ello, calculemos algunas conjugaciones. La primera conjugación nos permite ``intercambiar'' los elementos de una matriz.
\begin{equation}
  \begin{pmatrix} 0 & 1 \\ 1 & 0 \end{pmatrix}
  \begin{pmatrix} a & b \\ c & d \end{pmatrix}
  \begin{pmatrix} 0 & 1 \\ 1 & 0 \end{pmatrix}
    = \begin{pmatrix} d & c \\ b & a \end{pmatrix}. \label{eq:ConjI}
\end{equation}
Notemos que si $c = 0$ entonces la matriz es triangular superior, y aplicando esta conjugación obtenemos que es semejante a una matriz triangular inferior. Aplicando una conjugación similar es posible demostrar que toda matriz semejante a una matriz triangular superior es semejante a una matriz triangular inferior y viceversa. De este modo, para que una matriz sea triangularizable no importa si es semejante a una matriz triangular inferior o superior.

Para la siguiente conjugación, tomemos a cualquier $s \in \C$ y veamos que
\begin{equation}
  \begin{pmatrix} 1 & 0 \\ s & 1 \end{pmatrix}
  \begin{pmatrix} a & b \\ c & d \end{pmatrix}
  \begin{pmatrix} 1 & 0 \\ -s & 1 \end{pmatrix}
    = \begin{pmatrix} -bs+a & b \\ -bs^2 + (a-d)s + c  & bs+d \end{pmatrix}. \label{eq:ConjII}
\end{equation}
Ya que estamos trabajando sobre $\C$, notemos que si $b \neq 0$ entonces existe $s \in \C$ tal que $-bs^2 + (a-d)s + c = 0$, por lo que la matriz sería semejante a una triangular superior. En el caso que $b = 0$, la matriz ya es triangularizable y por lo ya comentando en la conjugación anterior podemos concluir que toda matriz de $\M_2(\C)$ es semejante a una matriz triangular superior. En otras palabras
\[ 
    \begin{pmatrix} a & b \\ c & d \end{pmatrix} \sim \begin{pmatrix} \alpha & \beta \\ 0 & \gamma \end{pmatrix}.
\]

Para la siguiente conjugación, consideremos a $s \in \C$ y para cualquier matriz triangular superior veamos que
\begin{equation}
  \begin{pmatrix} 1 & s \\ 0 & 1 \end{pmatrix}
  \begin{pmatrix} \alpha & \beta \\ 0 & \gamma \end{pmatrix}
  \begin{pmatrix} 1 & -s \\ 0 & 1 \end{pmatrix}
    = \begin{pmatrix} \alpha & (\gamma-\alpha)s + \beta \\ 0 & \gamma \end{pmatrix}.
\end{equation}
Notemos que si $\alpha \neq \gamma$  entonces existe $s \in \C$ tal que $(\gamma-\alpha)s + \beta = 0$, de este modo tenemos que las matrices pueden ser semejantes a una matriz diagonal con elementos distintos en la diagonal, o una matriz triangular con elementos en la diagonal iguales. Es decir, si $\alpha, \beta, \gamma \in \C$ entonces
\[
  \begin{pmatrix} a & b \\ c & d \end{pmatrix} \sim \begin{pmatrix} \alpha & 0 \\ 0 & \gamma \end{pmatrix}
    \Eqor
    \begin{pmatrix} a & b \\ c & d \end{pmatrix} \sim \begin{pmatrix} \alpha & \beta \\ 0 & \alpha \end{pmatrix}.
\]
Ahora, en la segunda equivalencia, si $\beta \neq 0$ entonces notemos que

\begin{equation}
  \begin{pmatrix} 1 & 0 \\ 0 & \beta \end{pmatrix}
  \begin{pmatrix} \alpha & \beta \\ 0 & \alpha \end{pmatrix}
  \begin{pmatrix} 1 & 0 \\ 0 & \beta^{-1} \end{pmatrix}
    = \begin{pmatrix} \alpha & 1 \\ 0 & \alpha \end{pmatrix}.
\end{equation}

De esta forma, tenemos que toda matriz compleja de $2 \times 2$ es semejante a alguna de estas matrices
\begin{equation}
  \begin{pmatrix} a & b \\ c & d \end{pmatrix} \sim \begin{pmatrix} \alpha & 0 \\ 0 & \gamma \end{pmatrix}
    \Eqor
    \begin{pmatrix} a & b \\ c & d \end{pmatrix} \sim \begin{pmatrix} \alpha & 1 \\ 0 & \alpha \end{pmatrix}.
      \label{eq:M2Form}
\end{equation}
con $\alpha$ y $\gamma$ arbitrarios. Aún queda por determinar si hay repeticiones entre los tipos de matrices calculados. Para realizar el cálculo más fácil, definamos las siguientes matrices 
\[
  D_{\alpha,\beta} = \begin{pmatrix} \alpha & 0 \\ 0 & \beta \end{pmatrix}
     \Eqand
  T_\alpha = \begin{pmatrix} \alpha & 1 \\ 0 & \alpha \end{pmatrix}
\]

Primero notemos que si $M = P^{-1} N P$ entonces $M^2 = (P^{-1} N P)(P^{-1} N P) = P^{-1}N^2 P$, de igual forma $M^3 = (P^{-1}N^2 P) (P^{-1} N P) = P^{-1}N^3 P$, de manera inductiva se puede ver que para todo $k \in \N$ se cumple que
\[
  M^k = P^{-1} N^k P
\]
Usando esta propiedad, notemos que si $c_k M^k + \cdots + c_2 M^2 + c_1 M + c_0 I = \bec 0$ entonces
\begin{align*}
  \bec 0 &= c_k M^k + \cdots + c_2 M^2 + c_1 M + c_0 I \\
    &= c_k P^{-1}N^k P + \cdots + c_2 P^{-1}N^2 P + c_1 P^{-1}N P + c_0 I \\
    &= P^{-1} (c_k N^k + \cdots + c_2 N^2 + c_1 N + c_0 I) P.
\end{align*}
De aquí es claro que $c_k N^k + \cdots + c_2 N^2 + c_1 N + c_0 I = \bec 0$, lo que nos deja con la siguiente proposición.

\begin{prop} \label{prop:MPolySem}
  Sean $M, N \in \M_n(\F)$, si $M \sim N$ entonces $\sum_{i=0}^k c_k M^k = \bec 0$ si y solo si $\sum_{i=0}^k c_k N^k = \bec 0$, donde $M^0 = N^0= I_n$. \qed
\end{prop}

Primero, revisemos que las dos formas dadas en \eqref{eq:M2Form} a las que puede ser semejante una matriz son ajenas. Supongamos que existen $\alpha, \beta, \gamma \in \C$ tal que $ D_{\alpha, \beta} \sim T_\gamma $. Notemos que
\begin{align*}
  (T_\gamma - \gamma I)^2 
    &= \corch{ \begin{pmatrix} \gamma & 1 \\ 0 & \gamma \end{pmatrix} - \begin{pmatrix} \gamma & 0 \\ 0 & \gamma \end{pmatrix} }^2 
     = \begin{pmatrix} 0 & 1 \\ 0 & 0  \end{pmatrix}^2  \\
    &= \begin{pmatrix} 0 & 0 \\ 0 & 0  \end{pmatrix}
\end{align*}
y que $(T_\gamma - \gamma I)^2 = T_\gamma^2 - 2\gamma T_\gamma + \gamma^2 I$, de esta forma si $ D_{\alpha,\beta} \sim T_\gamma $, por la proposición \ref{prop:MPolySem} tenemos que
\begin{align*}
  \begin{pmatrix} 0 & 0 \\ 0 & 0  \end{pmatrix}
    &= D_{\alpha,\beta}^2 - 2\gamma D_{\alpha,\beta} + \gamma^2 I \\
    &= \begin{pmatrix} \alpha & 0 \\ 0 & \beta \end{pmatrix}^2 - 2\gamma \begin{pmatrix} \alpha & 0 \\ 0 & \beta \end{pmatrix} + \begin{pmatrix} \gamma^2 & 0 \\ 0 & \gamma^2 \end{pmatrix} \\
    &= \begin{pmatrix} \alpha^2 - 2\alpha\gamma + \gamma^2 & 0 \\ 0 & \beta^2 - 2\beta\gamma + \gamma^2 \end{pmatrix} \\
    &= \begin{pmatrix} (\alpha-\gamma)^2 & 0 \\ 0 & (\beta-\gamma)^2 \end{pmatrix} 
\end{align*}
de aquí tenemos que $\alpha = \beta = \gamma$, pero si esto se cumple, notemos que entonces
\[
  D_{\alpha,\beta} - \gamma I = \begin{pmatrix} \alpha & 0 \\ 0 & \beta \end{pmatrix} - \begin{pmatrix} \gamma & 0 \\ 0 & \gamma \end{pmatrix}
  = \begin{pmatrix} 0 & 0 \\ 0 & 0 \end{pmatrix}
\]
pero la proposición \ref{prop:MPolySem}, esto implicaría que $T_\gamma - \gamma I  = \bec 0$ y por tanto
\[
  \begin{pmatrix} 0 & 0 \\ 0 & 0 \end{pmatrix} = T_\gamma - \gamma I = \begin{pmatrix} \gamma & 1 \\ 0 & \gamma \end{pmatrix} - \begin{pmatrix} \gamma & 0 \\ 0 & \gamma \end{pmatrix} = \begin{pmatrix}
    0 & 1 \\ 0 & 0 \end{pmatrix},
\]
lo que claramente es una contradicción. De esta forma la clase de equivalencia de $D_{\alpha,\beta}$ es ajena a la clase de equivalencia de $T_\gamma$ para cualesquiera $\alpha,\beta,\gamma \in \C$.

Ahora, veamos qué pasa con las semejanza de matrices del mismo tipo. Supongamos que existen $\alpha, \beta, \gamma, \alpha', \beta', \gamma' \in \C$ tales que $D_{\alpha,\beta} \sim D_{\alpha',\beta'}$ y $T_\gamma \sim T_{\gamma'}$. Aplicando el mismo proceso, tenemos que $(D_{\alpha,\beta} - \alpha I)(D_{\alpha,\beta} - \beta I) = \bec 0$ y $(T_\gamma - \gamma I)^2 = \bec 0$, usando la proposición \ref{prop:MPolySem} sobre $D_{\alpha',\beta'}$ y $T_{\gamma'}$ y simplificando, se tiene que 
\[
  D_{\alpha,\beta} \sim D_{\alpha',\beta'} \iff \{\alpha,\beta\} = \{\alpha',\beta'\}
    \Eqand
  T_\gamma \sim T_{\gamma'} \iff \gamma = \gamma'.
\]

Resumiendo todo el proceso hecho hasta ahora, tenemos el siguiente teorema.
\begin{teor} \label{teor:TDMat2x2}
  Toda matriz $M \in \M_2(\C)$ es semejante a una matriz de la forma $D_{\alpha,\beta}$ o $T_\gamma$ con $\alpha, \beta, \gamma \in \C$. Además, $D_{\alpha,\beta} \nsim T_\gamma$ para cualesquiera $\alpha,\beta,\gamma \in \C$. Y por último, $D_{\alpha,\beta} \sim D_{\alpha',\beta'}$ si y solo si $\{\alpha,\beta\} = \{\alpha',\beta'\}$ y $T_\gamma \sim T_{\gamma'}$ si y solo si $\gamma = \gamma'$. \qed
\end{teor}

Este teorema nos dice dos propiedades importantes. En primer lugar toda matriz compleja de tamaño $2 \times 2$ es triangularizable o diagonalizable. En segundo lugar, toda la clase de equivalencia de una matriz está caracterizado por uno o dos valores, los cuales son los valores en la diagonal de la matriz semejante de la forma $D_{\alpha,\beta}$ o $T_\gamma$.

La existencia de estos valores no es una coincidencia, en el capítulo 2 generalizaremos el proceso hecho en esta sección y se podrá ver que estos números son llamados los \emph{valores propios} de la matriz.


\ExerciseSection

\begin{exerciselist}
  \item Demuestra que $M$ es semejante a una matriz triangular superior si y solo si es semejante a una matriz triangular inferior.
  
  \item Realiza los cálculos para demostrar que $D_{\alpha,\beta} \sim D_{\alpha',\beta'}$ si y solo si $\{\alpha,\beta\} = \{\alpha',\beta'\}$ y $T_\gamma \sim T_{\gamma'}$ si y solo si $\gamma = \gamma'$.

  \item Demuestra que si $M \sim D_{\alpha,\beta}$ con $\alpha,\beta \in \C$, entonces existen vectores no nulos $v, v' \in \C^2$ tales que $Mv = \alpha v$ y $Mv' = \beta v'$.
  
  \item Demuestra que si $M \sim T_\gamma$ con $\gamma \in \C$, entonces existe un vector no nulo $v\in \C^2$ tal que $Mv = \gamma v$.
  
  \item Usa la proposición \ref{prop:MPolySem} para demostrar que $M \sim D_{5, -5}$ donde
    \[ M = \begin{pmatrix}
      -1 & 2 \\
      12 & 1
    \end{pmatrix}. \]
  
  \item Usa la proposición \ref{prop:MPolySem} para demostrar que $M \sim T_{3}$ donde
  \[ M = \begin{pmatrix}
    4 & -1 \\
    1 & 2
  \end{pmatrix}. \]
\end{exerciselist}