\chapter{Matrices y transformaciones}

\section{Notación e ideas}

Antes de empezar con las matrices y transformaciones lineales, hay que recordar algunas definiciones

\begin{defi}
  Sea $G$ un conjunto no vacío, una \emph{operación binaria} de $G$ es cualquier función $\cdot \colon G \times G \to G$. Por convención tenemos que $\cdot(x,y) = x \cdot y$. Además, decimos que el par $(G, \cdot)$ es un \emph{grupo} si cumplen los siguientes axiomas
  \begin{enumerate}
    \item (Asociatividad) Para todos $x, y, z \in G$ se cumple que $x\cdot(y\cdot z) = (x\cdot y)\cdot z$.
    \item (Existencia del elemento neutro) Existe un elemento $e$ tal que para todo $x \in G$ se tiene que $x\cdot e = e\cdot x = x$.
    \item (Existencia del elemento inverso) Para todo $x \in G$ existe $x^{-1} \in G$ tal que $x\cdot x^{-1} = x^{-1}\cdot x = e$.
  \end{enumerate}

  Por ultimo, decimos que un grupo es \emph{abeliano} si para todos $x, y \in G$ se cumple que $x\cdot y = y\cdot x$ (conmutatividad).
\end{defi}

Por convención, cuando nos referiremos a un grupo, usualmente solo haremos mención al conjunto de este, por ejemplo, cuando nos referiremos al grupo $(G, \cdot)$ lo haremos solo como ``el grupo $G$''. De igual forma, siempre que hablemos de una ``multiplicación'' podemos omitir el $\cdot$, es decir $x \cdot y = xy$.

\begin{defi}
  Sea $K$ un conjunto no vacío con dos operaciones binarias $+$ (suma) y $\cdot$ (producto), decimos que la terna $(K, +, \cdot)$ es un \emph{campo} si cumple los siguientes axiomas
  \begin{enumerate}
    \item $(K,+)$ es un grupo abeliano. Al elemento neutro lo denotaremos como $0_K$ y al inverso de $x \in K$ lo denotaremos como $-x$.
    \item El la operación $\cdot$ es asociativa y conmutatividad sobre $K$.
    \item (Existencia de neutro multiplicativo) Existe un elemento $1_K \in K-\{0\}$ tal que para todo $x \in K$ se cumple que $1_K\cdot x = x\cdot 1_K = x$.
    \item (Existencia de inverso multiplicativo) Para todo $x \in K-\{0\}$ existe un elemento $ x^{-1} \in K$ tal que $x \cdot x^{-1} = x^{-1} \cdot x = 1$.
    \item (Distributividad) Para todos $x, y, z \in K$ se cumple que $x \cdot (y + z) = x \cdot y + x \cdot z$.
  \end{enumerate}
\end{defi}

De igual forma que con los grupos, por convención cuando nos referiremos al campo, usualmente solo haremos mención al conjunto de este, para la notación de este libro usaremos $\F$ para denotar un campo. Si no existe confusión, nos referiremos simplemente como 0 y 1 a los neutros aditivo y multiplicativo de un campo.

Existen varios ejemplos de campos, por ejemplo los racionales $\Q$, los reales $\R$, los complejos $\C$ y todos los campos asociados a la aritmética de $\Z$ modulo un primo $p$, los cuales denotamos como $\F_p$.

\begin{defi}
  Sea $V$ un conjunto no vacío con una operación binaria $+$ (suma), $(\F, \oplus, \otimes)$ un campo y sea $\cdot\colon K \times V \to V$ una operación llamada \emph{producto escalar}. Decimos que $V$ es un $\F$-espacio vectorial si cumple los siguientes axiomas
  \begin{enumerate}
    \item $(V, +)$ es un grupo abeliano.
    \item (Neutro escalar) Para todo $v \in V$ se cumple que $1_\F v = v$.
    \item (Asociatividad escalar) Para todo $\lambda, \mu \in \F$ y $v \in V$ se cumple que $\lambda (\mu v) = (\lambda \otimes \mu) v$
    \item (Primera ley distributiva) Para todo $\lambda \in \F$ y $v,w \in V$ se cumple que $\lambda(v + w) = \lambda v + \lambda w$.
    \item (Segunda ley distributiva) Para todo $\lambda, \mu \in \F$ y $v\in V$ se cumple que $(\lambda\oplus\mu)v = \lambda v + \mu v$.
  \end{enumerate}
\end{defi}

Por convención, si no existe confusión, a la suma del campo $\F$ y de $V$ lo denotaremos con el mismo símbolo, de igual forma que con el producto del campo y el producto escalar.

Existen varios ejemplos de espacio vectoriales. Uno de ellos es el conjunto de matrices de $n \times m $ con entradas en un campo $\F$, o el conjunto de polinomios $\F[x]$.

\begin{defi}
  Sea $S$ un subconjunto de un $\F$-espácio vectorial $V$, entonces al conjunto de combinaciones lineales de $S$ lo denotamos como
  \[ \inner{S} = \set{ \sum_{i=1}^n \lambda_i v_i : \lambda_i \in K,  v_i \in S, n \in \N  }. \]

  Decimos que $S$ genera a $V$ si $\inner{S} = V$. De igual forma, $S$ es linealmente dependiente, si existe una combinación lineal de $S$ tal que
  \[ \sum_{i=1}^n \lambda_i v_i = 0, \]
  donde para al menos algún $i \in \{1,\ldots,n\}$ se cumple que $\lambda_i \neq 0_\F$. En caso que $S$ no sea linealmente dependiente, diremos que es linealmente independiente.
\end{defi}

Una de las características más interesantes de los espacios vectoriales son las bases, conjuntos linealmente independiente tales que generan a todo el espacio.
\begin{defi}
  Sea $V$ un $\F$-espacio vectorial, decimos que $B$ es una base de $V$ si es linealmente independiente y genera a $V$.
\end{defi}

\begin{teor}
  Sea $V$ un $\F$-espacio vectorial, $V$ tiene al menos una base. Además si $B_1$ y $B_2$ son bases de $B$ entonces $\abs{B_1} = \abs{B_2}$.
\end{teor}

La características de que cada espacio vectorial tiene una base y que además la cardinalidad de estas son constante, nos permite definir la dimensión de un espacio vectorial.

\begin{defi}
  Sea $V$ un $\F$-espacio vectorial y entonces la $\F$-dimensión de $V$ es la cardinalidad en común de las bases de $V$ y lo denotamos como $\dim_\F(V)$.
\end{defi}

Uno de los conceptos más importantes para el estudio de los espacios vectoriales son las transformaciones lineales, es decir funciones que preservan la linealidad.

\begin{defi}
  Sean $V$ y $W$ dos $\F$-espacios vectorial, se dice que la función $T\colon V\to W$ es una transformación $\F$-lineal si para todo $v,w \in V$ y $\lambda \in \F$ se cumple que 
  \[ T(\lambda v + w) = \lambda T(v) + T(w).\]
  Si $T$ es biyectiva, entonces es un isomorfismo de $\F$-espacios vectoriales.
\end{defi}

Usando los teoremas de la base y de la dimensión, es posible demostrar que si $V$ es un $\F$-espacio vectorial de dimensión $n$, entonces $V$ es isomorfo a $\F^n$.