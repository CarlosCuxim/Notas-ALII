\chapter{Matrices y transformaciones lineales}

\section{Notación e ideas}

Antes de empezar con las matrices y transformaciones lineales, hay que recordar algunas definiciones importantes así como las estructuras con las que vamos a estar trabajando.

En primer lugar, recordemos que una \emph{estructura algebraica} no es más que la combinación de un conjunto, funciones o relaciones asociadas a éste y propiedades relacionadas a estas últimas.

Las funciones pueden ser de varios tipos, pero la más común son las llamadas \emph{operaciones binarias}, que no son más que que funciones que toman como entrada dos elementos de un conjunto y retornan un elemento del mismo conjunto. Por ejemplo, si consideramos el conjunto $G$, entonces una operación binaria es cualquier función $\cdot \colon  G \times G \to G$. Por convención, para las operaciones binarias escribimos $x \cdot y$  para referirnos a $\cdot(a,b)$. Con esto en mente podemos definir las estructuras importante para el curso.

\subsection{Grupos}

\begin{defi}
  Si $G$ es un conjunto no vacío y $+$ una operación binaria de éste, decimos que el par $(G, \cdot)$ es un \emph{grupo} si cumplen los siguientes axiomas:
  \begin{enumerate}
    \item (Asociatividad) Para cualesquiera $x, y, z \in G$ se cumple que $x\cdot(y\cdot z) = (x\cdot y)\cdot z$.
    \item (Existencia del elemento neutro) Existe un elemento $e$ tal que para todo $x \in G$ cumple que $x\cdot e = e\cdot x = x$.
    \item (Existencia del elemento inverso) Para todo $x \in G$ existe $x^{-1} \in G$ tal que $x\cdot x^{-1} = x^{-1}\cdot x = e$.
  \end{enumerate}

  Por último, decimos que un grupo es \emph{abeliano} si para cualesquiera $x,y \in G$ se cumple que $x\cdot y = y \cdot x$ (conmutatividad).
\end{defi}

Por convención, cuando nos referiremos a un grupo, usualmente solo haremos mención al conjunto de éste, por ejemplo, cuando nos referiremos al grupo $(G, \cdot)$ lo haremos solo como ``el grupo $G$''. De igual forma, siempre que hablemos de una ``multiplicación'' podemos omitir el símbolo $\cdot$, es decir $x \cdot y = xy$.

\begin{teor}
  Sea $G$ un grupo, las siguientes propiedades se satisfacen.
    \begin{enumerate}
      \item El elemento neutro es único.
      \item Para cada $a \in G$ el inverso de $a$ es único.
      \item Sean $a,b,c \in G$, si $ab=ac$ o $ba=ca$, entonces $b=c$.
    \end{enumerate}
\end{teor}

Existen muchos ejemplos de grupos, por ejemplo los enteros, racionales, reales y complejos con la suma. Otro ejemplo son las matrices de tamaño $n \times n$ con entradas en los reales junto con la suma de matrices. Estos dos ejemplos fueron de  grupos abelianos, un ejemplo de un grupo no abeliano son los conjuntos de funciones biyectivas, como los grupos de permutación, con la composición usual de funciones o el grupo lineal, conformado de las matrices invertibles con la multiplicación de matrices.

\subsection{Campos}

\begin{defi}
  Sea $K$ un conjunto no vacío con dos operaciones binarias $+$ (suma) y $\cdot$ (producto), decimos que la terna $(K, +, \cdot)$ es un \emph{campo} si cumple los siguientes axiomas
  \begin{enumerate}
    \item $(K,+)$ es un grupo abeliano. Al elemento neutro lo denotaremos como $0_K$ y al inverso de $x \in K$ lo denotaremos como $-x$.
    \item La operación $\cdot$ es asociativa y conmutativa sobre $K$.
    \item (Existencia de neutro multiplicativo) Existe un elemento $1_K \in K-\{0\}$ tal que para todo $x \in K$ se cumple que $1_K\cdot x = x\cdot 1_K = x$.
    \item (Existencia de inverso multiplicativo) Para todo $x \in K-\{0\}$ existe un elemento $ x^{-1} \in K$ tal que $x \cdot x^{-1} = x^{-1} \cdot x = 1$.
    \item (Distributividad) Para cualesquiera $x, y, z \in K$ se cumple que $x \cdot (y + z) = x \cdot y + x \cdot z$.
  \end{enumerate}
\end{defi}

De igual forma que con los grupos, por convención cuando nos referiremos al campo, usualmente solo haremos mención al conjunto de éste, para la notación de este libro usaremos $\F$ para denotar un campo. Si no existe confusión, nos referiremos simplemente como 0 y 1 a los neutros aditivo y multiplicativo de un campo.

Existen varios ejemplos de campos, por ejemplo los racionales $\Q$, los reales $\R$, los complejos $\C$ y todos los campos asociados a la aritmética de $\Z$ modulo un primo $p$, los cuales denotamos como $\F_p$.

\subsection{Espacio vectoriales}

\begin{defi}
  Sea $V$ un conjunto no vacío con una operación binaria $+$ (suma), $(\F, +_\F, \cdot_\F )$ un campo y sea $\cdot\colon K \times V \to V$ una operación llamada \emph{producto escalar}. Decimos que $V$ es un $\F$-espacio vectorial si cumple los siguientes axiomas:
  \begin{enumerate}
    \item $(V, +)$ es un grupo abeliano.
    \item (Neutro escalar) Para todo $v \in V$ se cumple que $1_\F \cdot v = v$.
    \item (Asociatividad escalar) Para todo $\lambda, \mu \in \F$ y $v \in V$ se cumple que $\lambda \cdot (\mu \cdot v) = (\lambda \cdot_\F \mu)  \cdot v$.
    \item (Primera ley distributiva) Para todo $\lambda \in \F$ y $v,w \in V$ se cumple que $\lambda \cdot (v + w) = \lambda \cdot v + \lambda \cdot w$.
    \item (Segunda ley distributiva) Para todo $\lambda, \mu \in \F$ y $v\in V$ se cumple que $(\lambda +_\F \mu)\cdot v = \lambda \cdot v + \mu \cdot v$.
  \end{enumerate}
\end{defi}


Por convención, a los elementos de $V$ les llamaremos vectores y a los del campo, escalares. De igual forma y si no existe confusión, a partir de ahora a la suma del campo $\F$ y de $V$ lo denotaremos con el mismo símbolo. Para el producto del campo y el producto escalar, por convención a partir de ahora se omitirá el símbolo $\cdot$, de esta forma $\lambda v$ hará referencia a la multiplicación del escalar $\lambda$ con el vector $v$.

Existen varios ejemplos de espacios vectoriales. Uno de ellos es el conjunto de matrices de $n \times m $ con entradas en un campo $\F$, o el anillo de polinomios $\F[x]$.

\subsection{Subespacios vectoriales}

\begin{defi}
  Sea $V$ un $\F$-espacio vectorial. Decimos que $W \subseteq V$ es un $\F$-subespacio vectorial de $V$, si al restringir la suma y producto escalar de $V$ a $W$ se cumple que $W$ es un $\F$-espacio vectorial.
\end{defi}

Un ejemplo común es el subespacio trivial $\{0\}$. Otro ejemplo es el conjunto generado por todas las combinaciones lineales de $S \subset W$,  al cual denotamos por
\[ \inner{S} = \set{ \sum_{i=1}^n \lambda_i v_i : \lambda_i \in \F, v_i \in S, n \in \N }. \]
Además, este conjunto cumple que es el subespacio vectorial más pequeño tal que contiene a $S$, es decir que si $W$ es un subespacio vectorial tal que $S \subseteq W$, entonces $\inner{S} \subseteq W$. En el caso en que $\inner{S} = V$, entonces decimos que $S$ es un \emph{conjunto generador}.

\begin{teor}
  Sean $V$ un $\F$-espacio vectorial, la intersección de cualquier colección de subespacios de $V$ es un subespacio de $V$.
\end{teor}

\subsection{Independencia lineal y bases}

\begin{defi}
  Sea $S \subset W$, decimos que $S$ es linealmente dependiente, si existe una combinación lineal de $S$ tal que
  \[ \sum_{i=1}^n \lambda_i v_i = 0, \]
  donde para al menos algún $i \in \{1,\ldots,n\}$ se cumple que $\lambda_i \neq 0_\F$. En caso que $S$ no sea linealmente dependiente, diremos que es linealmente independiente.

  Dado $B \subset V$ diremos que $B$ es una \emph{base} de $V$ si es un conjunto generador linealmente independiente.
\end{defi}

Las bases son conjuntos muy importantes en los espacios vectoriales por que nos permiten generalizar el concepto de coordenadas. En primer lugar, todo conjunto linealmente independiente cumple que la expresión como combinación lineal de todo $v \in \inner{S}$ es única, de esta forma si $B$ es una base de cardinalidad $n \in \N$ de $V$, entonces para todo $v \in V$ existe una única combinación lineal tal que 
\[ v = \lambda_1 v_1 + \cdots + \lambda_n v_n, \qquad \lambda_1,\ldots,\lambda_n \in \F, v_1,\ldots,v_n \in V. \]
Esto nos permite crear una biyección entre los espacios $V$ y $\F^n$ mediante $B$, a la cual denotaremos como
\[ [v]_B = (\lambda_1, \ldots, \lambda_n)^t.\]

\begin{teor}
  Sea $V$ un $\F$-espacio vectorial, entonces $V$ tiene al menos una base. Además, si $B_1$ y $B_2$ son bases de $B$ entonces $\abs{B_1} = \abs{B_2}$.
\end{teor}

Este teorema es uno de los resultados más importantes del Álgebra Lineal, éste nos permite definir la \emph{dimensión} de un $\F$-espacio vectorial $V$ como la cardinalidad de cualquiera de las bases, es decir, si $B$ es una base de $V$, entonces
\[ \dim_\F (V) = \abs{B}. \]


\subsection{Suma y suma directa}

\begin{defi}
  Sean $W_1$ y $W_2$ dos $\F$-subespacios vectoriales de $V$, definimos la suma de $W_1$ con $W_2$ como
  \[ W_1 + W_2 = \{ w_1 + w_2 : w_1 \in W_1, w_2 \in W_2 \}.\]
  De manera general, si $W_1, \ldots, W_k$ son $\F$-subespacios vectoriales de $V$, entonces la suma de $W_1, \ldots, W_k$  la definimos como
  \[ W_1 + \cdots + W_k = \{ w_1 + \cdots + w_k : w_1 \in W_1, \ldots ,w_k \in W_k \}.\]
\end{defi}

Recordemos que la suma de subespacios vectoriales genera un nuevo subespacio vectorial que además es el subespacio generado por la unión de estos, es decir
  \[ W_1 + \cdots + W_k = \inner{ W_1 \cup \cdots \cup W_k }. \]
Hay una expresión bien conocida para la dimensión de la suma de dos subespacio vectoriales, esta fórmula es conocida como la fórmula de Grassmann.

\begin{teor}[fórmula de Grassmann]
  Si $W_1$ y $W_2$ son subespacios de un $\F$-espacio vectorial $V$ de dimensión finita, entonces
    \[ \dim_\F (W_1 + W_2) = \dim_\F (W_1) + \dim_\F (W_2) - \dim_\F (W_1 \cap W_2).\]
\end{teor}


Así como en los espacios vectoriales existe la independencia de vectores, podemos definir una independencia de subespacios.

\begin{defi}
  Si $W_1, \ldots, W_k$ son $\F$-subespacios vectoriales de $V$ y $W = W_1 + \cdots + W_k$, decimos que  $W_1, \ldots, W_k$ son independientes si $w_1 + \cdots + w_k = 0$ con $w_i \in W_i$ implica que $w_i = 0$, donde $i \in \{1,\ldots,k\}$.
\end{defi}

Así como la expresión de un vector como combinación lineal de un conjunto linealmente independiente es única, la independencia de subespacios nos permite afirmar que para todo $v \in W_1 + \cdots + W_k$ la expresión $v = w_1 + \cdots + w_k$, con $w_i \in W_i$ y $i \in \{1, \ldots, k\}$, es única.
\begin{teor}
  Sean $W_1, \ldots, W_k$ subespacios de un $\F$-espacio vectorial $V$ de dimensión finita y $W = W_1 + \cdots + W_k$, entonces las siguientes propiedades son equivalentes:
  \begin{enumerate}
    \item $W_1, \ldots, W_k$ son independientes.
    \item Para todo $i \in \{2, \ldots, k\}$ se tiene que
            \[ W_i \cap (W_1 + \cdots + W_{i-1})  = \{0\}. \]
    \item Si $B_i$ es una base de $W_i$ con $i \in \{1, \ldots, k\}$, entonces $B = B_1 \cup  \cdots \cup  B_k$ es una base de $W$.
  \end{enumerate}
\end{teor}

\begin{defi}
  Si $W_1, \ldots, W_k$ son $\F$-subespacios vectoriales independientes se dice que la suma $W = W_1 + \cdots + W_k$ es \emph{directa} o que $W$ es la \emph{suma directa} de  $W_1, \ldots, W_k$ y se denotará como:
    \[ W =  W_1 \oplus \cdots \oplus W_k.\]
\end{defi}

Notemos que si escribimos $W_1 \oplus \cdots \oplus W_k$, entonces por definición entendemos que los subespacios $W_1, \ldots, W_k$ son independientes. De igual forma, por la fórmula de Grassmann, se puede comprobar que en este caso
\[ \dim_\F (W_1 \oplus \cdots \oplus W_k) = \dim_\F (W_1) + \cdots + \dim_\F (W_k).\]

\subsection{Transformaciones lineales}

\begin{defi}
  Sean $V$ y $W$ dos $\F$-espacios vectoriales, se dice que la función $T\colon V\to W$ es una transformación $\F$-lineal si para todo $v,w \in V$ y $\lambda \in \F$ se cumple que 
  \[ T(\lambda v + w) = \lambda T(v) + T(w).\]

  Si $T$ es biyectiva, entonces es un isomorfismo de $\F$-espacios vectoriales. En este caso decimos que $V$ y $W$ son isomorfos y lo denotamos como $V \cong W$.
\end{defi}

Al conjunto de transformaciones lineales de $V$ a $W$ se le denota como $L(V,W)$. En el caso que $W = V$ entonces $T$ se dice que es un \emph{endomorfismo} de $V$ y al conjunto de endomorfismos de lo denotamos simplemente como $L(V)$.

\begin{teor}
  Sea $T\colon V \to W$ una transformación $\F$-lineal y $S$ un subconjunto de $V$.
  \begin{enumerate}
    \item Si $T$ es inyectiva y $S$ es linealmente independiente, entonces $T(S)$ es linealmente independiente.
    \item Si $T$ es suprayectiva y $S$ es un conjunto generador, entonces $T(S)$ es un conjunto generador.
    \item Si $T$ es biyectiva y $S$ es una base, entonces $T(S)$ es una base.
    \item Si $T$ es inyectiva, entonces $\dim_\F (V) \leq \dim_\F(W)$.
    \item Si $T$ es suprayectiva, entonces $\dim_\F (V) \geq \dim_\F(W)$.
    \item Si $T$ es biyectiva, entonces $\dim_\F (V) = \dim_\F(W)$.
  \end{enumerate}
\end{teor}

Las transformaciones lineales nos permiten enlazar espacios vectoriales, así como sus propiedades. Un ejemplo son los isomorfismos, estos nos permiten asegurar que dos espacios vectores se comportan de la misma forma, por lo que en esencia son el mismo espacio vectorial.

\begin{defi}
  Sea $T\colon V \to W$ una transformación $\F$-lineal, entonces definimos el \emph{núcleo} o \emph{kernel}  de $T$ como
    \[ \ker(T) = \{ v \in V : T(v) = 0_W \}.\]
  De manera análoga, definimos la imagen de $T$ como el conjunto
    \[ \Im(T) = \{w \in W : w = T(v) \text{ para alguna } v \in V \}. \]
\end{defi}

Estos dos conjuntos tienen algunas propiedades importantes. La primera es que el núcleo y la imagen son subespacios de $V$ y $W$, respectivamente. Otra propiedad, es que $T$ será inyectiva si y solo si $\ker(T) = \{0_V\}$. Y por último, tenemos uno de los resultados más importantes de las transformaciones lineales.
\begin{teor}
  Sea $T \colon V \to W$ una transformación $\F$-lineal. Si $V$ es de dimensión finita, entonces
    \[  \dim_\F( \ker T )  + \dim_\F( \Im T) = \dim_\F (V). \]
\end{teor}




\ExerciseSection


\begin{exerciselist}
  \item Si $V$ es un $\F$-espacio vectorial, demuestre que $0_\F \cdot v = 0$ y $(-1_\F) \cdot v = -v$ para todo vector $v \in V$.

  \item El conjunto $\{0\}$ tiene una estructura canónica de $\F$-espacio vectorial para cada campo $\F$: indique cuál es y verifique los axiomas. A tal espacio vectorial se le denomina \emph{trivial}.
  
  \item Sea $\F$ un campo dado. Indique cuál es la estructura canónica de $\F$ como $\F$-espacio vectorial y verifique los axiomas.
  
  \item Indique, en el campo de los complejos, sus estructuras canónicas como $\C$-espacio vectorial, como $\R$-espacio vectorial y como $\Q$-espacio vectorial. Verifique los axiomas.
  
  \item Sea $A$ un anillo con unidad $1_A \neq 0_A$ , y sean $\F$ un campo y $\phi\colon \F \to A$ un homomorfismo de anillos unitarios (es decir que $\phi(1_\F) = 1_A$ ). Demuestre que via $\phi$ se cumple que $A$ es un $\F$-espacio vectorial.
  
  \item Sea $\F[x]$ el anillo de polinomios con coeficientes en $\F$. Demuestre que $\F[x]$ es un $\F$-espacio vectorial.
  
  \item \label{exer:F^I} Sean $I$ un conjunto no vacío y $\F$ un campo dado. Denotemos como $\F^I$ al conjunto de las funciones $f\colon I \to \F$. Definimos la suma en $\F^I$ de la siguiente manera: dados $f, g \in \F^I$ se establece que $(f + g)(i) = f(i) + g(i)$. Definimos el producto por un escalar de la siguiente manera: dado $f \in \F^I$ y $c \in \F$, establecemos que $(c \cdot f )(i) = c \bigl( f(i) \bigr)$. Demuestre que con las operaciones definidas $\F^I$ tiene una estructura de $\F$-espacio vectorial.
  
  \item Considere a los números complejos. Dado $\alpha \in \C$ demuestre que $\alpha \R$ es un $\R$-subespacio vectorial de $\C$, pero que no es un $\C$-espacio vectorial.
  
  \item \label{exer:PolyGradLeqn} Denotemos como $\F[x]_{\leq n}$ al conjunto de polinomios con coeficientes en $\F$ y que tienen grado menor o igual a $n$. Demuestre que $\F[x]_{\leq n}$ es un $\F$-subespacio vectorial de $\F[x]$.
  
  \item Sea $U$ un $\F$-subespacio vectorial de $V$, demuestre que $0 \in U$.
  
  \item Sea $U$ un subconjunto no vacío del $\F$-espacio vectorial $V$. Demuestre que $U$ es un $\F$-subespacio vectorial si y solo si se cumplen las siguientes dos condiciones:
    \begin{enumerate}
      \item $\F U \subseteq \F[U]$, donde $\F U = \{cu : c \in \F \land u \in U\}$.
      \item $(U + U) \subseteq U$, donde $U+U = \{u_1 + u_2 \colon u_1, u_2 \in U\}$.
    \end{enumerate}

  \item Sea $U$ un $\F$-subespacio vectorial de $V$ y sea $W$ un $\F$-subespacio vectorial de $U$, demuestre que $W$ es un $\F$-subespacio vectorial de $V$.
  
  \item \label{exer:MatULD} Considere al $\F$-espacio vectorial $\M_{n,n} ( \F )$, al que denotaremos (solo un subíndice) $\M_n(\F)$. Sea $\bec U$ el subconjunto de todas las matrices triangulares superiores de $\M_n(\F)$, es decir que si $(a_{ij}) \in \bec U$ entonces $i > j$ implica que $a_{ij} = 0$. Similarmente sea $\bec L$ el subconjunto de todas las matrices triangulares inferiores de $\M_n(\F)$, y con $\bec D$ denotemos a las matrices diagonales. Demuestre que $\bec U$, $\bec L$ y $\bec D$ son $\F$-subespacios vectoriales de $\M_n(\F)$.
  
  \item Sea $V$ un $\F$-espacio vectorial, y sea $U_i$ un $\F$-subespacio vectorial de $V$ para cada $i \in I$, donde $I$ es un conjunto de índices no vacío. Demuestre que $U = \bigcap_{i \in I} U_i$ es un $\F$-subespacio vectorial de $V$ y un $\F$-subespacio vectorial de cada $U_i$.
  
  \item Sea $S$ un subconjunto del $\F$-espacio vectorial $V$, y denote por $\inner{S}$ a la intersección de todos los $\F$-subespacios vectoriales de $V$ que contienen a $S$. Por el ejercicio previo sabemos que $\inner{S}$ es un $\F$-subespacio vectorial de $V$, y notemos que si $S$ es el conjunto vacío entonces $\inner{S} = \{ 0 \}.$ Demuestre que $\inner{S}$ es el mínimo $\F$-subespacio vectorial de $V$ que contiene a $S$.
  
  \item Sea $S$ un subconjunto no vacío del $\F$-espacio vectorial $V$. Demuestre que para cualquier $v \in \inner{S}$ existen algún entero $n$, vectores $s_1, s_2, \ldots, s_n \in S$ y coeficientes $c_1, c_2, \ldots, c_n \in \F$ tales que $v = \sum _{i=1}^n c_i s_i$. Es por la propiedad mencionada que $\inner{S}$ es llamado el \emph{$\F$-subespacio vectorial generado por $S$.}
  

  \item Consideremos los $\F$-subespacios vectoriales $U_1$ y $U_2$ de $V$. Demuestre que $U_1 + U_2$ es un $\F$-subespacio vectorial de $V$, donde $U_1 + U_2 = \{ u_1 + u_2 : u_1 \in U_1 \land u_2 \in U_2 \}.$
  
  \item \label{exer:F^IFinito} Sea $\F^{I}$ como se mencionó en el ejercicio \ref{exer:F^I}, y sea $\F^{(I)}$ el subconjunto de funciones que a evaluarse son cero en casi todos sus valores, es decir que si $f \in \F^{(I)}$ entonces $f(i) \neq 0_\F$ solo para una cantidad finitas de elementos $i \in I$. Demuestre que $\F^{(I)}$ es un $\F$-subespacio vectorial de $\F^{I}$.
  
  \item Determine $\dim_\F(\F)$; justifique su respuesta.
  
  \item Sea $\F[x]_{\leq n}$ como se mencionó en el ejercicio \ref{exer:PolyGradLeqn}: determine su dimensión exhibiendo una base.
  
  \item El campo de los complejos tiene una estructura canónica como $\C$-espacio vectorial, como $\R$-espacio vectorial
  y como $\Q$-espacio vectorial. Determine la dimensión de $\C$ como $\C$-espacio vectorial y como $\R$-espacio vectorial exhibiendo bases correspondientes a cada caso.
  
  \emph{Reto:} Demuestre que la dimensión de $\C$ como $\Q$-espacio vectorial no es finita.
  
  \item Determine $\dim_{\F}\bigl( \M_{n,m}(\F) \bigr)$.
  
  \item Sean $\M_n(\F)$, $\bec U$, $\bec L$ y $\bec D$ como se mencionó en el ejercicio \ref{exer:MatULD}. Calcule la dimensión de cada uno de los $\F$-subespacios vectoriales.

  \item Considere el contexto del ejercicio previo. Demuestre que $\bec U + \bec L = \M_n(\F)$.
  
  \item Sea $\bec U_e$ el $\F$-subespacio vectorial de las matrices triangulares estrictamente superiores, es decir que si  $(a_{ij}) \in \bec U_e$ entonces $i \geq j$ implica $a_{ij} = 0$ y similarmente $\bec L_e$ denota al $\F$-subespacio vectorial de las matrices triangulares estrictamente inferiores, demuestre que $\bec U_e \oplus \bec D \oplus \bec L_e = \M_n(\F)$.
  
  \item Sea $V$ un $\F$-espacio vectorial de dimensión finita y sean $U_1, U_2, \ldots, U_k$ un conjunto de $\F$-subespacios vectoriales de $V$. Supongamos que $V = U_1 \oplus U_2 \oplus \cdots \oplus U_k$; demuestre que $\dim_\F(V) = \sum _{i=1}^k \dim_\F(U_i)$.
  
  \item Provea un ejemplo de la siguiente situación: $U_1$, $U_2$, $U_3$ son $\F$-subespacios vectoriales de $V$ tales que $i \neq j$ implica $U_i \cap U_j = \{0\}$, pero $U_1 + U_2 + U_3$ no es una suma directa.
  
  \item Sea $\F^I$ como se mencionó en el ejercicio \ref{exer:F^I}: para el caso $I$ finito exhiba alguna base y determine su dimensión.
  
  \item Sea $\F^{(I)}$ como se mencionó en el ejercicio \ref{exer:F^IFinito}: exhiba alguna base y determine su dimensión (para cualquier $I$).
  
  \item Sea $f \colon \N \to \Q$ una biyección. Considere los siguientes subconjuntos de $\Q$: $G_1 = \Q - \{ f(1)\}$, $G_2 = \{ f(1), f(2) \}$, \dots, $G_n = \Q - \{ f(1), f(2), \ldots, f(n) \}$, etc. Demuestre que cada $G_i$ es un conjunto generador de $\Q$ como $\Q$-espacio vectorial. Demuestre que $\bigcap _{i \in \mathbb{N}} G_i = \emptyset$.
  
  \item Sea $G$ un subconjunto generador finito del $\F$-espacio vectorial $V \neq \{0\}$ (es decir que $V$ es no trivial). Demuestre que existe algún subconjunto de $G$ que es una base de $V$.
  
  \item Sea $T\colon V \to W$ una transformación $\F$-lineal.
    \begin{enumerate}
      \item Demuestre que $\ker(T)$ es un $\F$-subespacio vectorial de $V$.
      \item Demuestre que $T$ es inyectiva si y solo si $\ker(T) = \{0\}$.
      \item Demuestre que $\Im(T)$ es un $\F$-subespacio vectorial de $W$.
    \end{enumerate}

  \item Sea $T\colon V \to W$ una transformación $\F$-lineal y sean $U_1$ y $U_2$ $\F$-subespacios vectoriales de $V$. Demuestre que:
    \begin{enumerate}
      \item $T(U_1 + U_2) = T(U_1) + T(U_2)$.
      \item Supongamos que $T$ es inyectiva y que $U_1 \oplus U_2$. Entonces $T(U_1) \oplus T(U_2)$.
    \end{enumerate}

  \item Sean $T_1\colon V \to W$ y $T_2\colon W \to U$ transformaciones $\F$-lineales. Demuestre que $T_2 \circ T_1 \colon V \to U$ es una transformación $\F$-lineal.
  
  \item Sea $T\colon V \to W$ una transformación $\F$-lineal biyectiva: demuestre que $T^{-1}\colon W \to V$ es una transformación $\F$-lineal biyectiva.
  
  \item Demuestra que $L(V,W)$ es un $\F$-espacio vectorial.
\end{exerciselist}