\section{El polinomio característico y minimal}

En el estudio de los valores y vectores propios de una matriz hay dos polinomio importantes asociados a cada matríz. Estos son el polinomio minimal, que nos dará una forma de encontrar los valores propios de una matriz. El otro polinomio nos será útil cuando estudiemos los espacios de núcleos.

\subsection{Polinomio característico.}

Para definir este polinomio, primero necesitamos recordar algunas de las propiedades del determinante. En primer lugar, el determinante no solo se puede definir sobre matrices con entradas en un campo $\F$, sino que es posible definirlo sobre cualquier anillo.

Además, algunas de las propiedades de la determinante, demostradas sobre campos, también se conservan para anillos. Las más importantes son que para cualesquiera matrices $M$ y $N$ con entradas sobre un anillo, se cumple que $\det(AB) = \det(A)\det(B)$ y $A \adj(A) = \adj(A) A = \det(A) I$, donde $\adj(A)$ es traspuesta de la matriz de cofactores de $A$. Las demostraciones de estas propiedades va más allá de el alcance de este libro, pero su demostración es muy similar a cuando las entradas son elementos de un campo. Con todo esto, podemos definir el primer polinomio importante de una matriz.

\begin{defi}
  Sea $M \in \M_n(\F)$, considerando la matriz con entradas en $\F[x]$ dada por $M-xI$, entonces llamaremos el \emph{polinomio característico} de $M$ al polinomio
    \[ p(x) = \det(M-xI). \]
\end{defi}

Ahora, que propiedades tiene este polinomio. Supongamos que tenemos una raíz $\lambda \in \F$ del polinomio, entonces $p(\lambda) = 0$ , pero eso implica que
\[ p\lambda) = \det(M-\lambda I) = 0, \]
usando el teorema \ref{teor:PropVP} tenemos que entonces $\lambda \in E(M)$. Análogamente se puede probar que si $\lambda \in E(M)$ entonces $\lambda$ será una raíz del polinomio característico de $M$.

Otra propiedad interesante, es que si $M = P^{-1} N P$ entonces, por propiedades de la determinante tenemos que
\begin{align*}
  \det(M - xI) &= \det(P^{-1}NP - x P^{-1}IP) \\
    &= \det\bigl(P^{-1}(N- xI)P\bigr) \\
    &= \det(P^{-1})\det(N - xI) \det(P) \\
    &= \det(N - xI) \det(P) [\det(P)]^{-1} \\
    &= \det(N - xI).
\end{align*}
De esta forma, dos matrices semejantes comparten el mismo polinomio característico. Así podemos formular el siguiente teorema.

\begin{teor}
  Sean $M, N \in \M_n(\F)$ tal que $M \sim N$ entonces las siguientes propiedades se sostienen.
  \begin{enumerate}
    \item $\lambda \in E(M)$ si y solo si es raíz de su polinomio característico.
    \item Si $M \sim N$ entonces tienen el mismo polinomio característico.
  \end{enumerate}
\end{teor}



\subsection{Polinomio minimal}

Este el el segundo polinomio importante de una matriz. Aquí buscamos cual es el polinomio no nulo $f$ de grado más pequeño tal que anula a una matriz, es decir $f(M) = \bec 0$.

En primer lugar, notemos que debe existir, esto se cumple dado que si consideramos el conjunto $\{I,M,M^2,\ldots,M^{n^2}\}$, este es un conjunto con $n^2+1$ elementos del $\F$-espacio vectorial $M_n(\F)$ que tiene dimensión $n^2$. Así, este conjunto no puede ser linealmente independiente, por lo que existen $c_0,\ldots,c_n\in \F$ tal que 
\[ c_n M^n + \cdots + c_2 M^2 + c_1 M + c_0 I = \bec 0. \]
De esta forma, es claro que si $f = c_n x^n + \cdots + c_2 x^2 + c_1 x + c_0$ entonces $f$ no es el polinomio cero y además $f(M) = \bec 0$. Ahora, mostremos que existe este polinomio de grado mínimo que anula la matriz.

\begin{teor}\label{teor:UnicidadPMin}
  Sea $M \in M_n(\F)$. Existe un único polinomio $f_M \in \F[x]$, de grado mayor a cero tal que
  \begin{enumerate}
    \item $f_M(M) = \bec 0$.
    \item El coeficiente líder de $f_M$ es la unidad $1_\F$.
    \item Si $g \in \F[x]$ cumple que $g(M) = 0$, entonces $g = 0$ o $\grad(g) \geq \grad(f)$. Más aún, $f_M$ divide a $g$.
  \end{enumerate}
\end{teor}
\begin{proof}
  Por lo ya visto, sabemos que existe al menos un polinomio no nulo que anula la matriz, así consideremos el conjunto $I$ de todos los polinomios no nulos tal que anulan a la matriz. Así, consideremos algún polinomio $f \in I$ de grado mínimo.

  En primer lugar, si $f$ tiene coeficiente líder $c$, entonces es fácil ver que que si $f(M) = \bec 0$ entonces $\frac{1}{c} f (M) = \bec 0$, así, el polinomio $f_M = \frac{1}{c} f$ anula a $M$, tiene grado mínimo y además tiene coeficiente líder $1$.

  Ahora probemos que se cumple la tercera propiedad. Sea $g \in I$, por el algoritmo de la división, tenemos que existen polinomios $q,r \in \F[x]$ tal que $g = qf_M+r$ y $\grad(f) > \grad(r)$. De esta forma, por la proposición \ref{prop:EvalPoly} y recordando que $g(M) = \bec 0$, entonces tenemos que
  \[
    g(M) = q(M) f_M(M) + r(M) = r(M) = \bec 0.
  \]
  Ahora, si $r \neq 0$ entonces eso implica que $r \in I$, pero como $\grad(r) < \grad(f_M)$, eso contradice que $f_M$ tenga grado mínimo. Así tenemos que $r = 0$, pero entonces $f$ divide a todos los elementos de $I$.
  
  Así, si existe otro polinomio $f'$ de mismo grado que $f_M$ en $I$ con coeficiente líder $1$, este debe ser un múltiplo escalar de $f_M$, pero como ambos tienen el mismo coeficiente líder, entonces la única posibilidad, es que $f_M = f'$. Así tenemos que $f_M$ es el único polinomio de grado mínimo con coeficiente líder $1$ tal que anula a $M$.
\end{proof}

\begin{defi}
  Sea $M \in \M_n(\F)$ llamaremos el \emph{polinomio minimal} de $M$ al único polinomio $f_M$ de grado mínimo con coeficiente líder $1$ tal que $f_M(M) = \bec 0$.
\end{defi}

El polinomio minimal de una matriz nos ayudará a definir luego unos espacios vectoriales especiales, por lo que detallaremos más adelante sus características más importantes. Pero hay 3 propiedades interesante a detallar, la primera es que si $M \sim N$ entonces comparten el mismo polinomio minimal.

\begin{prop}\label{prop:InvMinPoly}
  Sea $M,N \in \M_n(\F)$ si $M \sim N$ entonces tienen el mismo polinomio minimal.
\end{prop}
\begin{proof}
  Notemos que si $f_M(M) = \bec 0$ y $f_N(N) = \bec 0$, entonces por la proposición \ref{prop:MPolySem} se debe cumplir que $f_N(M) = \bec 0$ y $f_N(M) = \bec 0$, pero eso implica, por el teorema \ref{teor:UnicidadPMin} que $f_M \mid f_N$ y $f_N \mid f_M$.

  Así, por propiedad de los polinomio, entonces debe existir una constante $c \in \F$ tal que $f_M = c f_N$, pero como $f_M$ y $f_N$ tiene el mismo coeficiente líder $1$, entonces tenemos que $c = 1$ y por tanto $f_M = f_N$.
\end{proof}