\section{Espacios de vectores propios.}

En esta sección abundaremos un poco más en el conjunto de vectores propios asociados a un mismo valor propio.

\begin{defi}
  Sea $M \in \M_n(\F)$ y $\lambda \in E(M)$, al conjunto $V_\lambda = \{v\in\F : Mv = \lambda v\}$ le llamaremos el \emph{espacio propio} de $\lambda$.
\end{defi}

Notemos que el espacio propio de un valor propio están compuesto por todos los vectores propios asociados a este y el vector cero. Además, este conjunto, como su nombre lo indica, es un subespacio vectorial de $\F^n$. Esto se puede comprobar ya que para cualesquiera $v,w \in V_\lambda$ y $c \in \F$ se cumple que $v+cw \in V_\lambda$, ya que
  \[
    (v+cw)M = Mv + cMw = \lambda v + c \lambda w = \lambda (v+cw).
  \]
\begin{prop}
  Sea $M \in \M_n(\F)$ y $\lambda \in E(M)$, entonces el espacio propio de $\lambda$ es un subespacio vectorial de $\F^n$. \qed
\end{prop}

Los espacios propios no solamente son espacios vectoriales, como se se mencionó con anterioridad, estos espacios son independientes entre sí.

\begin{teor}
  Sea $M \in \M_n(\F)$ y $E(M) = \{\lambda_1, \ldots, \lambda_k\}$, entonces $V_{\lambda_1} \oplus \cdots \oplus V_{\lambda_k}$.
\end{teor}
\begin{proof}
  Por inducción, probemos que para todo $1\leq m \leq k$, si $v_1 + \cdots + v_m = \bec 0$, donde $v_i \in V_{\lambda_i}$ para cada $i\in\{1,\ldots,m\}$, entonces $v_1 = \cdots = v_m = \bec 0$. Es claro que si $m = 1$, entonces $v_1 = 0$. Así, supongamos que existe $1 \leq s < n$ tal que se sostiene la propiedad.

  Sea $v_i \in V_{\lambda_i}$ para cada $i\in\{1,\ldots,s+1\}$ tal que $v_1 + \cdots+ v_{s+1} = \bec 0$, por definición tenemos que
  \begin{align*}
    \bec 0 &= M\bec 0 = M(v_1 + \cdots+ v_{s+1}) \\
      &= Mv_1 + \cdots+ Mv_{s+1} \\
      &= \lambda_1v_1 + \cdots+ \lambda_{s+1}v_{s+1}.
  \end{align*}
  De esta forma, dado que $\lambda_{s+1} v_1 + \cdots+ \lambda_{s+1} v_{s+1} = \lambda_{s+1}\bec 0 = \bec 0$, entonces tenemos que 
    \begin{align*}
      \bec 0 &= (\lambda_1v_1 + \cdots+ \lambda_{s+1}v_{s+1}) - (\lambda_{s+1} v_1 + \cdots+ \lambda_{s+1} v_{s+1}) \\
        &= (\lambda_1 - \lambda_{s+1})v_1 + \cdots + (\lambda_s - \lambda_{s+1})v_s.
    \end{align*}
  
    Ahora, notemos que $(\lambda_i - \lambda_{s+1})v_{i} \in V_{\lambda_i}$ para cada $i \in \{1,\ldots,s\}$, de este modo, por hipótesis de inducción, tenemos que
      \[ (\lambda_1 - \lambda_{s+1})v_1 = \cdots = (\lambda_s - \lambda_{s+1})v_s = \bec 0.\]
    Ahora, como  $\lambda_i \neq \lambda_{s+1}$, para todo $i \in \{1,\ldots,s\}$, por construcción, entonces podemos concluir que $v_1 = \cdots = v_s = \bec 0$. Finalmente, por lo ya demostrado, tenemos que 
      \[ \bec 0 = v_1 + \cdots+ v_s + v_{s+1} = v_{s+1}, \]
    mostrando así que la propiedad se sostiene para $s+1$ y por tanto para todo $1 \leq m \leq k$.

    Ahora, tomando el caso particular $m = k$, notemos que si tomamos $v_i \in V_{\lambda_i}$ para cada $i \in \{1\ldots,k\}$ tal que $v_1 + \cdots + v_n$, entonces $v_1 = \cdots = v_k = \bec 0$, pero por definición, esto implica que $V_{\lambda_1}, \ldots, V_{\lambda_k}$ son independientes y por tanto $V_{\lambda_1} \oplus \cdots \oplus V_{\lambda_k}$.
\end{proof}