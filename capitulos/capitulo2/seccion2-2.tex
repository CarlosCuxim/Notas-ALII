\section{Álgebras y polinomios}

Antes de continuar revisando las propiedades de los valores y vectores propios, necesitamos conocer algunas de las principales propiedades de los polinomios, que serán necesarias para el desarrollo de temas posteriores.

\subsection{Propiedades básicas de los polinomios}

\begin{teor}
  Sea $\grad\colon \F[x] \to \{-\infty\} \cup \N_0$ la función grado dada por enviar al polinomio cero a $-\infty$ y al polinomio $c_n x^n + \cdots + c_1x + c_0$, con $c_n \neq 0$, en $n$. Se cumplen las siguientes propiedades
  \begin{enumerate}
    \item Dados $f, g \in \F[x]$ se cumple la identidad $\grad(fg) = \grad(f) + \grad(g)$.
    
    \item \emph{(Algoritmo de la división)} Dados $f \in \F[x]$ y $g \in \F[x] - \{0\}$ existen $q, r \in \F[x]$ tales que $f = gq+r$ y $\grad(r) < \grad(g)$.
    
    \item \emph{(Identidad de Bézout)} Dados $f_1, f_2, \ldots, f_k \in \F[x]$ existen $q_1, \ldots, q_k \in \F[x]$ tales que $\sum_{i=1}^k f_i q_i = d$ es un máximo común divisor (m.c.d) de los $f_i$, es decir que $d \mid f_i$ para cada $i \in \{1,\ldots,k\}$ y si $e_i \mid f_i$ para cada $i \in \{1,\ldots,k\}$, entonces $e \mid f$. Si bien puede haber más de un m.c.d., todos son asociados entre sí (es decir que dados dos m.c.d. uno se obtiene multiplicando alguna unidad por el otro).
    
    \item \emph{(Dominio de factorización única)} Todo ideal en $\F[x]$ es principal, es decir que si $I$ es un subconjunto no vacío de $\F[x]$ y para cualesquiera $f, g \in I$ y $h \in \F[x]$ se cumple que $f+g, -f, fh \in I$, entonces existe $p \in \F[x]$ tal que $I = p\F[x]$.
    
    \item Sean $f \in \F[x]$ tal que $\grad(f)>0$, entonces $f$ se puede factorizar como un producto de polinomios primos y tal factorización es única salvo orden y unidades. \qed
  \end{enumerate}
\end{teor}
 


\subsection{Anillos y álgebras}

Hasta ahora, hemos trabajado los polinomios $\F[x]$ únicamente como un $\F$-espacio vectorial, pero los polinomios tienen una estructura más compleja ya que, a diferencia de otros espacios vectoriales, además de una operación de suma los polinomios cuentan con una operación de multiplicación.

La única otra estructura con la que hayamos estado trabajando y que cuente con dos operaciones binarias es el campo, pero si revisamos la definición \ref{defi:campo}, notaremos que los polinomios no cumplen todas esas propiedades. En específico los polinomios no tienen inversos multiplicativos. Un ejemplo de esto es el polinomio $x$, no importa por que otro polinomio multipliques este nunca será $1$. En este sentido, la estructura de los polinomios es más general que la del campo, y recibe el nombre de anillo.

\begin{defi}
  Sea $A$ un conjunto no vacío con dos operaciones binarias $+$ (suma) y $*$ (producto), decimos que la terna $(A,+,*)$ es un \emph{anillo} si cumple los siguiente axiomas:
  \begin{enumerate}
    \item $(A,+)$ es un grupo abeliano. Al elemento neutro lo denotaremos como $0_A$ y al inverso de $x \in A$ lo denotaremos como $-x$.
    \item La operación $*$ es conmutativa sobre $A$.
    \item (Existencia de neutro multiplicativo) Existe un elemento $1_A \in A-\{0_A\}$ tal que para todo $x \in A$ se cumple que $1_A * x = x * 1_A = x$.
    \item (Distributividad) Para cualesquiera $x, y, z \in A$ se cumple que $x * (y + z) = x * y + x * z$ y $(y + z) * x =   y * x + z * x$.
  \end{enumerate}
  Si además, la operación $*$ es conmutativa sobre $A$, entonces decimos que es un \emph{anillo conmutativo}.
\end{defi}

Existen muchos ejemplos de anillos además de los polinomios, por ejemplo las matrices cuadradas $\M_n(\F)$, este anillo no es conmutativo, mientras que el de los polinomios si lo es. También, el conjunto de endomorfismos sobre un $\F$-espacio vectorial es un anillo no conmutativo. Por ultimo, tenemos el anillo modulo $n \in \N$, al que denotamos como $\Z_n$, el cual es un campo si y solo si $n$ es primo.

De los ejemplos de anillos ya mencionados, hemos visto que las matrices, los endomorfismos y los polinomio también tienen una estructura de $\F$-espacio vectorial y no solo eso, además, su multiplicación como anillo se comporta bien con su multiplicación escalar. Estos anillos especiales, que cuentan con una estructura mas rica, reciben el nombre de álgebras.

\begin{defi}
  Sea $(A,+,*)$ un anillo que tiene estructura de $\F$-espacio vectorial, decimos que es una $\F$-álgebra si para cualesquiera $x, y \in A$ y $\lambda, \mu \in \F$ se cumple que
    \[ (\lambda \cdot x) *  (\mu \cdot y) = (\lambda \mu) \cdot (a * b), \]
  donde $\cdot$ es la multiplicación escalar de $A$.
\end{defi}

Una pregunta común es cuándo un anillo puede obtener una estructura de $\F$-álgebra. Para contestar esta pregunta necesitamos algunas definiciones extra, pero la idea es que el campo $\F$ debe ser ``compatible'' con ciertos elementos del anillo.

En primer lugar, definimos el \emph{centro} del anillo como el conjunto $Z(A)$ de elementos del anillo que conmutan, es decir
  \[
    Z(A) = \{a \in A : a*x = x*a, \forall x \in A \}
  \]
Es claro que en todo anillo $1_A \in Z(A)$, además si $A$ es conmutativo, entonces $Z(A) = A$. De manera más general, $Z(A)$ siempre será un subanillo conmutativo de $A$.

Ahora, si $(A, +, *)$ y $(A', +', *')$ son anillos, decimos que la función $\phi\colon A \to A'$ es un \emph{homomorfismo} de anillos si para cualesquiera $a,b \in A$ se cumple que 
\[ \phi(a+b) = \phi(a) +' \phi(b)  \Eqand  \phi(a*b) = \phi(a) *' \phi(b). \]
Si además $\phi(1_A) = 1_{A'}$, entonces decimos que $\phi$ es un homomorfismo de anillos \emph{unitario}.

Ahora, para comprender cuándo un anillo se puede volver una álgebra, veamos qué sucede cuando $A$ es una $\F$-álgebra. Notemos que para cualesquiera $\lambda, \mu \in \F$, por la definición de $\F$-álgebra, se cumple que
  \[
    (\lambda +\mu) \cdot 1_A = \lambda \cdot 1_A + \mu \cdot 1_A,
      \qquad 
    (\lambda \mu) \cdot 1_A = \lambda \cdot 1_A * \mu \cdot 1_A,
      \Eqand
    1_\F \cdot 1_A = 1_A.
  \]
Además, para cualesquiera $x \in A$ y $\lambda \in \F$ se cumple que $x * (\lambda \cdot 1_A) = \lambda \cdot x = (\lambda \cdot 1_A) * x$ y por lo tanto $\lambda \cdot 1_A \in Z(A)$. De este modo, si definimos la función $\phi\colon \F\to Z(A)$ dada por $\phi(\lambda) = \lambda \cdot 1_A$, entonces $\phi$ es un homomorfismo de anillos unitario.

Ahora, qué pasa en el caso contrario, supongamos que existe un homomorfismo de anillos unitario $\phi\colon \F \to Z(A)$, si definimos el producto escalar $\cdot \colon \F\times A \to A$ como $\lambda \cdot x = \phi(\lambda) * x$, es fácil ver, bajo este producto escalar, que $A$ es un $\F$-álgebra. De este modo podemos concluir el siguiente teorema.

\begin{teor}
  Sea $A$ un anillo, $A$ será un $\F$-álgebra si y solo si existe un homomorfismo de anillos unitario $\phi\colon \F \to Z(A)$. \qed
\end{teor}

Ahora, notemos que no necesariamente existe un único homomorfismo de anillos unitario $\phi\colon\F\to Z(A)$, sino más bien, en un anillo cada uno de estos homomorfismo definirá una estructura de $\F$-álgebra diferente. De esta forma, para cada $\F$-álgebra, el homomorfismo de anillos unitario $\phi\colon\F\to Z(A)$ dado por $\phi(\lambda) = \lambda \cdot 1_A$ será el que lo determine.

Para facilitar la lectura, en lo que resta del capítulo y si no existe peligro de confusión, omitiremos los símbolos $\cdot$ y $*$ para referirnos a la multiplicación escalar y del anillo. De este modo, si $\lambda \in \F$ y $a$ entonces se entenderá que $\lambda a b = \lambda \cdot (a * b) $.



\subsection{Evaluación de un polinomio}

Ahora, recordemos que un polinomio no es una función como tal, pero existe una función asociada a partir de éste. Esto es lo que conocemos como evaluar el polinomio, y esta propiedad no es única del campo del polinomio, sino que se puede definir para cualquier $\F$-álgebra.

\begin{defi}
  Sea $A$ una $\F$-álgebra determinada por el homomorfismo de anillos unitario $\phi\colon \F\to Z(A)$ y sea $f \in \F[x]$ dado por $f = c_n x_n + \cdots+ c_1x + c_0$, entonces definimos la \emph{evaluación} de $f$ en $a \in A$ como
  \[
    f(a) = c_n a^n + \cdots + c_1 a + c_0 1_A
  \]
\end{defi}

Notemos que si $\phi\colon \F\to Z(A)$ es el homomorfismo que define a la estructura de $\F$-álgebra, por el análisis hecho anteriormente, entonces la evaluación de $f = c_n x_n + \cdots+ c_1x + c_0$ en $a \in A$ cumple que
  \[
    f(a) = \phi(c_n) a^n + \cdots + \phi(c_1) a + \phi(c_0).
  \]
De esta forma, dado que $\phi(c_i) \in Z(A)$ para todo $i \in \{1,\ldots, n\}$, haciendo los cálculos pertinentes, es fácil ver que para cualesquiera $f, g \in \F[x]$ y $a \in A$ se cumple que
\[
  (f+g)(a) = f(a) + g(a)
    \Eqand
  (fg)(a) = f(a) g(a).
\]
También, es fácil  comprobar que $f(1_\F) = 1_A$. Resumiendo, tenemos la siguiente proposición.

\begin{prop}
  Sea $A$ una $\F$-álgebra determinada por el homomorfismo de anillos $\phi\colon \F\to A$. Para cada $a \in A$, existe un único homomorfismo de anillos $\varphi_a \colon \F[x] \to A$ tal que $\varphi_a(c) = \phi(c)$ para cada $c \in \F$ y $\phi(x) = a$. Además, $\varphi_a(f) = f(a)$ para todo $f \in \F[x]$.
\end{prop}
\begin{proof}
  Ya hemos visto que $\varphi_a$ es un homomorfismo de anillos unitario, de esta forma solo queda demostrar la unicidad. Así, supongamos que existe otro homomorfismo $\psi\colon \F[x] \to A$ que cumple con las mismas propiedades, entonces para toda $f \in \F[x]$, si $f = c_n x_n + \cdots+ c_1x + c_0$ entonces
  \[
    \psi(f) = \psi\paren{\sum_{i=1}^n c_x^i} = \sum_{i=1}^n \psi(c_i) \psi(x^i) = \sum_{i=1}^n \phi(c_i) a^i = f(a) = \varphi_a(f).
  \]
  De esta forma, es claro que $\psi = \varphi_a$, mostrando así la unicidad.
\end{proof}