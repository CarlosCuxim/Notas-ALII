\section{La geometría de un espacio con producto interno}

Como mencionamos al principio, un producto interno nos ayudará a definir los conceptos de distancia y ángulo dentro de un espacio vectorial con producto interno. En esta sección abordaremos con más detalle estos conceptos geométricos.


\subsection{Norma de un vector}

\begin{defi}
  Sea $V$ un espacio con producto interno, definimos la \emph{norma} de un vector $v \in V$ como $\norm{v} = \sqrt{\inner{v,v}}$.
\end{defi}

La norma de un vector puede pensarse como su \emph{magnitud}. Además, la norma nos permite definir una noción de distancia entre dos vectores, si pensamos en un vector como un ``camino'' entonces para ir de un vector $v$ al vector $w$ tenemos que ``caminar'' un vector $w-v$ ya que $w = v + (w-v)$, así podemos pensar en $\norm{w-v}$ como la distancia del vector $v$ al vector $w$.

El siguiente teorema demostrará algunas propiedades de la norma que intuitivamente nos permiten considerarla como la magnitud de un vector.

\begin{teor} \label{teor:propNorm}
  Sea $V$ un espacio con producto interno, entonces para cualesquiera $v,w \in V$ y $\lambda \in \K$ se satisface las siguiente propiedades:
  \begin{enumerate}
    \item $\norm{v}>0$ para $v \neq 0$.
    \item $\norm{\lambda v} = \abs{\lambda}\norm{v}$.
    \item $\norm{v \pm w}^2 = \norm{v}^2 \pm 2 \Re(\inner{v,w}) + \norm{w}^2$.
    \item Desigualdad de Cauchy-Schwartz: $\abs{\inner{v,w}} \leq \norm{v} \norm{w}$.
    \item Desigualdad del triángulo: $\norm{v+w} \leq \norm{v} + \norm{w}$.
  \end{enumerate}
\end{teor}
\begin{proof}~
  \begin{enumerate}
    \item Si $v \neq 0$ por definición $\inner{v,v}>0$ y por propiedades de las desigualdades tenemos que $\norm{v} = \sqrt{\inner{v,v}}>0$.
    
    \item Por la linealidad izquierda y linealidad conjugada derecha tenemos que 
      \[ \norm{\lambda v} = \sqrt{\inner{\lambda v, \lambda v}} = \sqrt{\lambda \bar\lambda \inner{v,v}} = \sqrt{\abs{\lambda}^2} \sqrt{\inner{v,v}} = \abs{\lambda}\norm{v}. \]
    
    \item Aplicando la linealidad izquierda y linealidad conjugada derecha, tenemos que
      \begin{align*}
        \norm{v \pm w}^2 &= \inner{v\pm w, v\pm w} \\
          &= \inner{v,v\pm w} \pm \inner{w, v\pm w} \\
          &= (\inner{v,v} \pm \inner{v,w}) \pm (\inner{w,v} \pm \inner{w,w}) \\
          &= \norm{v}^2 \pm(\inner{v,w} + \inner{w,v}) + \norm{w}^2.
      \end{align*}

      Ahora, por hermiticidad y propiedades de los complejos tenemos que 
        \[
          \inner{v,w} + \inner{w,v} = \inner{v,w} + \overline{\inner{v,w}} = 2\Re(\inner{v,w}).
        \]
      De esta forma podemos concluir que $\norm{v\pm w}^2 = \norm{v}^2 \pm 2\Re(\inner{v,w}) + \norm{w}^2$.

    \item Notemos que si $v = 0$ entonces claramente se sostiene que $\inner{v,w} = 0 = \norm{v} \norm{w}$, de esta forma consideremos el caso donde $v \neq 0$. Ahora, dado que $v \neq 0$ entonces $\norm{v}^2 = \inner{v,v} > 0$ por definición, así consideremos el vector $u$ dado por
      \[
        u = w - \frac{\inner{w,v}}{\norm{v}^2} v.
      \]
    En primer lugar, por linealidad izquierda y definición, se cumple que 
      \begin{align*}
        \inner{u, v} &=  \Inner{w - \frac{\inner{w,v}}{\norm{v}^2} v, v} 
           = \inner{w,v} - \frac{\inner{w,v}}{\norm{v}^2}  \inner{v,v} \\
          &= \inner{w,v} - \frac{\inner{w,v}}{\norm{v}^2}  \norm{v}^2 
           = \inner{w,v} - \inner{w,v} \\
          &= 0.
      \end{align*}
    Así, para todo $\lambda \in \K$ tenemos que $\inner{u,\lambda v} = \bar \lambda \inner{u,v} = 0$. De este modo, dado que $\inner{w,v} \inner{v,w} =\overline{\inner{v,w}}  \inner{v,w} = \abs{\inner{v,w}}^2$, por hermiticidad y propiedades de los complejos, entonces veamos que 
      \begin{align*}
        \norm{u}^2 &= \inner{u, u}
           = \Inner{u, w - \frac{\inner{w,v}}{\norm{v}^2} v} \\
          &= \inner{u,w} + \Inner{u, - \frac{\inner{w,v}}{\norm{v}^2} v} 
           = \inner{u,w} \\
          &= \Inner{w - \frac{\inner{w,v}}{\norm{v}^2} v, w} 
           = \inner{w,w} - \frac{\inner{w,v}}{\norm{v}^2} \inner{v,w} \\
          &= \norm{w}^2 - \frac{\abs{\inner{v,w}}^2}{\norm{v}^2}.
      \end{align*}
    
    Dado que $\norm{u}^2 \geq 0$ por la propiedad 1, entonces podemos concluir que
      \[0 \leq \norm{w}^2 - \frac{\abs{\inner{v,w}}^2}{\norm{v}^2}
          \implies
        \abs{\inner{v,w}}^2 \leq \norm{v}^2 \norm{w}^2.
      \]
    Lo que por propiedades de las desigualdades, finalmente implica que $\abs{\inner{v,w}} \leq \norm{v} \norm{w}$.

    \item Por propiedades de los complejos tenemos que $\Re(\inner{v,w}) \leq \abs{\inner{v,w}}$, pero por la desigualdad de Cauchy-Schwartz tenemos que $\abs{\inner{v,w}} \leq \norm{v} \norm{w}$. De esta forma, por la propiedad 2 tenemos que 
      \begin{align*}
        \norm{v+w}^2 &= \norm{v}^2 + 2 \Re(\inner{v,w}) + \norm{w}^2 \\
          &\leq \norm{v}^2 + 2\norm{v} \norm{w} + \norm{w}^2 \\
          &= (\norm{v} + \norm{w})^2.
      \end{align*}
    Y por propiedades de las desigualdades, podemos concluir que $\norm{v+w} \leq \norm{v}+\norm{w}$. \qedhere
  \end{enumerate}
\end{proof}

\begin{coro} \label{coro:SchwartzEq}
  La igualdad en desigualdad de Cauchy-Schwartz se da si solo si los vectores son linealmente dependientes.
\end{coro}

\begin{proof}
  Supongamos que $v$ y $w$ son linealmente dependientes. Si $v = 0$, por la prueba de la propiedad 4 del teorema \ref{teor:propNorm} ya vimos que se cumple la igualdad, así, supongamos que $v \neq 0$.
  
  Dado que $v \neq 0$, entonces por independencia lineal debe existir $\lambda \in \K$ tal que $w = \lambda v$, de este modo, por el teorema \ref{teor:propNorm}, tenemos que
    \[
      \abs{\inner{v,w}} = \abs{\inner{v,\lambda v}} = \abs{\lambda\inner{v,v}} = \abs{\lambda} \abs{\inner{v,v}} = \abs{\lambda} \norm{v}^2 = \norm{v} (\abs{\lambda} \norm{v}) = \norm{v} \norm{\lambda v} = \norm{v} \norm{w}.
    \]
  
  Ahora, supongamos que se da la igualdad. Es trivial que $v$ y $w$ son linealmente independientes cuando $v = 0$, pero si $v \neq 0$, entonces por la prueba de la propiedad 4 del teorema \ref{teor:propNorm} tenemos que el vector $u = w - (\inner{w,v}/\norm{v}^2)v$ cumple que $\norm{u} = 0$, pero esto implica que $u = 0$ y por tanto
    \[ w = \frac{\inner{w,v}}{\norm{v}^2} v, \]
  lo que muestra que $u$ y $v$ son linealmente independientes.
\end{proof}

Ahora, hemos visto que el producto interno define el conceptos de distancia, pero este proceso funciona a la inversa, dado ciertas distancias es posible conocer el producto interno de dos vectores, esta propiedad es conocida como identidad de polarización. 

\begin{teor}[Identidad de polarización]
  Sea $V$ un espacio con producto interno, entonces para cualesquiera $v,w \in V$ se cumple que:
    \begin{enumerate}
      \item Si $\K = \R$ entonces
        \[ \inner{v,w} = \frac{1}{4} \norm{v+w}^2 - \frac{1}{4} \norm{v-w}^2.\]
      \item Si $\K = \C$ entonces
      \[ \inner{v,w} = \frac{1}{4} \norm{v+w}^2 - \frac{1}{4} \norm{v-w}^2 + \frac{i}{4}\norm{v+iw}^2 - \frac{i}{4}\norm{v-iw}^2.\]
    \end{enumerate}
\end{teor}
\begin{proof}
  Por el teorema \ref{teor:propNorm} tenemos que $\norm{v \pm w}^2 = \norm{v}^2 \pm 2 \Re(\inner{v,w}) + \norm{w}^2$, de este modo
    \begin{align*}
      \norm{v+w}^2 - \norm{v-w}^2
        &= (\norm{v}^2 + 2 \Re(\inner{v,w}) + \norm{w}^2) - (\norm{v}^2 - 2 \Re(\inner{v,w}) + \norm{w}^2) \\
        &= 4\Re(\inner{v,w}). \tagthis\label{eq:IdPolRe}
    \end{align*}
  Notemos que si $\K = \R$, entonces $\Re(\inner{v,w}) = \inner{v,w}$, de este modo tenemos que
    \[
      \inner{v,w} = \frac{1}{4} \norm{v+w}^2 - \frac{1}{4} \norm{v-w}^2.
    \]
  
  Ahora, si $\K = \C$, por el teorema \ref{teor:propNorm} y notando que $\Re(-i\inner{v,w}) = \Im(\inner{v,w})$, por propiedades de complejos, entonces tenemos que 
    \begin{align*}
      i\norm{v \pm iw}^2
        &= i\norm{v}^2 \pm 2 i\Re(\inner{v,iw}) + i\norm{w}^2 \\
        &= i\norm{v}^2 \pm 2 i\Re(-i\inner{v,w}) + i\norm{w}^2 \\
        &= i\norm{v}^2 \pm 2 i\Im(\inner{v,w}) + i\norm{w}^2, \\
      i\norm{v + iw}^2 - i\norm{v - iw}^2
        &= (i\norm{v}^2 + 2 i\Im(\inner{v,w}) + i\norm{w}^2) - (i\norm{v}^2 - 2 i\Im(\inner{v,w}) + i\norm{w}^2 ) \\
        &= 4i\Im(\inner{v,w}).  \tagthis \label{eq:IdPolIm}
    \end{align*}
  

  De esta forma, dado que $\inner{v,w} = \Re(\inner{v,w}) + i\Im(\inner{v,w})$, entonces despejando de las ecuaciones \ref{eq:IdPolRe} y \ref{eq:IdPolIm} obtenemos que
    \[
      \inner{v,w} = \frac{1}{4} \norm{v+w}^2 - \frac{1}{4} \norm{v-w}^2 + \frac{i}{4}\norm{v+iw}^2 - \frac{i}{4}\norm{v-iw}^2. \qedhere
    \]
\end{proof}



\subsection{Ángulo entre vectores}

Además de la desigualdad del triangulo, una de las consecuencias de la desigualdad de Cauchy-Schwartz es que para cualesquiera dos vectores $v,w \in V$ no nulos de un espacio euclidiano se cumple que
\[
  \frac{\abs{\inner{v,w}}}{\norm{v}\norm{w}} \leq 1 \implies  -1 \leq \frac{\inner{v,w}}{\norm{v}\norm{w}} \leq 1.
\]
Por las propiedades del coseno, esto implica que existe un único $\theta \in [0,\pi]$ tal que 
\[
  \cos\theta = \frac{\inner{v,w}}{\norm{v}\norm{w}}.
\]
Este número $\theta$ es el que se le suele asociar como el ángulo entre los vectores $v$ y $w$.

\begin{defi}
  Sea $V$ un espacio euclidiano y $v,w \in V$ son vectores no nulos, entonces definimos el ángulo entre $v$ y $w$ como el único número $\angle(v,w) \in [0,\pi]$ tal que
    \[ \cos\angle(v,w) = \frac{\inner{v,w}}{\norm{v}\norm{w}}. \]
\end{defi}

Notemos que por definición el ángulo no depende de la posición de $v$ y $w$ por lo que $\angle(v,w) = \angle(w,v)$. Además, como se podría suponer, el angulo entre dos vectores cumple muchas de las propiedades geométricas conocidas.

\begin{teor}
  Sea $V$ un espacio euclidiano y $v,w \in V$ son vectores no nulos, se cumplen las siguientes propiedades:
  \begin{enumerate}
    \item Los vectores $v$ y $w$ son linealmente independientes si y solo si el ángulo entre $v$ y $w$ es $0$ o $\pi$.
    \item Sea $\lambda \in \R^*$ entonces, los ángulos $\angle(v,w)$ y $\angle(\lambda v,w)$ son iguales si $\lambda > 0$ y son suplementarios si $\lambda < 0$.
    \item (Teorema de cosenos) Se cumple que
      \[ \norm{v \pm w}^2 = \norm{v}^2 + \norm{w}^2 \pm 2\norm{v}\norm{w}\cos\angle(v,w). \]
  \end{enumerate}
\end{teor}

\begin{proof}~
  \begin{enumerate}
    \item Por el corolario \ref{coro:SchwartzEq} tenemos que $v$ y $w$ son linealmente independientes si y solo si se cumple que $\abs{\inner{v,w}} = \norm{v}\norm{w}$, pero seo se cumple si y solo si $\cos\angle(v,w) = \pm 1$ y por propiedades del coseno se cumple si y solo si $\angle(v,w) = 0$  o $\angle(v,w) = \pi$.
    
    \item Notemos que por propiedades conocidas del producto interno y la norma se cumple que 
      \[
        \cos\angle(\lambda,w) = \frac{\inner{\lambda v,w}}{\norm{\lambda v} \norm{w}}
          = \frac{\lambda}{\abs{\lambda}}  \frac{\inner{v,w}}{\norm{v} \norm{w}}
          = \frac{\lambda}{\abs{\lambda}} \cos\angle(v,w).
      \]
      Si $\lambda > 1$ entonces $\lambda/\abs{\lambda} = 1$, por lo que $\angle(\lambda v ,w) = \angle(v,w)$, ya que sus cosenos son iguales. Ahora, si $\lambda<0$, entonces $\lambda/\abs{\lambda} = -1$, por lo que $\angle(\lambda v,w) = \pi - \angle(v,w)$ ya que $\cos\angle(\lambda v,w) = -\cos\angle(v,w)$.

      \item Por el teorema \ref{teor:propNorm} tenemos que $\norm{v - w}^2 = \norm{v}^2 \pm 2 \inner{v,w} + \norm{w}^2$ y por definición $\inner{v,w} = \norm{v}\norm{w} \cos\angle(v,w)$ de este modo tenemos que 
        \[ \norm{v \pm w}^2 = \norm{v}^2  + \norm{w}^2 \pm 2 \norm{v}\norm{w} \cos\angle(v,w). \qedhere \]
  \end{enumerate}
\end{proof}