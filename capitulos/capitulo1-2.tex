\section{Coordenadas y cambio de base}

Como ya mencionamos en la sección anterior, dada una base existe una única combinación lineal de elementos de la base para cada vector del espacio, esto nos permite generalizar el concepto de coordenadas. En esta sección abordaremos a más detalle este tema.


\begin{defi}
  Sea $V$ un $\F$-espacio vectorial de dimensión $n$, decimos que $B = (v_1, v_2, \ldots, v_n)$ es una \emph{base ordenada} si $\{v_1, v_2, \ldots, v_n\}$ es una base de $V$. Además para todo $v \in V$ donde $v = \lambda_1 v_1 + \cdots \lambda_n v_n$, definiremos su \emph{coordenada con respecto a $B$} como
    \[ [v]_B = \begin{pmatrix}
      v_1 \\ \vdots \\ v_n
    \end{pmatrix}.\]
\end{defi}

Notemos que bajo esta definición $(v_1, v_1, \ldots, v_n)$ y $(v_2, v_1, \ldots, v_n)$ son dos bases ordenadas distintas, al igual que cualquier otra permutación, aunque provengan de la misma base. De este modo nos aseguraremos que la coordenada de cualquier elementos sea única. Una propiedad interesante es que esta asociación de vectores de $V$ con $n$-tuplas de $\F^n$ forma una transformación lineal biyectiva.

\begin{prop}
  Sea $B$ una base ordenada del $\F$-espacio vectorial $V$, entonces la función $[\cdot]_B\colon V \to \F^n$ es un isomorfismo.
\end{prop}
\begin{proof}
  Sea $v,w \in V$ con $v = \lambda_1 v_1 + \cdots + \lambda_n v_n$, $w = \mu_1 v_1 + \cdots + \mu_n v_n$ y $\alpha \in K$, entonces veamos que
  \[
  \alpha v+w =  (\alpha\lambda_1+\mu_1) v_1 + \cdots + (\alpha\lambda_n+\mu_n) v_n,
  \]
  de esta forma es claro que
  \begin{align*}
    [v+w]_B 
      &= \begin{pmatrix} \alpha\lambda_1+\mu_n \\ \vdots \\ \alpha\lambda_n+\mu_n \end{pmatrix} \\
      &= \alpha\begin{pmatrix} \lambda_1 \\ \vdots \\ \lambda_n \end{pmatrix}
       + \begin{pmatrix} \mu_n \\ \vdots \\ \mu_n  \end{pmatrix} \\
      &= \alpha[v]_B + [w]_B.
  \end{align*}
  Mostrando así, que es una transformación lineal. Para la inyectividad, es claro que si $[v]_B = (0,\cdots,0)$ entonces $v = 0v_1 + \cdots + 0v_n = 0$, por lo que $\ker([\cdot]_B) = \{0\}$. Para la suprayectividad, claramente tenemos que para todo $(\lambda_1, \ldots, \lambda_n)^T \in \F^n$ se cumple que $\lambda_1 v_1 + \cdots \lambda_n v_n \in V$ y  $[\lambda_1 v_1 + \cdots \lambda_n v_n]_B = (\lambda_1, \ldots, \lambda_n)^T$.
\end{proof}


\subsection{Cambio de base}

Dadas dos bases ordenadas $B = (v_1,\cdots,v_n)$ y $B' = (v_1',\cdots,v_n')$ de $B$ una pregunta común sale a flote, dada la coordenadas de $v \in V$ con respecto a $B$ como obtenemos su coordenada con respecto a $B$. Resulta que para dos bases $B$ y $B'$ existe exactamente una matriz invertible $M$ tal que $M[v]_B = [v]_{B'}$ para todo $v \in V$, a esta matriz la denotaremos como $M_{B,B'}$.

La idea es la siguiente, si $e_i$ es la $i$-ésima columna de la matriz identidad, entonces por definición $[v_i]_B = e_i$, de este modo si $M_{B,B'} [v]_B = [v]_{B'}$ para todo $v \in V$, entonces
  \[ M_{B,B'} [v_i]_B = M_{B,B'} e_i = (M_{B,B'})_{*i} = [v_i]_{B'}. \]
Esto nos dice que $[v_i]_{B'}$ es la $i$-ésima columna de la matriz $M_{B,B'}$, en otras palabras que
  \[ M_{B,B'} = \begin{spmatrix}{c|c|c}  [v_1]_{B'} & \cdots & [v_n]_{B'}  \end{spmatrix}. \]

\begin{teor}
  Sea $V$ un $F$-espacio vectorial con bases ordenadas $B = (v_1,\cdots,v_n)$ y $B' = (v_1',\cdots,v_n')$, entonces existe una única matriz invertible $M_{B,B'}$ tal que para todo $v \in V$ se cumple que
    \[ M_{B,B'}[v]_B = [v]_{B'} \Eqand M_{B,B'}^{-1}[v]_{B'} = [v]_B. \]
  Además $(M_{B,B'})_{*i} = [v_i]_{B'}$ para todo $i \in \{1,\ldots,n\}$.
\end{teor}
\begin{proof}
  Primero, notemos que por propiedades conocidas y la definición de $M_{B,B'}$ es claro que 
    \[ M_{B,B'}[v_i]_B = M_{B,B'} e_i =  (M_{B,B'})_{*i} = [v_i]_{B'}. \]

  De este modo, sea $v \in V$ con $[v]_B = (\lambda_1,\ldots,\lambda_n)^T$, por la linealidad de $[\cdot]_{B'}$ y las propiedad ya mencionadas, tenemos que
    \begin{align*}
      v        &= \lambda_1 v_1 + \cdots + \lambda_n v_n, \\
      [v]_{B'} &= \lambda_1 [v_1]_{B'} + \cdots + \lambda_n [v_n]_{B'} \\
               &= \lambda_1 M_{B,B'}[v_1]_B + \cdots + \lambda_n M_{B,B'}[v_n]_B \\
               &= M_{B,B'} [\lambda_1 v_1 + \cdots + \lambda_n v_n ]_B \\
               &= M_{B,B'} [ v ]_B.
    \end{align*}
  De este modo $M_{B,B'}[v]_B = [v]_{B'}$ para toda $v \in B$. Es fácil ver que si $M_{B,B'}$ es invertible, entonces $M_{B,B'}^{-1}[v]_{B'} = [v]_B$, así solo nos queda demostrar que $M_{B,B'}$ es invertible.
  
  Consideremos el sistema de ecuaciones $M_{B,B'}x = \bec 0$, como $[\cdot]_B$ es suprayectivo, entonces existe $v \in V$ tal que $[v]_B = x$, de este modo $M_{B,B'}x = \bec 0$ implica que $M_{B,B'}[v]_B = [v]_{B'} = \bec 0$. Ahora, recordemos por inyectividad de $[\cdot]_{B'}$ que $[v]_{B'} = \bec 0$ si y solo si $v = 0$, de este modo $x = [v]_{B} = \bec 0$, pero esto implica que $M_{B,B'}x = \bec 0$ no tiene soluciones no triviales, así, por propiedades de las matrices, tenemos que $M_{B,B'}$ es invertible.

  La unicidad se da por el análisis hecho al principio. Si existiera otra matriz $P$ con las mismas cualidades, entonces para todo $i \in \{1,\ldots, n\}$ cumple que
  \[ (M_{B,B'})_{*i} = [v_i]_{B'} = P[v_i]_{B} = Pe_i = P_{*i},\]
  de este modo, dado que todas las columnas son iguales, se cumple que $P = M_{B,B'}$.
\end{proof}

Este teorema nos afirma que si tenemos una base ordenada $B$ y entonces para cualquier otra base ordenada $B'$ existe una matriz invertible asociada que realiza el cambio de coordenadas. La vuelta de esta proposición también es cierta, de este modo existe una biyección entre el conjunto de bases ordenadas y el conjunto de matrices invertibles.

\begin{prop}
  Sea $P$ una matriz de $n\times n$ con entradas en $\F$, $V$ un $\F$-espacio vectorial y $B = (v_1, \ldots, v_n)$ una base ordenada, entonces existe una única base ordenada $B'$ tal que $M_{B',B} = P$.
\end{prop}
\begin{proof}
  Si $P = (p_{ij})$ entonces definamos $B' = (v_1', \ldots, v_n')$ donde
    \[ v_j' = p_{1j}v_1 + \cdots +  p_{nj}v_n,     \qquad j \in \{1,\ldots, n\}. \]
  Notemos que $[v_i]_B = P_{*i}$ para toda $i \in 1,\cdots,n$, de este modo si $B'$ fuese una base ordenada, entonces $P = M_{B',B}$, por el teorema anterior. Así probemos que $B'$ es una base ordenada.

  Sea $T(v) = Pv$ y $S(v) = [v]_B$ por propiedades conocidas y recordando que $P$ es invertible, tenemos que $T$ y $S$ son transformaciones lineales biyectivas, así notemos, para toda $i \in \{1,\cdots,n\}$, que
    \[ v_i' = S^{-1}(P_{*i}) = S^{-1} \bigl( Pe_i \bigr) = (S^{-1} \circ T)(e_i) = (S^{-1} \circ T)([v_i]_B ) = (S^{-1} \circ T \circ S) (v_i).\]
  Como $T$ y $S$ son transformaciones lineales biyectivas, entonces $S^{-1} \circ T \circ S$ es una transformación lineal biyectiva, así, por propiedades conocidas y recordando que $\{ v_1,\ldots,v_n \}$ es una base, entonces $\{ (S^{-1} \circ T \circ S)(v_i), \ldots, (S^{-1} \circ T \circ S)(v_n) \} = \{v_1', \ldots, v_n'\}$ también será una base. De este modo, por definición $B'$ es una base ordenada.

  Ahora, supongamos que existe otra base ordenada $B'' = (v_1'', \ldots, v_n'')$ tal que $P = M_{B'',B}$ entonces, para toda $i \in \{1,\ldots,n\}$ se cumple que
  \[ [v_i'']_B =  P[v_i'']_{B''} = Pe_i = P[v_i']_{B'} = [v_i']_B, \]
  así, por la biyectividad de $[\cdot]_B$, tenemos que $v_i'' = v_i'$ para toda $i \in \{1,\ldots,n\}$, mostrando así que $B'' = B'$.
\end{proof}
