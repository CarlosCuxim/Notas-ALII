\documentclass[11pt]{report}

% =========================================================
% PAQUETES
% =========================================================

% Idioma --------------------------------------------------
\usepackage[spanish, mexico]{babel}
\usepackage[utf8]{inputenc}

% Página --------------------------------------------------
\usepackage{geometry}

% Fuente --------------------------------------------------
\usepackage{lmodern}
\usepackage[T1]{fontenc}

% Macros --------------------------------------------------
\usepackage{qx-Macros}
\usepackage{qx-Delimiters}

% Otros ---------------------------------------------------
\title{Notas Algebra Lineal II}
\author{Pendiente}
\date{\today}

\begin{document}
\maketitle

\chapter{Matrices y transformaciones}

\section{Notación e ideas}

Antes de empezar con las matrices y trasnformaciones lineales, hay que recordar algunas definiciones

\begin{defi}
  Sea $G$ un conjunto no vacío, una \emph{opración binaria} de $G$ es cualquier función $\cdot \colon G \times G \to G$. Por convención tenemos que $\cdot(x,y) = x \cdot y$. Además, decimos que el par $(G, \cdot)$ es un \emph{grupo} si cumplen los siguientes axiomas
  \begin{enumerate}
    \item (Asociatividad) Para todos $x, y, z \in G$ se cumple que $x\cdot(y\cdot z) = (x\cdot y)\cdot z$.
    \item (Existencia del elemento neutro) Existe un elemento $e$ tal que para todo $x \in G$ se tiene que $x\cdot e = e\cdot x = x$.
    \item (Exsitencia del elemento inverso) Para todo $x \in G$ existe $x^{-1} \in G$ tal que $x\cdot x^{-1} = x^{-1}\cdot x = e$.
  \end{enumerate}

  Por ultimo, decimos que un grupo es \emph{abeliano} si para todos $x, y \in G$ se cumple que $x\cdot y = y\cdot x$ (conmutatividad).
\end{defi}

\begin{defi}
  Sea $K$ un conjunto no vacío con dos operaciones binarias $+$ (suma) y $\cdot$ (producto), decimos que la terna $(K, +, \cdot)$ es un \emph{campo} si cumple los siguientes axiomas
  \begin{enumerate}
    \item $(K,+)$ es un grupo abeliano. Al elemento neutro lo denotaremos como $0$ y al inverso de $x \in K$ lo denotaremos como $-x$.
    \item El la operación $\cdot$ es asociativa y conmutatividad sobre $K$.
    \item (Existencia de neutro multiplicativo) Existe un elemento $1 \in K-\{0\}$ tal que para todo $x \in K$ se cumple que $1\cdot x = x\cdot 1 = x$.
    \item (Existencia de inverso multiplicativo) Para todo $x \in K-\{0\}$ existe un elemento $ x^{-1} \in K$ tal que $x \cdot x^{-1} = x^{-1} \cdot x = 1$.
    \item (Distributividad) Para todos $x, y, z \in K$ se cumple que $x \cdot (y + z) = x \cdot y + x \cdot z$.
  \end{enumerate}
\end{defi}


\end{document}