% !TeX root = ./main.tex
\documentclass[11pt]{report}

% =========================================================
% PAQUETES
% =========================================================

% Idioma --------------------------------------------------
\usepackage[spanish, mexico]{babel}

% Página --------------------------------------------------
\usepackage{geometry}

% Macros --------------------------------------------------
\usepackage{qx-Macros}
\usepackage{qx-Delimiters}

% Fuente --------------------------------------------------
\usepackage{newtxtext, newtxmath}
\usepackage[T1]{fontenc}

% Hipervínculos ------------------------------------------
\usepackage{hyperref}

% Otros ---------------------------------------------------
\title{\textbf{Notas Álgebra Lineal II}}
\author{Jesús Efrén Pérez Terrazas y Carlos Enrique Cuxim Tuz}
\date{\today}

\begin{document}
\maketitle


\tableofcontents


% =============================================================================
% CAPÍTULO 1
% =============================================================================

\chapter{Matrices y transformaciones lineales}

\chapter{Matrices y transformaciones lineales}

\section{Notación e ideas}

Antes de empezar a estudiar las propiedades de las matrices y transformaciones lineales hay que recordar algunas definiciones importantes, así como las estructuras con las que vamos a estar trabajando.

En primer lugar, recordemos que una \emph{estructura algebraica} no es más que la combinación de un conjunto, funciones o relaciones asociadas a éste y propiedades relacionadas a estas últimas.

Las funciones pueden ser de varios tipos, pero la más común son las llamadas \emph{operaciones binarias}, que no son más que que funciones que toman como entrada dos elementos de un conjunto y retornan un elemento del mismo conjunto. Por ejemplo, si consideramos el conjunto $G$, entonces una operación binaria es cualquier función $\cdot \colon  G \times G \to G$. Por convención, para las operaciones binarias escribimos $x \cdot y$  para referirnos a $\cdot(a,b)$. Con esto en mente podemos definir las estructuras importante para el curso.


\subsection{Grupos}

\begin{defi}
  Si $G$ es un conjunto no vacío y $+$ una operación binaria de éste, decimos que el par $(G, \cdot)$ es un \emph{grupo} si cumple los siguientes axiomas:
  \begin{enumerate}
    \item (Asociatividad) Para cualesquiera $x, y, z \in G$ se cumple que $x\cdot(y\cdot z) = (x\cdot y)\cdot z$.
    \item (Existencia del elemento neutro) Existe un elemento $e$ tal que para todo $x \in G$ cumple que $x\cdot e = e\cdot x = x$.
    \item (Existencia del elemento inverso) Para todo $x \in G$ existe $x^{-1} \in G$ tal que $x\cdot x^{-1} = x^{-1}\cdot x = e$.
  \end{enumerate}
  Por último, decimos que un grupo es \emph{abeliano} si para cualesquiera $x,y \in G$ se cumple que $x\cdot y = y \cdot x$ (conmutatividad).
\end{defi}

Por convención, cuando nos referiremos a un grupo, usualmente solo haremos mención al conjunto de éste, por ejemplo, cuando nos referiremos al grupo $(G, \cdot)$ lo haremos solo como ``el grupo $G$''. De igual forma, siempre que hablemos de una ``multiplicación'' podemos omitir el símbolo $\cdot$, es decir $x \cdot y = xy$.

\begin{teor}
  Sea $G$ un grupo, las siguientes propiedades se satisfacen.
    \begin{enumerate}
      \item El elemento neutro es único.
      \item Para cada $a \in G$ el inverso de $a$ es único.
      \item Sean $a,b,c \in G$, si $ab=ac$ o $ba=ca$, entonces $b=c$.
    \end{enumerate}
\end{teor}

Existen muchos ejemplos de grupos, por ejemplo los enteros, racionales, reales y complejos con la suma. Otro ejemplo son las matrices de tamaño $n \times n$ con entradas en los reales junto con la suma de matrices. Estos dos ejemplos fueron de  grupos abelianos, un ejemplo de un grupo no abeliano son los conjuntos de funciones biyectivas, como los grupos de permutación, con la composición usual de funciones o el grupo lineal, conformado de las matrices invertibles con la multiplicación de matrices.

\subsection{Campos}

\begin{defi}
  Sea $K$ un conjunto no vacío con dos operaciones binarias $+$ (suma) y $\cdot$ (producto), decimos que la terna $(K, +, \cdot)$ es un \emph{campo} si cumple los siguientes axiomas
  \begin{enumerate}
    \item $(K,+)$ es un grupo abeliano. Al elemento neutro lo denotaremos como $0_K$ y al inverso de $x \in K$ lo denotaremos como $-x$.
    \item La operación $\cdot$ es asociativa y conmutativa sobre $K$.
    \item (Existencia de neutro multiplicativo) Existe un elemento $1_K \in K-\{0\}$ tal que para todo $x \in K$ se cumple que $1_K\cdot x = x\cdot 1_K = x$.
    \item (Existencia de inverso multiplicativo) Para todo $x \in K-\{0\}$ existe un elemento $ x^{-1} \in K$ tal que $x \cdot x^{-1} = x^{-1} \cdot x = 1$.
    \item (Distributividad) Para cualesquiera $x, y, z \in K$ se cumple que $x \cdot (y + z) = x \cdot y + x \cdot z$.
  \end{enumerate}
\end{defi}

De igual forma que con los grupos, por convención cuando nos referiremos al campo, usualmente solo haremos mención al conjunto de éste, para la notación de este libro usaremos $\F$ para denotar un campo. Si no existe confusión, nos referiremos simplemente como 0 y 1 a los neutros aditivo y multiplicativo de un campo.

Existen varios ejemplos de campos, por ejemplo los racionales $\Q$, los reales $\R$, los complejos $\C$ y todos los campos asociados a la aritmética de $\Z$ modulo un primo $p$, los cuales denotamos como $\F_p$.

\subsection{Espacio vectoriales}

\begin{defi}
  Sea $V$ un conjunto no vacío con una operación binaria $+$ (suma), $(\F, +_\F, \cdot_\F )$ un campo y sea $\cdot\colon K \times V \to V$ una operación llamada \emph{producto escalar}. Decimos que $V$ es un $\F$-espacio vectorial si cumple los siguientes axiomas:
  \begin{enumerate}
    \item $(V, +)$ es un grupo abeliano.
    \item (No trivialidad) Para todo $v \in V$ se cumple que $1_\F \cdot v = v$.
    \item (Compatibilidad) Para todo $\lambda, \mu \in \F$ y $v \in V$ se cumple que $\lambda \cdot (\mu \cdot v) = (\lambda \cdot_\F \mu)  \cdot v$.
    \item (Primera ley distributiva) Para todo $\lambda \in \F$ y $v,w \in V$ se cumple que $\lambda \cdot (v + w) = \lambda \cdot v + \lambda \cdot w$.
    \item (Segunda ley distributiva) Para todo $\lambda, \mu \in \F$ y $v\in V$ se cumple que $(\lambda +_\F \mu)\cdot v = \lambda \cdot v + \mu \cdot v$.
  \end{enumerate}
\end{defi}


Por convención, a los elementos de $V$ les llamaremos vectores y a los del campo, escalares. De igual forma y si no existe confusión, a partir de ahora a la suma del campo $\F$ y de $V$ lo denotaremos con el mismo símbolo. Para el producto del campo y el producto escalar, por convención a partir de ahora se omitirá el símbolo $\cdot$, de esta forma $\lambda v$ hará referencia a la multiplicación del escalar $\lambda$ con el vector $v$.

Existen varios ejemplos de espacios vectoriales. Uno de ellos es el conjunto de matrices de $n \times m $ con entradas en un campo $\F$, o el anillo de polinomios $\F[x]$.

\subsection{Subespacios vectoriales}

\begin{defi}
  Sea $V$ un $\F$-espacio vectorial. Decimos que $W \subseteq V$ es un $\F$-subespacio vectorial de $V$, si al restringir la suma y producto escalar de $V$ a $W$ se cumple que $W$ es un $\F$-espacio vectorial.
\end{defi}

Un ejemplo común es el subespacio trivial $\{0\}$. Otro ejemplo es el subespacio generado por un subconjunto $S \subseteq W$, al cual denotamos por $\inner{S}$ (Ejercicio \ref{exer:SubEsGene}), que en el caso donde $S\neq \emptyset$, éste se compone de todas las combinaciones lineales de elementos de $S$ (Ejercicio \ref{exer:SubEsCombLin}), es decir
\[ \inner{S} = \set{ \sum_{i=1}^n \lambda_i v_i : \lambda_i \in \F, v_i \in S, n \in \N }. \]
Además, este conjunto cumple que es el subespacio vectorial más pequeño tal que contiene a $S$, es decir que si $W$ es un subespacio vectorial tal que $S \subseteq W$, entonces $\inner{S} \subseteq W$. En el caso en que $\inner{S} = V$, entonces decimos que $S$ es un \emph{conjunto generador}.

\begin{teor}
  Sean $V$ un $\F$-espacio vectorial, la intersección de cualquier colección de subespacios de $V$ es un subespacio de $V$.
\end{teor}

\subsection{Independencia lineal y bases}

\begin{defi}
  Sea $S \subseteq W$, decimos que $S$ es linealmente dependiente, si existe una combinación lineal de $S$ tal que
  \[ \sum_{i=1}^n \lambda_i v_i = 0, \]
  donde para al menos algún $i \in \{1,\ldots,n\}$ se cumple que $\lambda_i \neq 0_\F$. En caso que $S$ no sea linealmente dependiente, diremos que es linealmente independiente.

  Dado $B \subset V$ diremos que $B$ es una \emph{base} de $V$ si es un conjunto generador linealmente independiente.
\end{defi}

Las bases son conjuntos muy importantes en los espacios vectoriales porque nos permiten generalizar el concepto de coordenadas. En primer lugar, todo conjunto linealmente independiente cumple que la expresión como combinación lineal de todo $v \in \inner{S}$ es única, de esta forma si $B$ es una base de cardinalidad $n \in \N$ de $V$, entonces para todo $v \in V$ existe una única combinación lineal tal que 
\[ v = \lambda_1 v_1 + \cdots + \lambda_n v_n, \qquad \lambda_1,\ldots,\lambda_n \in \F, v_1,\ldots,v_n \in B. \]
Esto nos permite crear una biyección entre los espacios $V$ y $\F^n$ mediante $B$, a la cual denotaremos como
\[ [v]_B = (\lambda_1, \ldots, \lambda_n)^t.\]

\begin{teor}
  Sea $V$ un $\F$-espacio vectorial, entonces $V$ tiene al menos una base. Además, si $B_1$ y $B_2$ son bases de $B$, entonces $\abs{B_1} = \abs{B_2}$.
\end{teor}

Este teorema es uno de los resultados más importantes del Álgebra Lineal, éste nos permite definir la \emph{dimensión} de un $\F$-espacio vectorial $V$ como la cardinalidad de cualquiera de las bases, es decir, si $B$ es una base de $V$, entonces
\[ \dim_\F (V) = \abs{B}. \]


\subsection{Suma y suma directa}

\begin{defi}
  Sean $W_1$ y $W_2$ dos $\F$-subespacios vectoriales de $V$, definimos la suma de $W_1$ con $W_2$ como
  \[ W_1 + W_2 = \{ w_1 + w_2 : w_1 \in W_1, w_2 \in W_2 \}.\]
  De manera general, si $W_1, \ldots, W_k$ son $\F$-subespacios vectoriales de $V$, entonces la suma de $W_1, \ldots, W_k$  la definimos como
  \[ W_1 + \cdots + W_k = \{ w_1 + \cdots + w_k : w_1 \in W_1, \ldots ,w_k \in W_k \}.\]
\end{defi}

Recordemos que la suma de subespacios vectoriales genera un nuevo subespacio vectorial que además es el subespacio generado por la unión de estos, es decir
  \[ W_1 + \cdots + W_k = \inner{ W_1 \cup \cdots \cup W_k }. \]
Hay una expresión bien conocida para la dimensión de la suma de dos subespacio vectoriales, esta fórmula es conocida como la fórmula de Grassmann.

\begin{teor}[Fórmula de Grassmann]
  Si $W_1$ y $W_2$ son subespacios de un $\F$-espacio vectorial $V$ de dimensión finita, entonces
    \[ \dim_\F (W_1 + W_2) = \dim_\F (W_1) + \dim_\F (W_2) - \dim_\F (W_1 \cap W_2).\]
\end{teor}


Así como en los espacios vectoriales existe la independencia de vectores, podemos definir una independencia de subespacios.

\begin{defi}
  Si $W_1, \ldots, W_k$ son $\F$-subespacios vectoriales de $V$ y $W = W_1 + \cdots + W_k$, decimos que  $W_1, \ldots, W_k$ son independientes si $w_1 + \cdots + w_k = 0$ con $w_i \in W_i$ implica que $w_i = 0$, donde $i \in \{1,\ldots,k\}$.
\end{defi}

Así como la expresión de un vector como combinación lineal de un conjunto linealmente independiente es única, la independencia de subespacios nos permite afirmar que para todo $v \in W_1 + \cdots + W_k$ la expresión $v = w_1 + \cdots + w_k$, con $w_i \in W_i$ y $i \in \{1, \ldots, k\}$, es única.
\begin{teor}
  Sean $W_1, \ldots, W_k$ subespacios de un $\F$-espacio vectorial $V$ de dimensión finita y $W = W_1 + \cdots + W_k$, entonces las siguientes propiedades son equivalentes:
  \begin{enumerate}
    \item $W_1, \ldots, W_k$ son independientes.
    \item Para todo $i \in \{2, \ldots, k\}$ se tiene que
            \[ W_i \cap (W_1 + \cdots + W_{i-1})  = \{0\}. \]
    \item Si $B_i$ es una base de $W_i$ con $i \in \{1, \ldots, k\}$, entonces $B = B_1 \cup  \cdots \cup  B_k$ es una base de $W$.
  \end{enumerate}
\end{teor}

\begin{defi}
  Si $W_1, \ldots, W_k$ son $\F$-subespacios vectoriales independientes se dice que la suma $W = W_1 + \cdots + W_k$ es \emph{directa} o que $W$ es la \emph{suma directa} de  $W_1, \ldots, W_k$ y se denotará como:
    \[ W =  W_1 \oplus \cdots \oplus W_k.\]
\end{defi}

Notemos que si escribimos $W_1 \oplus \cdots \oplus W_k$, entonces por definición entendemos que los subespacios $W_1, \ldots, W_k$ son independientes. De igual forma, por la fórmula de Grassmann, se puede comprobar que en este caso
\[ \dim_\F (W_1 \oplus \cdots \oplus W_k) = \dim_\F (W_1) + \cdots + \dim_\F (W_k).\]

\subsection{Transformaciones lineales}

\begin{defi}
  Sean $V$ y $W$ dos $\F$-espacios vectoriales, se dice que la función $T\colon V\to W$ es una transformación $\F$-lineal si para todo $v,w \in V$ y $\lambda \in \F$ se cumple que 
  \[ T(\lambda v + w) = \lambda T(v) + T(w).\]

  Si $T$ es biyectiva, entonces es un isomorfismo de $\F$-espacios vectoriales. En este caso decimos que $V$ y $W$ son isomorfos y lo denotamos como $V \cong W$.
\end{defi}

Al conjunto de transformaciones lineales de $V$ a $W$ se le denota como $L(V,W)$. En el caso que $W = V$ entonces $T$ se dice que es un \emph{endomorfismo} de $V$ y al conjunto de endomorfismos lo denotamos simplemente como $L(V)$.

\begin{teor}
  Sea $T\colon V \to W$ una transformación $\F$-lineal y $S$ un subconjunto de $V$.
  \begin{enumerate}
    \item Si $T$ es inyectiva y $S$ es linealmente independiente, entonces $T(S)$ es linealmente independiente.
    \item Si $T$ es suprayectiva y $S$ es un conjunto generador, entonces $T(S)$ es un conjunto generador.
    \item Si $T$ es biyectiva y $S$ es una base, entonces $T(S)$ es una base.
    \item Si $T$ es inyectiva, entonces $\dim_\F (V) \leq \dim_\F(W)$.
    \item Si $T$ es suprayectiva, entonces $\dim_\F (V) \geq \dim_\F(W)$.
    \item Si $T$ es biyectiva, entonces $\dim_\F (V) = \dim_\F(W)$.
  \end{enumerate}
\end{teor}

Las transformaciones lineales nos permiten enlazar espacios vectoriales, así como sus propiedades. Un ejemplo son los isomorfismos, estos nos permiten asegurar que dos espacios vectoriales se comportan de la misma forma, por lo que en esencia son el mismo espacio vectorial.

\begin{defi}
  Sea $T\colon V \to W$ una transformación $\F$-lineal, entonces definimos el \emph{núcleo} o \emph{kernel}  de $T$ como
    \[ \ker(T) = \{ v \in V : T(v) = 0_W \}.\]
  De manera análoga, definimos la imagen de $T$ como el conjunto
    \[ \Im(T) = \{w \in W : w = T(v) \text{ para alguna } v \in V \}. \]
\end{defi}

Estos dos conjuntos tienen algunas propiedades importantes. La primera es que el núcleo y la imagen son subespacios de $V$ y $W$, respectivamente. Otra propiedad, es que $T$ será inyectiva si y solo si $\ker(T) = \{0_V\}$. Y por último, tenemos uno de los resultados más importantes de las transformaciones lineales.
\begin{teor}
  Sea $T \colon V \to W$ una transformación $\F$-lineal. Si $V$ es de dimensión finita, entonces
    \[  \dim_\F( \ker T )  + \dim_\F( \Im T) = \dim_\F (V). \]
\end{teor}




\ExerciseSection


\begin{exerciselist}
  \item Si $V$ es un $\F$-espacio vectorial, demuestre que $0_\F \cdot v = 0$ y $(-1_\F) \cdot v = -v$ para todo vector $v \in V$.

  \item El conjunto $\{0\}$ tiene una estructura canónica de $\F$-espacio vectorial para cada campo $\F$: indique cuál es y verifique los axiomas. A tal espacio vectorial se le denomina \emph{trivial}.
  
  \item Sea $\F$ un campo dado. Indique cuál es la estructura canónica de $\F$ como $\F$-espacio vectorial y verifique los axiomas.
  
  \item Indique, en el campo de los complejos, sus estructuras canónicas como $\C$-espacio vectorial, como $\R$-espacio vectorial y como $\Q$-espacio vectorial. Verifique los axiomas.
  
  \item Sea $A$ un anillo con unidad $1_A \neq 0_A$ , y sean $\F$ un campo y $\phi\colon \F \to A$ un homomorfismo de anillos unitarios (es decir que $\phi(1_\F) = 1_A$ ). Demuestre que via $\phi$ se cumple que $A$ es un $\F$-espacio vectorial.
  
  \item Sea $\F[x]$ el anillo de polinomios con coeficientes en $\F$. Demuestre que $\F[x]$ es un $\F$-espacio vectorial.
  
  \item \label{exer:F^I} Sean $I$ un conjunto no vacío y $\F$ un campo dado. Denotemos como $\F^I$ al conjunto de las funciones $f\colon I \to \F$. Definimos la suma en $\F^I$ de la siguiente manera: dados $f, g \in \F^I$ se establece que $(f + g)(i) = f(i) + g(i)$. Definimos el producto por un escalar de la siguiente manera: dado $f \in \F^I$ y $c \in \F$, establecemos que $(c \cdot f )(i) = c \bigl( f(i) \bigr)$. Demuestre que con las operaciones definidas $\F^I$ tiene una estructura de $\F$-espacio vectorial.
  
  \item Considere a los números complejos. Dado $\alpha \in \C$ demuestre que $\alpha \R$ es un $\R$-subespacio vectorial de $\C$, pero que no es un $\C$-espacio vectorial.
  
  \item \label{exer:PolyGradLeqn} Denotemos como $\F[x]_{\leq n}$ al conjunto de polinomios con coeficientes en $\F$ y que tienen grado menor o igual a $n$. Demuestre que $\F[x]_{\leq n}$ es un $\F$-subespacio vectorial de $\F[x]$.
  
  \item Sea $U$ un $\F$-subespacio vectorial de $V$, demuestre que $0 \in U$.
  
  \item Sea $U$ un subconjunto no vacío del $\F$-espacio vectorial $V$. Demuestre que $U$ es un $\F$-subespacio vectorial si y solo si se cumplen las siguientes dos condiciones:
    \begin{enumerate}
      \item $\F U \subseteq \F U$, donde $\F U = \{cu : c \in \F \land u \in U\}$.
      \item $(U + U) \subseteq U$, donde $U+U = \{u_1 + u_2 \colon u_1, u_2 \in U\}$.
    \end{enumerate}

  \item Sea $U$ un $\F$-subespacio vectorial de $V$ y sea $W$ un $\F$-subespacio vectorial de $U$, demuestre que $W$ es un $\F$-subespacio vectorial de $V$.
  
  \item \label{exer:MatULD} Considere al $\F$-espacio vectorial $\M_{n,n} ( \F )$, al que denotaremos (solo un subíndice) $\M_n(\F)$. Sea $\bec U$ el subconjunto de todas las matrices triangulares superiores de $\M_n(\F)$, es decir que si $(a_{ij}) \in \bec U$ entonces $i > j$ implica que $a_{ij} = 0$. Similarmente sea $\bec L$ el subconjunto de todas las matrices triangulares inferiores de $\M_n(\F)$, y con $\bec D$ denotemos a las matrices diagonales. Demuestre que $\bec U$, $\bec L$ y $\bec D$ son $\F$-subespacios vectoriales de $\M_n(\F)$.
  
  \item Sea $V$ un $\F$-espacio vectorial, y sea $U_i$ un $\F$-subespacio vectorial de $V$ para cada $i \in I$, donde $I$ es un conjunto de índices no vacío. Demuestre que $U = \bigcap_{i \in I} U_i$ es un $\F$-subespacio vectorial de $V$ y un $\F$-subespacio vectorial de cada $U_i$.
  
  \item \label{exer:SubEsGene} Sea $S$ un subconjunto del $\F$-espacio vectorial $V$, y denote por $\inner{S}$ a la intersección de todos los $\F$-subespacios vectoriales de $V$ que contienen a $S$. Por el ejercicio previo sabemos que $\inner{S}$ es un $\F$-subespacio vectorial de $V$, y notemos que si $S$ es el conjunto vacío entonces $\inner{S} = \{ 0 \}.$ Demuestre que $\inner{S}$ es el mínimo $\F$-subespacio vectorial de $V$ que contiene a $S$.
  
  \item \label{exer:SubEsCombLin} Sea $S$ un subconjunto no vacío del $\F$-espacio vectorial $V$. Demuestre que para cualquier $v \in \inner{S}$ existen algún entero $n$, vectores $s_1, s_2, \ldots, s_n \in S$ y coeficientes $c_1, c_2, \ldots, c_n \in \F$ tales que $v = \sum _{i=1}^n c_i s_i$. Es por la propiedad mencionada que $\inner{S}$ es llamado el \emph{$\F$-subespacio vectorial generado por $S$.}
  

  \item Consideremos los $\F$-subespacios vectoriales $U_1$ y $U_2$ de $V$. Demuestre que $U_1 + U_2$ es un $\F$-subespacio vectorial de $V$.
  
  \item \label{exer:F^IFinito} Sea $\F^{I}$ como se mencionó en el ejercicio \ref{exer:F^I}, y sea $\F^{(I)}$ el subconjunto de funciones que al evaluarse son cero en casi todos sus valores, es decir que si $f \in \F^{(I)}$ entonces $f(i) \neq 0_\F$ solo para una cantidad finita de elementos $i \in I$. Demuestre que $\F^{(I)}$ es un $\F$-subespacio vectorial de $\F^{I}$.
  
  \item Determine $\dim_\F(\F)$; justifique su respuesta.
  
  \item Sea $\F[x]_{\leq n}$ como se mencionó en el ejercicio \ref{exer:PolyGradLeqn}: determine su dimensión exhibiendo una base.
  
  \item El campo de los complejos tiene una estructura canónica como $\C$-espacio vectorial, como $\R$-espacio vectorial
  y como $\Q$-espacio vectorial. Determine la dimensión de $\C$ como $\C$-espacio vectorial y como $\R$-espacio vectorial, exhibiendo bases correspondientes a cada caso.
  
  \emph{Reto:} Demuestre que la dimensión de $\C$ como $\Q$-espacio vectorial no es finita.
  
  \item Determine $\dim_{\F}\bigl( \M_{n,m}(\F) \bigr)$.
  
  \item Sean $\M_n(\F)$, $\bec U$, $\bec L$ y $\bec D$ como se mencionó en el ejercicio \ref{exer:MatULD}. Calcule la dimensión de cada uno de los $\F$-subespacios vectoriales.

  \item Considere el contexto del ejercicio previo. Demuestre que $\bec U + \bec L = \M_n(\F)$.
  
  \item Sea $\bec U_e$ el $\F$-subespacio vectorial de las matrices triangulares estrictamente superiores, es decir que si  $(a_{ij}) \in \bec U_e$ entonces $i \geq j$ implica $a_{ij} = 0$ y similarmente $\bec L_e$ denota al $\F$-subespacio vectorial de las matrices triangulares estrictamente inferiores. Demuestre que $\bec U_e \oplus \bec D \oplus \bec L_e = \M_n(\F)$.
  
  \item Sea $V$ un $\F$-espacio vectorial de dimensión finita y sean $U_1, U_2, \ldots, U_k$ un conjunto de $\F$-subespacios vectoriales de $V$. Supongamos que $V = U_1 \oplus U_2 \oplus \cdots \oplus U_k$; demuestre que $\dim_\F(V) = \sum _{i=1}^k \dim_\F(U_i)$.
  
  \item Provea un ejemplo de la siguiente situación: $U_1$, $U_2$, $U_3$ son $\F$-subespacios vectoriales de $V$ tales que $i \neq j$ implica $U_i \cap U_j = \{0\}$, pero $U_1 + U_2 + U_3$ no es una suma directa.
  
  \item Sea $\F^I$ como se mencionó en el ejercicio \ref{exer:F^I}: para el caso $I$ finito exhiba alguna base y determine su dimensión.
  
  \item Sea $\F^{(I)}$ como se mencionó en el ejercicio \ref{exer:F^IFinito}: exhiba alguna base y determine su dimensión (para cualquier $I$).
  
  \item Sea $f \colon \N \to \Q$ una biyección. Considere los siguientes subconjuntos de $\Q$: $G_1 = \Q - \{ f(1)\}$, $G_2 = \{ f(1), f(2) \}$, \dots, $G_n = \Q - \{ f(1), f(2), \ldots, f(n) \}$, etc. Demuestre que cada $G_i$ es un conjunto generador de $\Q$ como $\Q$-espacio vectorial. Demuestre que $\bigcap _{i \in \mathbb{N}} G_i = \emptyset$.
  
  \item Sea $G$ un subconjunto generador finito del $\F$-espacio vectorial $V \neq \{0\}$ (es decir que $V$ es no trivial). Demuestre que existe algún subconjunto de $G$ que es una base de $V$.
  
  \item Sea $T\colon V \to W$ una transformación $\F$-lineal.
    \begin{enumerate}
      \item Demuestre que $\ker(T)$ es un $\F$-subespacio vectorial de $V$.
      \item Demuestre que $T$ es inyectiva si y solo si $\ker(T) = \{0\}$.
      \item Demuestre que $\Im(T)$ es un $\F$-subespacio vectorial de $W$.
    \end{enumerate}

  \item Sea $T\colon V \to W$ una transformación $\F$-lineal y sean $U_1$ y $U_2$ $\F$-subespacios vectoriales de $V$. Demuestre que:
    \begin{enumerate}
      \item $T(U_1 + U_2) = T(U_1) + T(U_2)$.
      \item Supongamos que $T$ es inyectiva y que $U_1 \oplus U_2$. Entonces $T(U_1) \oplus T(U_2)$.
    \end{enumerate}

  \item Sean $T_1\colon V \to W$ y $T_2\colon W \to U$ transformaciones $\F$-lineales. Demuestre que $T_2 \circ T_1 \colon V \to U$ es una transformación $\F$-lineal.
  
  \item Sea $T\colon V \to W$ una transformación $\F$-lineal biyectiva: demuestre que $T^{-1}\colon W \to V$ es una transformación $\F$-lineal biyectiva.
  
  \item Demuestra que $L(V,W)$ es un $\F$-espacio vectorial.
\end{exerciselist}

\section{Coordenadas y cambio de base}

Como ya mencionamos en la sección anterior, dada una base existe una única combinación lineal de elementos de la base para cada vector del espacio, esto nos permite generalizar el concepto de coordenadas. En esta sección abordaremos a más detalle este tema.


\begin{defi}
  Sea $V$ un $\F$-espacio vectorial de dimensión $n$, decimos que $B = (v_1, v_2, \ldots, v_n)$ es una \emph{base ordenada} si $\{v_1, v_2, \ldots, v_n\}$ es una base de $V$. Además, para todo $v \in V$ donde $v = \lambda_1 v_1 + \cdots + \lambda_n v_n$, definiremos su \emph{vector coordenada con respecto a $B$} como
    \[ [v]_B = \begin{pmatrix}
      v_1 \\ \vdots \\ v_n
    \end{pmatrix}.\]
\end{defi}

Notemos que bajo esta definición $(v_1, v_1, \ldots, v_n)$ y $(v_2, v_1, \ldots, v_n)$ son dos bases ordenadas distintas, al igual que cualquier otra permutación, aunque provengan de la misma base. De este modo nos aseguraremos que la coordenada de cualquier elemento sea única.

\begin{example}
  Consideremos el espacio vectorial $\R[x]_{\leq 2}$ junto la base ordenada $B = (1, x+1, x^2+1)$ y calculemos los vectores coordenadas con respecto a $B$ de: $x^2+5x-1$, $x^2 + 2x+1$ y $ax^2 + bx + c$.

  \examplesolution 

  Para el primero, recordemos que por definición la coordenada de $x^2+5x-1$ sobre $B$ es el vector $(\lambda_1, \lambda_2, \lambda_3)^t \in \R^3$ tal que
    \begin{align*}
      x^2+5x-1 &= \lambda_1 (1) + \lambda_2(x+1) + \lambda_3(x^2+1) \\
        &= \lambda_3 x^2 + \lambda_2x + (\lambda_1 + \lambda_2 + \lambda_3).
    \end{align*}  
    Igualando los coeficientes obtenemos un sistema de ecuaciones, resolviéndolo tenemos que $\lambda_3 = 1$, $\lambda_2 = 5$ y $\lambda_1 = -1 - \lambda_2 - \lambda_3 = -7$, así tenemos que
      \[
        [x^2+5x-1]_B = \begin{pmatrix} -7 \\ 5 \\ 1 \end{pmatrix},
      \]
    
    El proceso es el mismo para los polinomios $x^2 + 2x+1$ y $ax^2 + bx + c$, por lo que sus coordenadas son
    \[
        [x^2+2x+1]_B = \begin{pmatrix} -2 \\ 2 \\ 1 \end{pmatrix}
          \Eqand
        [ax^2+bx+c]_B = \begin{pmatrix} c-a-b \\b \\ a \end{pmatrix}.
      \]
\end{example}

Una propiedad interesante es que esta asociación de vectores de $V$ con las $n$-tuplas de $\F^n$ forma una transformación lineal biyectiva.

\begin{prop}
  Sea $B$ una base ordenada de un $\F$-espacio vectorial $V$ de dimensión $n$, entonces la función $[\cdot]_B\colon V \to \F^n$ es un isomorfismo.
\end{prop}
\begin{proof}
  Sean $v,w \in V$ con $v = \lambda_1 v_1 + \cdots + \lambda_n v_n$, $w = \mu_1 v_1 + \cdots + \mu_n v_n$ y $\alpha \in \F$, entonces veamos que
  \[
  \alpha v+w =  (\alpha\lambda_1+\mu_1) v_1 + \cdots + (\alpha\lambda_n+\mu_n) v_n,
  \]
  de esta forma es claro que
  \begin{align*}
    [\alpha v+w]_B 
      &= \begin{pmatrix} \alpha\lambda_1+\mu_n \\ \vdots \\ \alpha\lambda_n+\mu_n \end{pmatrix} \\
      &= \alpha\begin{pmatrix} \lambda_1 \\ \vdots \\ \lambda_n \end{pmatrix}
       + \begin{pmatrix} \mu_n \\ \vdots \\ \mu_n  \end{pmatrix} \\
      &= \alpha[v]_B + [w]_B.
  \end{align*}
  Mostrando así, que es una transformación lineal. Para la inyectividad, es claro que si $[v]_B = (0,\cdots,0)$ entonces $v = 0v_1 + \cdots + 0v_n = 0$, por lo que $\ker([\cdot]_B) = \{0\}$. Para la suprayectividad, claramente tenemos que para todo $(\lambda_1, \ldots, \lambda_n)^t \in \F^n$ se cumple que $\lambda_1 v_1 + \cdots + \lambda_n v_n \in V$ y  $[\lambda_1 v_1 + \cdots  + \lambda_n v_n]_B = (\lambda_1, \ldots, \lambda_n)^t$.
\end{proof}


\subsection{Cambio de base}

Sabemos que dadas dos bases ordenadas $B = (v_1,\ldots,v_n)$ y $B' = (v_1',\ldots,v_n')$ de $V$ es posible definir un sistema de coordenadas para cada una, pero dado que modelan el mismo espacio, debería existir una forma de asociar ambas coordenadas, es decir, debería existir una función tal que dado el vector de coordenadas de $v \in V$ con respecto a $B$ nos devuelva su vector coordenada con respecto a $B'$. Más precisamente, buscamos que para $B$ y $B'$ exista una matriz invertible $M$ tal que $M [v]_B = [v]_{B'}$ para todo $v \in V$. A esta matriz la denotaremos como $M_{BB'}$ y la llamaremos como matriz de \emph{cambio de base} o \emph{cambio de coordenadas}.

La idea es la siguiente, si $e_i$ es la $i$-ésima columna de la matriz identidad, entonces por definición $[v_i]_B = e_i$, de este modo si $M_{BB'} [v]_B = [v]_{B'}$ para todo $v \in V$, entonces
  \[ M_{BB'} [v_i]_B = M_{BB'} e_i = (M_{BB'})_{*i} = [v_i]_{B'},\]
donde $(M_{BB'})_{*i}$ representa la $i$-ésima columna de la matriz $M_{BB'}$.
Esto nos dice que $[v_i]_{B'}$ es la $i$-ésima columna de la matriz $M_{BB'}$, en otras palabras que
  \[ M_{BB'} = \begin{spmatrix}{c|c|c}  [v_1]_{B'} & \cdots & [v_n]_{B'}  \end{spmatrix}. \]

\begin{teor} \label{teor:CambioBase}
  Sea $V$ un $F$-espacio vectorial con bases ordenadas $B = (v_1,\ldots,v_n)$ y $B' = (v_1',\ldots,v_n')$, entonces existe una única matriz invertible $M_{BB'}$ tal que para todo $v \in V$ se cumple que
    \[ M_{BB'}[v]_B = [v]_{B'} \Eqand M_{BB'}^{-1}[v]_{B'} = [v]_B. \]
  Además $(M_{BB'})_{*i} = [v_i]_{B'}$ para todo $i \in \{1,\ldots,n\}$.
\end{teor}
\begin{proof}
  Primero, notemos que por propiedades conocidas y la definición de $M_{BB'}$ es claro que 
    \[ M_{BB'}[v_i]_B = M_{BB'} e_i =  (M_{BB'})_{*i} = [v_i]_{B'}. \]

  De este modo, sea $v \in V$ con $[v]_B = (\lambda_1,\ldots,\lambda_n)^t$, por la linealidad de $[\cdot]_{B'}$ y las propiedades ya mencionadas, tenemos que
    \begin{align*}
      v        &= \lambda_1 v_1 + \cdots + \lambda_n v_n, \\
      [v]_{B'} &= \lambda_1 [v_1]_{B'} + \cdots + \lambda_n [v_n]_{B'} \\
               &= \lambda_1 M_{BB'}[v_1]_B + \cdots + \lambda_n M_{BB'}[v_n]_B \\
               &= M_{BB'} [\lambda_1 v_1 + \cdots + \lambda_n v_n ]_B \\
               &= M_{BB'} [ v ]_B.
    \end{align*}
  De este modo $M_{BB'}[v]_B = [v]_{B'}$ para toda $v \in B$. Es fácil ver que si $M_{BB'}$ es invertible, entonces $M_{BB'}^{-1}[v]_{B'} = [v]_B$, así solo nos queda demostrar que $M_{BB'}$ es invertible.
  
  Consideremos el sistema de ecuaciones $M_{BB'}x = \bec 0$, como $[\cdot]_B$ es suprayectivo, entonces existe $v \in V$ tal que $[v]_B = x$, de este modo $M_{BB'}x = \bec 0$ implica que $M_{BB'}[v]_B = [v]_{B'} = \bec 0$. Ahora, recordemos por inyectividad de $[\cdot]_{B'}$ que $[v]_{B'} = \bec 0$ si y solo si $v = 0$, de este modo $x = [v]_{B} = \bec 0$, pero esto implica que $M_{BB'}x = \bec 0$ no tiene soluciones no triviales, así, por propiedades de las matrices, tenemos que $M_{BB'}$ es invertible.

  La unicidad se da por el análisis hecho al principio. Si existiera otra matriz $P$ con las mismas cualidades, entonces para todo $i \in \{1,\ldots, n\}$ cumple que
  \[ (M_{BB'})_{*i} = [v_i]_{B'} = P[v_i]_{B} = Pe_i = P_{*i},\]
  de este modo, dado que todas las columnas son iguales, se cumple que $P = M_{BB'}$.
\end{proof}

Este teorema nos afirma que si tenemos una base ordenada $B$, entonces para cualquier otra base ordenada $B'$ existe una matriz invertible asociada que realiza el cambio de coordenadas. La vuelta de esta proposición también es cierta, de este modo existe una biyección entre el conjunto de bases ordenadas y el conjunto de matrices invertibles.

\begin{prop}\label{prop:ExBase}
  Sea $P$ una matriz invertible de $n\times n$ con entradas en $\F$, $V$ un $\F$-espacio vectorial y $B = (v_1, \ldots, v_n)$ una base ordenada de $V$, entonces existe una única base ordenada $B'$ tal que $M_{B'B} = P$.
\end{prop}
\begin{proof}
  Si $P = (p_{ij})$ entonces definamos $B' = (v_1', \ldots, v_n')$ donde
    \[ v_j' = p_{1j}v_1 + \cdots +  p_{nj}v_n,     \qquad j \in \{1,\ldots, n\}. \]
  Notemos que $[v_i]_B = P_{*i}$ para toda $i \in 1,\ldots,n$, de este modo, si $B'$ fuese una base ordenada, entonces $P = M_{B'B}$, por el teorema anterior. Así probemos que $B'$ es una base ordenada.

  Sea $\lambda_1 v_i' + \cdots + \lambda_n v_n' = 0$, por linealidad de $[\cdot]_B$ y definición, tenemos que si $\lambda = (\lambda_1, \ldots, \lambda_n)^t$, entonces
  \begin{align*}
    [\lambda_1 v_i' + \cdots + \lambda_n v_n'] &= \bec 0 \\
    \lambda_1 [v_1']_B + \cdots + \lambda_n [v_n']_B &= \bec 0 ,\\
    \lambda_1 P_{*1} + \cdots + \lambda_n P_{*n} &= \bec 0 ,\\
    P\lambda &= \bec 0.
  \end{align*}
  Ahora, como $P$ es invertible por hipótesis, entonces la única solución al sistema $Px = \bec 0$ es la trivial, de este modo $\lambda = \bec 0$ y por tanto $B'$ es una base ordenada.

  Ahora, para la unicidad, supongamos que existe otra base ordenada $B'' = (v_1'', \ldots, v_n'')$ tal que $P = M_{B''B}$, entonces, para toda $i \in \{1,\ldots,n\}$ se cumple que
  \[ [v_i'']_B =  P[v_i'']_{B''} = Pe_i = P[v_i']_{B'} = [v_i']_B, \]
  así, por la biyectividad de $[\cdot]_B$, tenemos que $v_i'' = v_i'$ para toda $i \in \{1,\ldots,n\}$, mostrando así que $B'' = B'$.
\end{proof}

\begin{example}
  Consideremos las bases ordenadas $B = ( 1, it )$ y $B' = (t+1, t-1)$ de $\C[t]_{\leq 1}$ y calculemos la matriz de cambio de coordenadas $M_{BB'}$.

  \examplesolution

  Por el teorema \ref{teor:CambioBase} sabemos que $M_{BB'} = \bigl( [1]_{B'} \mid [it]_{B'} \bigr)$, y se tienen las identidades
  \[
    1 = \frac{1}{2} (t+1) - \frac{1}{2}(t-1)
      \Eqand
    it =  \frac{i}{2}(t+1) + \frac{i}{2}(t-1),
  \]
  de esta forma tenemos que 
  \[ M_{BB'} = \begin{pmatrix}
    1/2  & i/2 \\
    -1/2 & i/2 \\
  \end{pmatrix}. \]
\end{example}

\begin{example}
  Consideremos las bases ordenadas $B = \bigl( (0,0,1)^t, (0,1,0)^t, (1,0,0)^t \bigr)$ y $B' = \bigl( (1,1,0)^t, (-1, 1, 0)^t, (0,0,-1)^t \bigr)$ de $\R^3$ y calculemos la matriz de cambio de coordenadas $M_{BB'}$.

  \examplesolution
  
  Aunque podríamos calcular de manera tradicional la matriz de cambio de coordenadas usando la definición de coordenada, existe un método más sencillo en este caso. Consideremos la base ordenada canónica $C = (e_1, e_2, e_3)$, es bastante claro que si $v \in \R^3$ entonces $[v]_C = v$, de esta forma tenemos que
  \[
    M_{BC} = \begin{pmatrix}
      0 & 0 & 1 \\
      0 & 1 & 0 \\
      1 & 0 & 0
    \end{pmatrix} \Eqand
    M_{B'C} = \begin{pmatrix}
      1 & -1 & 0 \\
      1 & 1  & 0 \\
      0 & 0  & -1
    \end{pmatrix}
  \]

  Ahora, dado que $M_{CB'} = M_{B'C}^{-1}$ veamos que $M_{B'C}^{-1} M_{BC}[v]_B = M_{B'C}^{-1}[v]_C = [v]_{B'}$, y por la unicidad de la matriz de cambio de base, entonces
  \begin{align*}
    M_{BB'} &= M_{B'C}^{-1} M_{BC} \\
      &= \begin{pmatrix}
        1 & -1 & 0 \\
        1 & 1  & 0 \\
        0 & 0  & -1
        \end{pmatrix}^{-1}
        \begin{pmatrix}
        0 & 0 & 1 \\
        0 & 1 & 0 \\
        1 & 0 & 0
        \end{pmatrix}  \\
      &= \begin{pmatrix}
        1/2  & 1/2 & 0 \\
        -1/2 & 1/2 & 0 \\
        0    & 0   & -1
        \end{pmatrix}
        \begin{pmatrix}
        0 & 0 & 1 \\
        0 & 1 & 0 \\
        1 & 0 & 0
        \end{pmatrix} \\
      &= \begin{pmatrix}
        0 & 1/2  & 1/2  \\
        0 & 1/2  & -1/2 \\
        -1 & 0    & 0
        \end{pmatrix}
  \end{align*}
\end{example}



\ExerciseSection

\begin{exerciselist}
  \item Demuestra que $v_1 = (1, 1, 0, 0)^t$, $v_2 = (0, 0, 1, 1)^t$, $v_3 = (1, 0, 0, 4)^t$ y $v_4 = (0, 0, 0, 2)^t$ forman una base de $\R^4$. Halla las coordenadas de cada uno de los vectores de la base canónica con respecto de la base ordenada $B = (v_1, v_2, v_3, v_4)$.
  
  \item Halla el vector de coordenadas de $(1, 0, 1)$ en la base de $\C^3$ formada por los vectores $(2i, 1, 0)$, $(2, -1, 1)$, $(0, 1+i, 1-i)$, en ese orden.
  
  \item Sea $V$ una $\F$-espacio vectorial de diménsión finita $n$ con una base ordenada $B$. Si $S = \{w_1, \ldots, w_n\} \subset V$, demuestra que $S$ es linealmente independiente si y solo si la matriz $M = \bigl( [w_1]_B \mid \cdots \mid [w_n]_B\bigr)$ es invertible.
  
  \item Sea $B = (v_1, v_2, v_3)$ la base ordenada de $\R^3$ formada por
    \[ 
      v_1 = \begin{pmatrix} 1 \\ 0 \\ -1 \end{pmatrix}, \quad
      v_2 = \begin{pmatrix} 1 \\ 1 \\ 1 \end{pmatrix}, \quad
      v_3 = \begin{pmatrix} 1 \\ 0 \\ 0 \end{pmatrix}.
    \]
    ¿Cuales son las coordenadas del vector $(a,b,c)$ en la base ordenada $B$?

  \item Sea $W$ el subespacio de $\C^3$ generado por $v_1 = (1, 0, i)^t$ y $v_2 = (1+i, 1, -1)^t$.
    \begin{enumerate}
      \item Demostrar que $v_1$ y $v_2$ forman una base de $W$.
      \item Demostrar que los vectores $w_1 = (1, 1, 0)^t$ y $w_2 = (1, i, 1+i)^t$ pertenecen a $W$ y forman otra base de $W$.
      \item ¿Cuales son las coordenadas de $v_1$ y $v_2$ en la base ordenada $(w_1, w_2)$ de $W$?
    \end{enumerate}
  
  \item Sean $v = (x_1, x_2)^t$ y $w = (y_1, y_2)^t$ dos vectores de $\R^2$ tales que
    \[ x_1 y_1 + x_2 y_2 = 0, \quad x_1^2 + x_2^2 = y_1^2 + y_2^2 = 1. \]
    Demostrar que $B = (v, w)$ es una base ordenada de $\R^2$. Hallar las coordenadas del vector $(a,b)^t$ en la base ordenada $B$. (Las condiciones impuestas a $v$ y $w$ dicen, geométricamente, que $v$ y $w$ son perpendiculares y de longitud 1.)

  \item Sea $V$ es espacio vectorial sobre los números complejos de todas las funciones de $\R$ en $\C$, dea $f_1(x) = 1$, $f_2(x) = e^{ix}$,  $f_3(x) = e^{-ix}$
    \begin{enumerate}
      \item Demostrar que $f_1$, $f_2$, $f_3$ son linealmente independientes.
      \item Sea $g_1(x)= 1$, $g_2(x) = \cos x$, $g_3(x) = \sen x$. Hallar una matriz invertible $P = (p_{ij})$ de $3 \times 3$ tal que
        \[ g_j = \sum_{i=1}^3 p_{ij} f_i, \qquad j \in \{1, 2, 3\}. \]
    \end{enumerate}

  \item Sea $t$ un número real fijo y defínase $g_1(x) = 1$, $g_2(x) = x+t$ y $g_3(x) = (x+t)^2$. Demuestra que $B = (g_1, g_2, g_3)$ es una base ordenada de $\R[x]_{\leq 2}$. Si $f(x) = ax^2 + bx + c$ ¿Cuales son las coordenadas de $f$ en la base ordenada $B$?
\end{exerciselist}

\section{Matriz asociada a una transformación lineal}

Hemos visto que para cualquier espacio vectorial podemos crear un sistema de coordenadas a partir de una base ordenada, esto nos permite trabajar con los elementos del espacio de una forma más simple. El siguiente paso es poder hacer lo mismo con las transformaciones lineales.

Sabemos que dada una matriz $M$ de $m \times n$ entonces la función $T_M \colon \F^n \to \F^m$ dada por $T(v) = Mv$  es una transformación lineal, a la cual llamaremos la transformación \emph{inducida} por $M$. Ahora, buscamos hacer el mismo proceso pero de manera inversa, es decir, si consideremos dos $\F$-espacios vectoriales $V$ y $W$ con bases ordenadas $B = (v_1,\ldots,v_n)$ y $B' = (w_1,\ldots,w_m)$, respectivamente, si $T\colon V \to W$ es una transformación lineal, lo que buscamos es una matriz $M$ de tamaño $m\times n$ tal que $M[v]_B = [T(v)]_{B'}$, para todo $v \in V$. A esta matriz la denotaremos como $T_{BB'}$ y diremos que es la matriz \emph{asociada} a $T$.

La idea es análoga a la de cambio de base, dado que $[v_i] = e_i$ y $T_{BB'}e_i = (T_{BB''})_{*i}$ para todo $i \in \{1,\ldots,n\}$, si $T_{BB'}[v]_B = [T(v)]_{B'}$ para todo $v \in V$, entonces tenemos que
  \[ T_{BB'} [v_i]_B = T_{BB'} e_i = (T_{BB'})_{*i} = [T(v_i)]_{B'}. \]
Así vemos que $[T(v_i)]_{B'}$ es la $i$-ésima columna de la matriz $T_{BB'}$, y por lo tanto
  \[ T_{BB'} = \begin{spmatrix}{c|c|c}  [T(v_1)]_{B'} & \cdots & [T(v_n)]_{B'}  \end{spmatrix}. \]

\begin{teor} \label{teor:MatTrLin}
  Sean $V$ y $W$ dos $\F$-espacios vectoriales con bases ordenadas $B = (v_1,\ldots,v_n)$ y $W' = (w_1,\ldots,w_m)$, respectivamente, y sea $T\colon V \to W$ una transformación lineal, entonces existe una única matriz $T_{BB'}$ de tamaño $m\times n$ tal que para toda $v \in V$ se cumple que
    \[ T_{BB'} [v]_B = [T(v)]_{B'} \]
  Además $(T_{BB'})_{*i} = [T(v_i)]_{B'}$ para todo $i \in \{1,\ldots,n\}$.
\end{teor}
\begin{proof}
  Por el análisis hecho con anterioridad y definición, se cumple que
  \[ T_{BB'} [v_i]_B = T_{BB'} e_i = (T_{BB'})_{*i} = [T(v_i)]_{B'}. \]
  
  De este modo sea $v \in V$ tal que $[v]_B = (\lambda_1,\ldots,\lambda_n)^t$, por la linealidad de $[\cdot]_B$ y las propiedades de las matrices, tenemos que
  \begin{align*}
    T_{BB'}[v]_B &= T_{BB'}[\lambda_1v_1 + \cdots + \lambda_n v_n]_B \\
      &= \lambda_1 T_{BB'} [v_1]_B + \cdots + \lambda_n T_{BB'} [v_n]_B \\
      &= \lambda_1 [T(v_1)]_{B'} + \cdots + \lambda_n [T(v_n)]_{B'} \\
      &= [T(\lambda v_1  + \cdots + \lambda_n v_n )]_{B'} \\
      &= [T(v)]_{B'}.
  \end{align*}
  Así tenemos que $T_{BB'} [v]_B = [T(v)]_{B'}$ para toda $v \in V$. Ahora, supongamos que existe otra matriz $P$ con la misma propiedad. Por el análisis del principio, para toda $i \in \{1,\ldots,n\}$ tenemos que
  \[ (T_{BB'})_{*i} = [T(v_i)]_B = P[v_i]_B = Pe_i = P_{*i}, \]
  y dado que todas sus columnas son iguales, es claro que $P = T_{BB'}$.
\end{proof}

Una consecuencia directa de este teorema es que si consideramos las bases canónicas de $\F^n$ y $\F^m$, entonces toda transformación lineal $T\colon \F^n \to \F^m$ es de la forma $T(v) = M_T v$ donde
  \[ M_T = \begin{spmatrix}{c|c|c} T(e_i) & \cdots & T(e_n) \end{spmatrix}, \]
mostrando que toda transformación lineal es inducida por alguna matriz.

\begin{example}
  Considera el operador de derivación $\bec d_2 \colon \F[x]_{\leq 2} \to \F[x]_{\leq 1}$, definido como
    \[
      \bec d_2 (ax^2 + bx + c) = 2ax + b.
    \]
  Este operador es una transformación lineal, así que calculemos la matriz asociada sobre las bases ordenadas $B = (x^2, x, 1)$ y $B' = (1, x+1)$.

  \examplesolution
  
  Por el teorema \ref{teor:MatTrLin} sabemos que $(\bec d_2)_{BB'} = \bigl( [\bec d_2(x^2)]_{B'} \mid [\bec d_2(x)]_{B'} \mid [\bec d_2(1)]_{B'} \bigr)$. Ahora, veamos que
  \begin{align}
    \bec d_2(x^2) &= 2x = -2(1) + 2(x+1) \\
    \bec d_2(x)   &= 1  = 1(1) + 0(x+1) \\
    \bec d_2(1)   &= 0  = 0(1) + 0(x+1) \\
  \end{align}
  De este modo tenemos que 
    \[ (\bec d_2)_{BB'} = \begin{pmatrix}
      -2 & 1 & 0 \\
      2 & 0 & 0
    \end{pmatrix} \]
\end{example}

\begin{teor}\label{teo:isomTrMat}
  Sean $V$ y $W$ dos $\F$-espacios vectoriales con bases ordenadas $B = (v_1,\ldots,v_n)$ y $W' = (w_1,\ldots,w_m)$. Si consideramos la función $\Psi_{BB'}\colon L(V,W) \to \M_{m\times n}(\F)$ dada por $\Psi_{BB'}(T) = T_{BB'}$, entonces $\Psi_{BB'}$ es un isomorfismo de $L(V,W)$ y $\M_{m\times n}(\F)$.
\end{teor}
\begin{proof}
  Primero demostremos que la función $\Psi_{BB'}$ es una transformación lineal. Sean $T, S \in L(V,W)$ y $\lambda \in \F$, notemos que para todo $v \in V$ se cumple que
  \begin{align*}
    (T_{BB'} + \lambda S_{BB'})[v]_B &= T_{BB'}[v]_B + \lambda S_{BB'} [v]_B \\
      &= [T(v)]_{B'} + \lambda [S(v)]_{B'} \\
      &= [T(v) + \lambda S(v)]_{B'} \\
      &= [ (T + \lambda S)(v)]_{B'}.
  \end{align*}
  Pero por la unicidad de la matriz asociada, esto implica que $(T + \lambda S)_{BB'} = T_{BB'} + \lambda S_{BB'}$, en otras palabras, que
  \[ \Psi_{BB'}(T + \lambda S) = \Psi_{BB'}(T) + \lambda \Psi_{BB'}(S). \]

  Ahora, notemos que si $\Psi_{BB'}(T) = \bec 0_{m\times n}$, entonces, por el teorema anterior, tenemos para todo $v \in V$ que  
    \[ T_{BB'}[v]_B  = [T(v)]_{B'} = \bec 0_m,\]
  pero por propiedades conocidas sabemos que $[v]_{B'} = \bec 0_m$ si y solo si $v = 0_W$, de esta forma $T(v) = 0_W$, pero esto implica que $\ker(\Psi_{BB'}) = \{  0_{L(V,W)} \}$, mostrando así que $\Psi_{BB'}$ es inyectiva.

  Por ultimo, si $P \in \M_{m\times n}(\F)$ donde $P = (p_{ij})$, dado que podemos definir una transformación lineal únicamente dando los valores para una base, consideremos la transformación $T\colon V \to W$ dada por
  \[ T(v_i) =  p_{1i}w_1 + \cdots + p_{mi}w_m, \]
  para todo $i \in \{1,\ldots,m\}$. Aplicando el teorema anterior tenemos que $T_{BB'} = P$, de este modo $\Psi_{BB'}(T) = P$, mostrando así que $\Psi_{BB'}$ es sobreyectiva y por tanto un isomorfismo de $L(V,W)$ a $\M_{m\times n}(\F)$.
\end{proof}

La relación de las matrices y las transformaciones lineales va más allá de que sean isomorfos como espacios vectoriales, además existe una relación entre la multiplicación de matrices y la composición de transformaciones lineales.

\begin{teor}
  Sean $V$, $W$ y $U$ tres $\F$-espacios vectoriales con bases ordenadas $B = (v_1,\ldots,v_n)$, $B' = (w_1,\ldots,w_m)$ y $B'' = (u_1,\ldots,u_p)$. Si $T\colon V \to W$ y $S\colon W \to U$ son transformaciones lineales, entonces
  \[ (S \circ T)_{BB''} = S_{B'B''} T_{BB'}. \]
\end{teor}
\begin{proof}
  Sea $v \in V$, notemos por definición que
    \begin{align*}
      S_{B'B''} T_{BB'}[v_B] &= S_{B'B''}[T(v)]_{B'} \\
        &= \bigl[ S\bigl(T(v)\bigr) \bigr]_{B''} \\
        &= [(S\circ T)(v)]_{B''}.
    \end{align*}
  Pero por la unicidad de la matriz asociada, esto implica que $(S \circ T)_{BB''} = S_{B'B''} T_{BB'}$.
\end{proof}

\begin{coro}
  Sean $V$ y $W$ dos $\F$-espacios vectoriales de dimensión finita con bases ordenadas $B = (v_1,\ldots,v_n)$ y $B' = (w_1,\ldots,w_m)$, respectivamente, y sea $T\colon V \to W$ una transformación lineal, entonces $T$ es invertible si y solo si $T_{BB'}$ es invertible, además
    \[ T_{B'B}^{-1} = (T_{BB'})^{-1}. \]
\end{coro}
\begin{proof}
  Dado que $T$ es invertible entonces es biyectiva, por lo que $m = n$. Así, por el teorema anterior, tenemos que
  \[ T^{-1}_{B'B} T_{BB'} = (T^{-1}\circ T)_{BB} = (\Id_V)_{BB} = I_n, \]
  pero esto impĺica que $T_{B'B}^{-1} = (T_{BB'})^{-1}$, por definición.

  Ahora, si $T_{BB'}$ es invertible entonces $m = n$ y por el teorema \ref{teo:isomTrMat} existe una transformación lineal $S\colon W \to V$ tal que $S_{B'B} =  (T_{BB'})^{-1}$. Aplicando el teorema anterior, esto implica que
    \[ (S \circ T)_{BB} =  S_{B'B}T_{BB'} = I_n = (\Id_V)_{BB}. \]
  De nuevo, por el teorema \ref{teo:isomTrMat}, tenemos que $S \circ T = \Id_{V}$ y análogamente se tiene que $T \circ S = \Id_{V}$, mostrando así que $T$ es invertible.
\end{proof}


\ExerciseSection

\begin{exerciselist}
  \item Sea $T$ la transformación lineal sobre $\C^2$ definida como $T(x_1, x_2) = (x_1, 0)$. Sea $C$ la base ordenada canónica de $\C^2$ y $B = (v_1, v_2)$ la base ordenada definida por $v_1 = (1, i)^t$ y $v_2 = (-i, 2)^t$.
    \begin{enumerate}
      \item ¿Cual es la matriz de $T$ con respecto a las bases ordenadas $C$ y $B$?
      \item ¿Cual es la matriz de $T$ con respecto a las bases ordenadas $B$ y $C$?
      \item ¿Cual es la matriz de $T$ con respecto a la base ordenada $B$?
      \item ¿Cual es la matriz de $T$ con respecto a la base ordenada $(v_2, v_1)$?
    \end{enumerate}

  \item Sea $T$ la transformación lineal de $\R^3$ en $\R^2$ definida por 
    \[ T\begin{pmatrix} x_1 \\ x_2 \\ x_3 \end{pmatrix} = \begin{pmatrix} x_1 + x_2 \\ 2x_3 - x_1 \end{pmatrix}. \]
    \begin{enumerate}
      \item Si $C$ es la base canónica de $\R^3$ y $C'$ es la base canónica de $\R^2$, ¿cuál es la matriz de $T$ con respecto a las bases ordenadas $C$ y $C'$?
      \item Si $B = (v_1, v_2, v_3)$ y $B' = (w_1, w_2)$, donde
        \[
          v_1 = \begin{pmatrix} 1 \\ 0 \\ -1 \end{pmatrix}, \quad
          v_2 = \begin{pmatrix} 1 \\ 1 \\ 1 \end{pmatrix}, \quad
          v_3 = \begin{pmatrix} 1 \\ 0 \\ 0 \end{pmatrix}, \quad
          w_1 = \begin{pmatrix} 0 \\ 1 \end{pmatrix}, \quad
          w_2 = \begin{pmatrix} 1 \\ 0 \end{pmatrix}
        \]
        ¿Cual es la matriz de $T$ respecto a las bases ordenadas $B$ y $B'$?
    \end{enumerate}

  \item Sea $T$ una transformación lineal sobre $\F^n$, sea $A$ la matriz de $T$ en la base ordenada canónica de $\F^n$ y sea $W$ es subespacio de $\F^n$ generado por los vectores columna de $A$, ¿qué relación existe entre $W$ y $T$?
  
  \item Sea $V$ un $\F$-espacio vectorial de dimensión 2 y sea $B$ una base ordenada de $V$. Si $T$ es una transformación lineal en $V$ y 
    \[ T_{BB} = \begin{pmatrix}
      a & b \\ c & d
    \end{pmatrix}, \]
    demuestra que $T^2 - (a+d)T + (ad-bc)\Id = 0$.

  \item Sea $T$ la transformación lineal sobre $\R^2$ definido por
    \[ T\begin{pmatrix} x_1 \\ x_2 \end{pmatrix} = \begin{pmatrix} -x_2 \\ x_1 \end{pmatrix}. \]
    \begin{enumerate}
      \item ¿Cual es la matriz de $T$ en la base canónica de $\R^2$?
      \item ¿Cual es la matriz de $T$ en la base ordenada $B = (v_1, v_2)$, donde $v_1 = (1,2)^t$ y $v_2 = (1, -2)$?
      \item Demuestra que para todo $c \in T$ la transformación $T - c\Id$ es invertible.
    \end{enumerate}
    
  \item Sea $T$ la transformación lineal en $\R[x]_{\leq 2}$ definida por
    \[ T(ax^2+bx+c) = (3a+c)x^2 + (b-2a)x + (2b-a+4c). \]
    \begin{enumerate}
      \item ¿Cual es la matriz de $T$ en la base ordenada $(x^2, x, 1)$?
      \item ¿Cual es la matriz de $T$ en la base ordenada $(f_1, f_2, f_3)$ donde $f_1(x) = x^2-1$, $f_2(x) = -x^2+2x+1$ y $f_3(x) = 2x^2 +x +1$?
      \item Demuestra que $T$ es invertible y da una expresión de $T^{-1}$ similar a como se definió $T$.
    \end{enumerate}

  \item Consideremos la transformación lineal $\bec d_n \colon \F[x]_{\leq n} \to \F[x]_{\leq n}$ dada por
    \[ \bec d_n \paren{\sum_{i=0}^n c_ix^i} = \sum_{i=1}^n ic_ix^{i-1}. \]
    Calcule la matriz de $\bec d_n$ sobre la base ordenada $B = (x^n, \ldots, x, 1)$.
\end{exerciselist}

\section{Semejanza de matrices}

Hemos visto que podemos asociar una matriz a cada transformación siempre que definamos unas bases ordenadas. El siguiente paso, es ver cómo obtener la matriz asociada de la misma transformación pero con bases diferentes. Esto lo haremos con la matriz de cambio de base.

\begin{teor}
  Sea $V$ y $W$ dos $\F$-espacios vectoriales y $T \colon V \to W$ una transformación lineal. Si $B_1$ y $B_2$ son bases ordenadas de $V$, $B_1'$ y $B_2'$ son bases ordenadas de $W$ entonces
  \[T_{B_2 B_2'} = M_{B_1' B_2'} T_{B_1 B_1'}M_{B_2 B_1}. \]
\end{teor}
\begin{proof}
  Sea $v \in V$, notemos, por definición y propiedades conocidas, que
  \begin{align*}
    M_{B_1' B_2'} T_{B_1 B_1'}M_{B_2 B_1} [v]_{B_2} &= M_{B_1' B_2'} T_{B_1 B_1'} [v]_{B_1} \\
      &= M_{B_1' B_2'} [T(v)]_{B_1'} \\
      &= [T(v)]_{B_2'}.
  \end{align*}
  Y por la unicidad de la matriz asociada, tenemos que $T_{B_2 B_2'} = M_{B_1' B_2'} T_{B_1 B_1'}M_{B_2 B_1}$.
\end{proof}

Notemos que en el teorema anterior las matrices $T_{B_1B_1'}$ y $T_{B_2B_2'}$ son distintas, pero que son matrices asociadas a la misma transformación. Lo mismo ocurre con con la matriz de cambio de base, notemos que si $B$ y $B$ son dos bases ordenadas de $V$ entonces $[\Id_V]_{BB'}[v]_B = [v]_B'$, por lo tanto $[\Id_V]_{BB'} = M_{BB'}$, dado que todas las matrices invertibles son de cambio de base, eso quiere decir que todas las matrices invertibles están asociadas a la transformación identidad, bajo unas ciertas bases.

Esta idea se puede generalizar: si $V$ y $W$ son $\F$-espacios vectoriales con bases ordenadas $B_1$ y $B_1'$, respectivamente, y $T\colon V \to W$ es una transformación lineal, si $M = PT_{B,B'}Q$ donde $P$ y $Q$ son dos matrices invertibles, por la proposición \ref{prop:ExBase} y el teorema anterior existirán bases ordenadas $B_2$ y $B_2'$ de $V$ y $W$, respectivamente, tales que $M = T_{B_2 B_2'}$. Esto quiere decir que todas las matrices de la forma $PT_{BB'}Q$, donde $P$ y $Q$ son dos matrices invertibles, están asociadas a $T$.

Un caso de especial interés son las matrices asociadas cuando $T$ es un endomorfismo de $V$ sobre la misma base. Si $B$ es una base ordenada de $V$, definiremos $T_B = T_{BB}$. Notemos que si deseamos obtener $T_{B'}$, dado que $M_{BB'} = (M_{B'B})^{-1}$ entonces 
  \[ T_{B'} = (M_{B'B})^{-1} T_B M_{B'B} \]
y de nuevo,por la proposición \ref{prop:ExBase} podemos ver que todas las matrices $P^{-1}T_{B}P$, donde $P$ es una matriz invertible, están asociadas a la misma transformación. De esta forma, la meta será agrupar todas las matrices que están asociadas a una misma transformación lineal.

\begin{defi}
  Sean $M$ y $N$ dos matrices de $n \times n$ con elementos en $\F$, se dice que $M$ es \emph{semejante} a $N$ si existe una matriz invertible $P$ tal que $M = P^{-1}NP$, lo que denotaremos como $M \sim N$.
\end{defi}

Como se puede intuir, la semejanza de matrices es una relación de equivalencia. Todas las clases de equivalencia tienen como común que las matrices que las componen están asociadas a una misma transformación lineal, además las clases de equivalencia de la matriz $\bec 0$ y la identidad $I_n$ están compuestas unicamente de éstas. 

\begin{prop} \label{prop:invdetytr}
  Sean $M,N \in \M_n(\F)$ tales que $M \sim N$, entonces $\det(M) = \det(N)$ y $\tr(M) = \tr(N)$.
\end{prop}
\begin{proof}
  Por definición, existe una matriz invertible $P$ tal que $M = P^{-1}NP$. La primera igualdad es fácil de probar, ya que por por propiedad de la determinante tenemos  que
  \begin{align*}
    \det(M) &= \det(P^{-1}NP) = \det(P^{-1})\det(N) \det(P)  \\
      &= \bigl(\det(P)\bigr)^{-1} \det(N) \det(P) = \bigl(\det(P)\bigr)^{-1}\det(P)\det(N)  \\
      &= \det(N).
  \end{align*}
  
  Para la segunda propiedad, notemos que si $A,B \in \M_n(\F)$ dado por $A = (a_{ij})$ y $B = (b_{ij})$ entonces
  \begin{align*}
    \tr(AB) &= \sum_{i=1}^n (AB)_{ii} = \sum_{i=1}^n \sum_{j=1}^n a_{ij} b_{ji} \\
      &=  \sum_{j=1}^n \sum_{i=1}^n b_{ji} a_{ij} = \sum_{j=1}^n (BA)_{jj} \\
      &= \tr(BA).
  \end{align*}
  De esta forma, tenemos que por definición y la propiedad anterior
  \[ \tr(M) = \tr(P^{-1}NP) = \tr([P^{-1}N]P) = \tr(P[P^{-1}N]) = \tr(I_n N) = \tr(N). \]
\end{proof}

Esta propiedad nos muestra que todas las matrices asociadas a una transformación lineal, y por tanto pertenecientes a la misma clases de equivalencia, tienen la misma determinante y traza, en otras palabras, la traza y determinante son invariantes ante el cambio de base.


\ExerciseSection

\begin{exerciselist}
  \item Sea $T$ la transformación lineal de $\R^3$ a $\R^2$ dada por $T\bigl((x_1, x_2, x_3)^t\bigr) = (x_1 - x_2, x_2 - x_3)^t$
    \begin{enumerate}
      \item Encuentre la matriz asociada a $T$ sobre la bases canónicas de $\R^3$ y $\R^2$.
      \item A partir de la matriz encontrada en el inciso anterior, encuentre la matriz asociada a $T$ con respecto a las bases $B = \bigl( (0, 1, 1)^t, (1, 0, 1)^t, (1, 1, 0)^t \bigr)$ y $B' = \bigl( (1, 0)^t, (1, 1)^t \bigr)$.
    \end{enumerate}

  \item Demuestra que la semejanza de matrices es una relación de semejanza. Es decir, demuestra las siguientes propiedades
    \begin{enumerate}
      \item (Reflexividad) $M \sim M$ para todo $M \in \M_n(\F)$.
      \item (Simetría) Para cualesquiera $M,N \in \M_n(\F)$ si $M \sim N$ entonces $N \sim M$.
      \item (Transitividad) Para cualesquiera $M,N,P \in \M_n(\F)$ si $M \sim N$ y $N \sim P$ entonces $M \sim P$.
    \end{enumerate}
  
  \item Demuestra que las clases de equivalencia de la matriz cero e identidad solo contienen a estas.
\end{exerciselist}

\section{Matrices triangularizables y diagonalizables} \label{sec:TyDdeMat2x2}

Uno de los objetivos de la semejanza de matrices es encontrar la matriz más ``simple'' asociada a una transformación lineal, para realizar operaciones con las matrices de manera más sencilla. Un especial interés se da a las matrices semejantes a matrices triangulares o diagonales.

\begin{defi}
  Sea $M$ una matriz, decimos que es \emph{triangularizable} (\emph{diagonalizable}) si existe una matriz $T$ triangular (diagonal) tal que $M \sim T$.
\end{defi}

A \emph{priori}, no tenemos una forma de saber si una matriz es triangularizable o diagonalizable. Hasta ahora, solo sabemos que dentro de una misma clase se preserva la determinante y traza, pero la vuelta no es cierta, aunque dos matrices compartan traza y determinante no necesariamente son semejantes. Aun necesitamos herramientas más sofisticadas para saber cuando una matriz es triangularizable o diagonalizable.

\subsection{Clasificación de matrices de \texorpdfstring{$2\times 2$}{2x2}}

Con los conocimientos que tenemos hasta este punto podemos clasificar las clases de equivalencia de $\M_2(\C)$. Por lo que en lo que resta de esta sección, nos dedicaremos a determinar cuando una matriz compleja de $2\times 2$ es triangularizable o diagonalizable.

Para ello, calculemos algunas conjugaciones. La primera conjugación nos permite ``intercambiar'' los elementos de una matriz.
\begin{equation}
  \begin{pmatrix} 0 & 1 \\ 1 & 0 \end{pmatrix}
  \begin{pmatrix} a & b \\ c & d \end{pmatrix}
  \begin{pmatrix} 0 & 1 \\ 1 & 0 \end{pmatrix}
    = \begin{pmatrix} d & c \\ b & a \end{pmatrix}. \label{eq:ConjI}
\end{equation}
Notemos que si $c = 0$ entonces la matriz es triangular superior, y aplicando esta conjugación obtenemos que es semejante a una matriz triangular inferior. Aplicando una conjugación similar es posible demostrar que toda matriz semejante a una matriz triangular superior es semejante a una matriz triangular inferior y viceversa. De este modo, para que una matriz sea triangularizable no importa si es semejante a una matriz triangular inferior o superior.

Para la siguiente conjugación, tomemos a cualquier $s \in \C$ y veamos que
\begin{equation}
  \begin{pmatrix} 1 & 0 \\ s & 1 \end{pmatrix}
  \begin{pmatrix} a & b \\ c & d \end{pmatrix}
  \begin{pmatrix} 1 & 0 \\ -s & 1 \end{pmatrix}
    = \begin{pmatrix} -bs+a & b \\ -bs^2 + (a-d)s + c  & bs+d \end{pmatrix}. \label{eq:ConjII}
\end{equation}
Ya que estamos trabajando sobre $\C$, notemos que si $b \neq 0$ entonces existe $s \in \C$ tal que $-bs^2 + (a-d)s + c = 0$, por lo que la matriz sería semejante a una triangular superior. En el caso que $b = 0$, la matriz ya es triangularizable y por lo ya comentando en la conjugación anterior podemos concluir que toda matriz de $\M_2(\C)$ es semejante a una matriz triangular superior. En otras palabras
\[ 
    \begin{pmatrix} a & b \\ c & d \end{pmatrix} \sim \begin{pmatrix} \alpha & \beta \\ 0 & \gamma \end{pmatrix}.
\]

Para la siguiente conjugación, consideremos a $s \in \C$ y para cualquier matriz triangular superior veamos que
\begin{equation}
  \begin{pmatrix} 1 & s \\ 0 & 1 \end{pmatrix}
  \begin{pmatrix} \alpha & \beta \\ 0 & \gamma \end{pmatrix}
  \begin{pmatrix} 1 & -s \\ 0 & 1 \end{pmatrix}
    = \begin{pmatrix} \alpha & (\gamma-\alpha)s + \beta \\ 0 & \gamma \end{pmatrix}.
\end{equation}
Notemos que si $\alpha \neq \gamma$  entonces existe $s \in \C$ tal que $(\gamma-\alpha)s + \beta = 0$, de este modo tenemos que las matrices pueden ser semejantes a una matriz diagonal con elementos distintos en la diagonal, o una matriz triangular con elementos en la diagonal iguales. Es decir, si $\alpha, \beta, \gamma \in \C$ entonces
\[
  \begin{pmatrix} a & b \\ c & d \end{pmatrix} \sim \begin{pmatrix} \alpha & 0 \\ 0 & \gamma \end{pmatrix}
    \Eqor
    \begin{pmatrix} a & b \\ c & d \end{pmatrix} \sim \begin{pmatrix} \alpha & \beta \\ 0 & \alpha \end{pmatrix}.
\]
Ahora, en la segunda equivalencia, si $\beta \neq 0$ entonces notemos que

\begin{equation}
  \begin{pmatrix} 1 & 0 \\ 0 & \beta \end{pmatrix}
  \begin{pmatrix} \alpha & \beta \\ 0 & \alpha \end{pmatrix}
  \begin{pmatrix} 1 & 0 \\ 0 & \beta^{-1} \end{pmatrix}
    = \begin{pmatrix} \alpha & 1 \\ 0 & \alpha \end{pmatrix}.
\end{equation}

De esta forma, tenemos que toda matriz compleja de $2 \times 2$ es semejante a alguna de estas matrices
\begin{equation}
  \begin{pmatrix} a & b \\ c & d \end{pmatrix} \sim \begin{pmatrix} \alpha & 0 \\ 0 & \gamma \end{pmatrix}
    \Eqor
    \begin{pmatrix} a & b \\ c & d \end{pmatrix} \sim \begin{pmatrix} \alpha & 1 \\ 0 & \alpha \end{pmatrix}.
      \label{eq:M2Form}
\end{equation}
con $\alpha$ y $\gamma$ arbitrarios. Aún queda por determinar si hay repeticiones entre los tipos de matrices calculados. Para realizar el cálculo más fácil, definamos las siguientes matrices 
\[
  D_{\alpha,\beta} = \begin{pmatrix} \alpha & 0 \\ 0 & \beta \end{pmatrix}
     \Eqand
  T_\alpha = \begin{pmatrix} \alpha & 1 \\ 0 & \alpha \end{pmatrix}
\]

Primero notemos que si $M = P^{-1} N P$ entonces $M^2 = (P^{-1} N P)(P^{-1} N P) = P^{-1}N^2 P$, de igual forma $M^3 = (P^{-1}N^2 P) (P^{-1} N P) = P^{-1}N^3 P$, de manera inductiva se puede ver que para todo $k \in \N$ se cumple que
\[
  M^k = P^{-1} N^k P
\]
Usando esta propiedad, notemos que si $c_k M^k + \cdots + c_2 M^2 + c_1 M + c_0 I = \bec 0$ entonces
\begin{align*}
  \bec 0 &= c_k M^k + \cdots + c_2 M^2 + c_1 M + c_0 I \\
    &= c_k P^{-1}N^k P + \cdots + c_2 P^{-1}N^2 P + c_1 P^{-1}N P + c_0 I \\
    &= P^{-1} (c_k N^k + \cdots + c_2 N^2 + c_1 N + c_0 I) P.
\end{align*}
De aquí es claro que $c_k N^k + \cdots + c_2 N^2 + c_1 N + c_0 I = \bec 0$, lo que nos deja con la siguiente proposición.

\begin{prop} \label{prop:MPolySem}
  Sean $M, N \in \M_n(\F)$, si $M \sim N$ entonces $\sum_{i=0}^k c_k M^k = \bec 0$ si y solo si $\sum_{i=0}^k c_k N^k = \bec 0$, donde $M^0 = N^0= I_n$. \qed
\end{prop}

Primero, revisemos que las dos formas dadas en \eqref{eq:M2Form} a las que puede ser semejante una matriz son ajenas. Supongamos que existen $\alpha, \beta, \gamma \in \C$ tal que $ D_{\alpha, \beta} \sim T_\gamma $. Notemos que
\begin{align*}
  (T_\gamma - \gamma I)^2 
    &= \corch{ \begin{pmatrix} \gamma & 1 \\ 0 & \gamma \end{pmatrix} - \begin{pmatrix} \gamma & 0 \\ 0 & \gamma \end{pmatrix} }^2 
     = \begin{pmatrix} 0 & 1 \\ 0 & 0  \end{pmatrix}^2  \\
    &= \begin{pmatrix} 0 & 0 \\ 0 & 0  \end{pmatrix}
\end{align*}
y que $(T_\gamma - \gamma I)^2 = T_\gamma^2 - 2\gamma T_\gamma + \gamma^2 I$, de esta forma si $ D_{\alpha,\beta} \sim T_\gamma $, por la proposición \ref{prop:MPolySem} tenemos que
\begin{align*}
  \begin{pmatrix} 0 & 0 \\ 0 & 0  \end{pmatrix}
    &= D_{\alpha,\beta}^2 - 2\gamma D_{\alpha,\beta} + \gamma^2 I \\
    &= \begin{pmatrix} \alpha & 0 \\ 0 & \beta \end{pmatrix}^2 - 2\gamma \begin{pmatrix} \alpha & 0 \\ 0 & \beta \end{pmatrix} + \begin{pmatrix} \gamma^2 & 0 \\ 0 & \gamma^2 \end{pmatrix} \\
    &= \begin{pmatrix} \alpha^2 - 2\alpha\gamma + \gamma^2 & 0 \\ 0 & \beta^2 - 2\beta\gamma + \gamma^2 \end{pmatrix} \\
    &= \begin{pmatrix} (\alpha-\gamma)^2 & 0 \\ 0 & (\beta-\gamma)^2 \end{pmatrix} 
\end{align*}
de aquí tenemos que $\alpha = \beta = \gamma$, pero si esto se cumple, notemos que entonces
\[
  D_{\alpha,\beta} - \gamma I = \begin{pmatrix} \alpha & 0 \\ 0 & \beta \end{pmatrix} - \begin{pmatrix} \gamma & 0 \\ 0 & \gamma \end{pmatrix}
  = \begin{pmatrix} 0 & 0 \\ 0 & 0 \end{pmatrix}
\]
pero la proposición \ref{prop:MPolySem}, esto implicaría que $T_\gamma - \gamma I  = \bec 0$ y por tanto
\[
  \begin{pmatrix} 0 & 0 \\ 0 & 0 \end{pmatrix} = T_\gamma - \gamma I = \begin{pmatrix} \gamma & 1 \\ 0 & \gamma \end{pmatrix} - \begin{pmatrix} \gamma & 0 \\ 0 & \gamma \end{pmatrix} = \begin{pmatrix}
    0 & 1 \\ 0 & 0 \end{pmatrix},
\]
lo que claramente es una contradicción. De esta forma la clase de equivalencia de $D_{\alpha,\beta}$ es ajena a la clase de equivalencia de $T_\gamma$ para cualesquiera $\alpha,\beta,\gamma \in \C$.

Ahora, veamos qué pasa con las semejanza de matrices del mismo tipo. Supongamos que existen $\alpha, \beta, \gamma, \alpha', \beta', \gamma' \in \C$ tales que $D_{\alpha,\beta} \sim D_{\alpha',\beta'}$ y $T_\gamma \sim T_{\gamma'}$. Aplicando el mismo proceso, tenemos que $(D_{\alpha,\beta} - \alpha I)(D_{\alpha,\beta} - \beta I) = \bec 0$ y $(T_\gamma - \gamma I)^2 = \bec 0$, usando la proposición \ref{prop:MPolySem} sobre $D_{\alpha',\beta'}$ y $T_{\gamma'}$ y simplificando, se tiene que 
\[
  D_{\alpha,\beta} \sim D_{\alpha',\beta'} \iff \{\alpha,\beta\} = \{\alpha',\beta'\}
    \Eqand
  T_\gamma \sim T_{\gamma'} \iff \gamma = \gamma'.
\]

Resumiendo todo el proceso hecho hasta ahora, tenemos el siguiente teorema.
\begin{teor} \label{teor:TDMat2x2}
  Toda matriz $M \in \M_2(\C)$ es semejante a una matriz de la forma $D_{\alpha,\beta}$ o $T_\gamma$ con $\alpha, \beta, \gamma \in \C$. Además, $D_{\alpha,\beta} \nsim T_\gamma$ para cualesquiera $\alpha,\beta,\gamma \in \C$. Y por último, $D_{\alpha,\beta} \sim D_{\alpha',\beta'}$ si y solo si $\{\alpha,\beta\} = \{\alpha',\beta'\}$ y $T_\gamma \sim T_{\gamma'}$ si y solo si $\gamma = \gamma'$. \qed
\end{teor}

Este teorema nos dice dos propiedades importantes. En primer lugar toda matriz compleja de tamaño $2 \times 2$ es triangularizable o diagonalizable. En segundo lugar, toda la clase de equivalencia de una matriz está caracterizado por uno o dos valores, los cuales son los valores en la diagonal de la matriz semejante de la forma $D_{\alpha,\beta}$ o $T_\gamma$.

La existencia de estos valores no es una coincidencia, en el capítulo 2 generalizaremos el proceso hecho en esta sección y se podrá ver que estos números son llamados los \emph{valores propios} de la matriz.


\ExerciseSection

\begin{exerciselist}
  \item Demuestra que $M$ es semejante a una matriz triangular superior si y solo si es semejante a una matriz triangular inferior.
  
  \item Realiza los cálculos para demostrar que $D_{\alpha,\beta} \sim D_{\alpha',\beta'}$ si y solo si $\{\alpha,\beta\} = \{\alpha',\beta'\}$ y $T_\gamma \sim T_{\gamma'}$ si y solo si $\gamma = \gamma'$.

  \item Demuestra que si $M \sim D_{\alpha,\beta}$ con $\alpha,\beta \in \C$, entonces existen vectores no nulos $v, v' \in \C^2$ tales que $Mv = \alpha v$ y $Mv' = \beta v'$.
  
  \item Demuestra que si $M \sim T_\gamma$ con $\gamma \in \C$, entonces existe un vector no nulo $v\in \C^2$ tal que $Mv = \gamma v$.
  
  \item Usa la proposición \ref{prop:MPolySem} para demostrar que $M \sim D_{5, -5}$ donde
    \[ M = \begin{pmatrix}
      -1 & 2 \\
      12 & 1
    \end{pmatrix}. \]
  
  \item Usa la proposición \ref{prop:MPolySem} para demostrar que $M \sim T_{3}$ donde
  \[ M = \begin{pmatrix}
    4 & -1 \\
    1 & 2
  \end{pmatrix}. \]
\end{exerciselist}

\section{Proyecto -- Sucesión de Fibonacci}


Esta sección esta dedicada a desarrollar un ejemplo de uso de todos los temas que hemos visto hasta el momento y una introducción al siguiente capítulo.

Una sucesión muy conocida en las matemáticas es la llamada, sucesión de Fibonacci. Esta sucesión se define de manera recursiva como
\[
  F_0 = 0, \quad F_1 = 1, \quad F_{i+1} = F_i + F_{i-1}.
\]

Un problema con las sucesiones recursivas, es que para calcular un término hay que calcular todos los términos anteriores, de este modo, para este tipo de sucesiones siempre se busca un fórmula que nos permita calcular un término de manera directa. En el caso de la sucesión de Fibonacci, esta fórmula existe y es conocida como la \emph{fórmula de Binet}. El fin de este proyecto es encontrar la fórmula a partir de los que conocemos hasta ahora.

En primer lugar, notemos que podemos ver la sucesiones de Fibonacci como un sistema de ecuaciones dado por
\begin{equation}
  \left\{\begin{aligned}
    f_{n} + f_{n-1} &= f_{n+1} \\
    f_n             &= f_n
  \end{aligned}\right. 
    \implies
  \begin{pmatrix}
   1 & 1 \\ 1 & 0
  \end{pmatrix} \begin{pmatrix}  F_n \\ F_{n-1} \end{pmatrix}
    = \begin{pmatrix}  F_{n+1} \\ F_n \end{pmatrix}.
    \label{eq:FibMtx}
\end{equation}
Definamos como $M$ a la matriz asociada al sistema de ecuaciones de \eqref{eq:FibMtx}. Notemos que por definición, se cumple que
\[
  M^n \begin{pmatrix}  F_1 \\ F_0 \end{pmatrix} 
    = \begin{pmatrix} 1 & 1 \\ 1 & 0 \end{pmatrix}^n  \begin{pmatrix}  1 \\ 0 \end{pmatrix}
    = \begin{pmatrix}  F_{n+1} \\ F_n \end{pmatrix}.
\]

De esta forma tenemos una forma, relativamente directa, de calcular el $n$-ésimo término de la sucesión de Fibonacci. El problema entonces es calcular cuanto vale $M^n$. Por la proposición \ref{prop:MPolySem}, sabemos que si $M = P^{-1}NP$, entonces $M^n = P^{-1} N^n P$ y para hacer más sencillos los cálculos busquemos la matriz $D_{\alpha,\beta}$ o $T_\gamma$ a la que es semejante $M$.

Como ya dijimos antes, los valores de la diagonal la su matriz $D_{\alpha,\beta}$ o la matriz $T_\gamma$ son llamados los valores propios de la matriz, aunque hasta el siguiente capítulo se explicará con más detalle lo que esto significa, podemos adelantar que la forma más eficiente de encontrarlos, es ver qué valores $\lambda \in \C$ cumplen que $\det(M-\lambda I) = 0$. Así tenemos que los valores propios de $M$ son los $\lambda \in \C$ tales que
\[
  \det\begin{pmatrix} 1-\lambda & 1 \\ 1 & -\lambda \end{pmatrix} = \lambda^2 - \lambda - 1 = 0.
\]
Aplicando la fórmula general, tenemos que las raíces del polinomio son $\phi = (1+\sqrt{5})/2$ y $-\phi^{-1} = (1-\sqrt{5})/2$. Dado que son dos valores distintos, por el teorema \ref{teor:TDMat2x2}, entonces tenemos que $M \sim D$, donde
\[
  D = \begin{pmatrix}
    \phi & 0 \\
    0 & -\phi^{-1}
  \end{pmatrix}
\]

Ahora, dado que $M \sim D$, entonces debe existir una matriz invertible $P$ tal que $D = P^{-1} M P$. Para encontrar esta matriz $P$ notemos que
\begin{align*}
  De_1 &=  P^{-1} M Pe_1, \\
  \phi P_{*1} &= M P_{*1}, \\
  (M - \phi I)P_{*1} &= \bec 0.
\end{align*}
De manera análoga, se puede ver que $(M + \phi^{-1} I)P_{*2} = \bec 0.$ Si recordamos como obtuvimos a $\phi$ y $-\phi^{-1}$ es fácil ver que las ecuaciones $(M - \phi I)x = \bec 0$ y $(M + \phi^{-1} I)x = \bec 0$ tienen infinitas soluciones no triviales, en el capitulo 2 veremos que estos vectores son llamados los \emph{vectores propios} de $M$ asociados al valor propio. Cualquiera de estos vectores propios sirve, así que podemos elegir líbremente que $P_{*1} = (1, \phi^{-1})^t$ y $P_{*2} = (1, -\phi)^t$. Un cálculo rápido nos puede confirmar que $P$ es invertible y que
\[
  P^{-1} M P = 
  \frac{1}{\sqrt{5}}\begin{pmatrix} \phi & 1 \\ \phi^{-1} & -1 \end{pmatrix}
  \begin{pmatrix} 1 & 1 \\ 1 & 0 \end{pmatrix}
  \begin{pmatrix} 1 & 1 \\ \phi^{-1} & -\phi \end{pmatrix} =
  \begin{pmatrix} \phi & 0 \\ 0 & -\phi^{-1} \end{pmatrix} = D
\]

De aquí, y aplicando todo lo mencionado con anterioridad, tenemos que
\begin{align*}
  \begin{pmatrix} F_{n+1} \\ F_n \end{pmatrix}
    &= M^n \begin{pmatrix} 1 \\ 0 \end{pmatrix}  = P D^n P^{-1} \begin{pmatrix} 1 \\ 0 \end{pmatrix} \\
    &= \frac{1}{\sqrt{5}} \begin{pmatrix} 1 & 1 \\ \phi^{-1} & -\phi \end{pmatrix}
        \begin{pmatrix} \phi^n & 0 \\ 0 & (-\phi^{-1})^n \end{pmatrix}
        \begin{pmatrix} \phi \\ \phi^{-1} \end{pmatrix} \\
    &= \frac{1}{\sqrt{5}} \begin{pmatrix} 1 & 1 \\ \phi^{-1} & -\phi \end{pmatrix}
        \begin{pmatrix} \phi^{n+1} \\ (-1)^n \phi^{-(n+1)} \end{pmatrix} \\
    &= \frac{1}{\sqrt{5}} \begin{pmatrix}
          \phi^{n+1} + (-1)^n \phi^{-(n+1)} \\
          \phi^n + (-1)^{n+1} \phi^{-n}
        \end{pmatrix}.
\end{align*}
De todo lo anterior, finalmente hemos deducido que la Formula de Binet es
\[
  F_n = \frac{\phi^n + (-1)^{n+1} \phi^{-n} }{\sqrt{5}}.
\]

\ExerciseSection

\begin{exerciselist}
  \item Demuestra que $\lim_{n\to\infty} (F_{n+1}/F_n) = \phi$.
  
  \item ¿Puedes calcular la correspondiente fórmula para la sucesión de Lucas?
  \[
    L_0 = 2, \quad L_1 = 1, \quad F_{i+1} = F_i + F_{i-1}.
  \]

  \item \emph{Reto}. Aplicar un procedimiento similar para la sucesión: $0$, $1$, $2$, $5$, $12$, $29$, $70$, $169$, \dots (La regla de recursión es $S_{n+1} = 2S_{n} + S_{n-1}$).
\end{exerciselist}


% =============================================================================
% CAPÍTULO 2
% =============================================================================

\chapter{Teoría Espectral}

\section{Valores propios}

Como vimos en la sección \ref{sec:TyDdeMat2x2} las matrices complejas de $2\times 2$ tienen asociadas unos números que nos permiten caracterizar toda su clase de equivalencia y que nos permiten diagonalizarlas o triangularizarlas. Aunque el proceso lo hicimos para un tipo muy especial de matrices, este se puede generalizar.

Supongamos que tenemos una matriz $M\in\M_n(\F)$ diagonalizable, es decir, existe una matriz $D = (d_{ij})$ diagonal y una matriz invertible $P$ tal que $D =  P^{-1}MP$. Notemos que si $v_i = P_{*i}$ con $i \in \{1,\ldots,n\}$ y $D = \bigl( \lambda_1 e_1 \mid \cdots \mid \lambda_n e_n \bigr)$, dado que $PD = MP$, entonces se cumple que
  \[
    Mv_i = MPe_i = PD e_i = P(\lambda_i e_i) = \lambda_i P_{*i} = \lambda_i v_i
  \]
De este modo, si quisiéramos determinar si una matriz $M$ es diagonalizable, es natural pensar que deberíamos buscar los valores $\lambda$ y vectores $v$ tales que $Mv = \lambda v$.

\begin{defi}
  Sea $M \in \M_n(\F)$, decimos que $\lambda \in \F$ es un \emph{valor propio} de $M$ si existe un vector $v \in \F^n$ no nulo, tal que
    \[Mv = \lambda v.\]
  Si $\lambda$ es in valor propio de $M$ entonces cualquier $v \in \F^n$ no nulo tal que $Mv = \lambda v$ se dirá que es un \emph{vector propio} de $M$ asociado al valor propio $\lambda$. Al conjunto de valores propios de $M$ le llamaremos el \emph{espectro} de $M$ y lo denotaremos como $E(M)$.
\end{defi}

Notemos que si $M$ es la matriz cero, entonces todo vector de $\F^n - \{\bec 0\}$ es un vector propio de $M$ asociado al valor propio $0$. De manera más general, si $M$ es una matriz no invertible, entonces todo vector en $\ker(M)-\{\bec 0\}$ es un vector propio de $M$ asociado al valor propio $0$.

\begin{teor}\label{teor:PropVP}
  Sean $M \in \M_n(\F)$ y $\lambda \in \F$, las siguientes afirmaciones son equivalentes
  \begin{enumerate}
    \item $\lambda$ es un valor propio de $M$.
    \item La matriz $M-\lambda I$ no es invertible.
    \item $\det(M-\lambda I) = 0$.
  \end{enumerate}
\end{teor}
\begin{proof}
  Por las propiedades de la determinante, es claro que $M-\lambda I$ es no invertible si y solo si $\det(M-\lambda I)=0$. Así, demostremos que $\lambda$ es un valor propio de $M$ si y solo sí $M-\lambda I$ no es invertible.

  Para la ida, notemos que si $\lambda$ es un valor propio, entonces existe $v \in \F^n-\{\bec 0\}$ tal que 
    \[ Mv=\lambda v = \lambda Iv \implies (M-\lambda I)v = \bec 0. \]
  Dado que $v \neq \bec 0$, entonces la ecuación $(M-\lambda I)x = \bec 0$ tiene una solución no trivial, lo que implica que $M-\lambda I$ es no invertible.

  Análogamente para la vuelta, si $M-\lambda I$ es no invertible, entonces existe $v \in \F^n-\{\bec 0\}$ tal que $(M-\lambda I)v = \bec 0$ y por tanto
  \[ Mv - \lambda Iv = \bec 0 \implies Mv = \lambda v. \]
  Mostrando así que $\lambda$ es un vector propio de $M$.
\end{proof}

Este último teorema nos da una forma muy sencilla de calcular los valores propios de una matriz, basta con ver cuales son las soluciones de la ecuación $\det(M-\lambda I) = 0$. Más adelante veremos que $\det(M-\lambda I)$ conformará un polinomio el cual recibirá el nombre de \emph{polinomio característico de la matriz} el cual es bastante importante para encontrar la matriz más simple a la cual una matriz es semejante.

\begin{example}
  Consideremos la matriz $A$ con entradas reales de $3 \times 3$ dada por
  \[
    A = \begin{pmatrix} 3 & 1 & -1 \\ 2 & 2 & -1 \\ 2 & 2 & 0 \end{pmatrix}.
  \]
  Encontremos sus valores propios.

  \examplesolution

  Encontremos los valores propios de $A$ mediante el polinomio característico de este. Veamos que
  \[
    \det(M - \lambda I) = \det\begin{pmatrix} 3 - \lambda & 1 & -1 \\ 2 & 2 - \lambda & -1 \\ 2 & 2 & -\lambda \end{pmatrix}
      = -\lambda^3+5\lambda^2-8\lambda+4 = -(\lambda-1)(\lambda-2)^2.
  \]
  De esta forma, podemos ver que $\det(M - \lambda I) = 0$ si y solo si $\lambda=1$ o $\lambda = 2$, de este modo $E(A) = \{1, 2\}$. Para encontrar vectores propios asociados, basta con encontrar vectores no nulos en $\ker(M-\lambda I)$.
\end{example}

Como vimos en la sección \ref{sec:TyDdeMat2x2}, los valores propios de una matriz son iguales para todas las matrices semejantes a esta, aunque como se verá más adelante, no son suficientes para determinar a toda la clase de equivalencia. De igual forma, aunque para matrices complejas de $2\times 2$ siempre existen sus valores propios, no necesariamente todas las matrices tienen un valor propio.

\begin{teor}\label{teor:SemEspectro}
  Sean $M,N \in \M_n(\F)$, si $M \sim N$ entonces $E(M) = E(N)$.
\end{teor}
\begin{proof}
  Primero, veamos que por definición debe existir una matriz invertible $P$ tal que $M = P^{-1} N P$. Sea $\lambda \in E(M)$, por definición debe existir un vector no nulo $n \in \F^n$ tal que $Mv = \lambda v$. De esta forma, sustituyendo $M$, notemos que
    \[ ( P^{-1} N P) v =  \lambda v  \implies  N(Pv) = P(\lambda v) = \lambda (Pv). \]
  Ahora, dado que $P$ es invertible y $v$ es un vector no nulo, entonces por propiedades de las matrices invertibles, tenemos que $Pv$ será un vector no nulo, pero por definición, esto implica que $\lambda \in E(N)$. Así $E(M) \subseteq E(N)$.

  Como $N \sim M$ por simetría de la semejanza de matrices, entonces por la parte anterior tenemos que $E(N) \subseteq E(M)$, mostrando así que $E(N) = E(M)$.
\end{proof}

\section{El anillo de polinomios}

\section{El polinomio característico y minimal}

En el estudio de los valores y vectores propios de una matriz hay dos polinomio importantes asociados a cada matríz. Estos son el polinomio minimal, que nos dará una forma de encontrar los valores propios de una matriz. El otro polinomio nos será útil cuando estudiemos los espacios de núcleos.

\subsection{Polinomio característico.}

Para definir este polinomio, primero necesitamos recordar algunas de las propiedades del determinante. En primer lugar, el determinante no solo se puede definir sobre matrices con entradas en un campo $\F$, sino que es posible definirlo sobre cualquier anillo conmutativo.

Además, algunas de las propiedades de la determinante, demostradas sobre campos, también se conservan para anillos. Las más importantes son que para cualesquiera matrices $M$ y $N$ con entradas sobre un anillo, se cumple que $\det(AB) = \det(A)\det(B)$ y $A \adj(A) = \adj(A) A = \det(A) I$, donde $\adj(A)$ es traspuesta de la matriz de cofactores de $A$. Las demostraciones de estas propiedades va más allá de el alcance de este libro, pero su demostración es muy similar a cuando las entradas son elementos de un campo. Con todo esto, podemos definir el primer polinomio importante de una matriz.

\begin{defi}
  Sea $M \in \M_n(\F)$, considerando la matriz con entradas en $\F[x]$ dada por $M-xI$, entonces llamaremos el \emph{polinomio característico} de $M$ al polinomio
    \[ p(x) = \det(M-xI). \]
\end{defi}

Ahora, veamos que propiedades tiene este polinomio. Supongamos que tenemos una raíz $\lambda \in \F$ del polinomio, entonces $p(\lambda) = 0$ , pero eso implica que
\[ p(\lambda) = \det(M-\lambda I) = 0, \]
usando el teorema \ref{teor:PropVP} tenemos que entonces $\lambda \in E(M)$. Análogamente se puede probar que si $\lambda \in E(M)$ entonces $\lambda$ será una raíz del polinomio característico de $M$.

Otra propiedad interesante, es que si $M = P^{-1} N P$ entonces, por propiedades de la determinante tenemos que
\begin{align*}
  \det(M - xI) &= \det(P^{-1}NP - x P^{-1}IP) \\
    &= \det\bigl(P^{-1}(N- xI)P\bigr) \\
    &= \det(P^{-1})\det(N - xI) \det(P) \\
    &= \det(N - xI) \det(P) [\det(P)]^{-1} \\
    &= \det(N - xI).
\end{align*}
De esta forma, dos matrices semejantes comparten el mismo polinomio característico. Así podemos formular el siguiente teorema.

\begin{teor} \label{teor:PropPCaract}
  Sean $M, N \in \M_n(\F)$ tal que $M \sim N$ entonces las siguientes propiedades se sostienen.
  \begin{enumerate}
    \item $\lambda \in E(M)$ si y solo si es raíz de su polinomio característico.
    \item Si $M \sim N$ entonces tienen el mismo polinomio característico.
  \end{enumerate}
\end{teor}



\subsection{Polinomio minimal}

Éste es el segundo polinomio importante de una matriz. Aquí buscamos cual es el polinomio no nulo $f$ de grado más pequeño tal que anula a una matriz, es decir $f(M) = \bec 0$.

En primer lugar, notemos que debe existir, esto se cumple dado que si consideramos el conjunto $\{I,M,M^2,\ldots,M^{n^2}\}$, este es un conjunto con $n^2+1$ elementos del $\F$-espacio vectorial $M_n(\F)$ que tiene dimensión $n^2$. Así, este conjunto no puede ser linealmente independiente, por lo que existen $c_0,\ldots,c_n\in \F$ tal que 
\[ c_n M^n + \cdots + c_2 M^2 + c_1 M + c_0 I = \bec 0. \]
De esta forma, es claro que si $f = c_n x^n + \cdots + c_2 x^2 + c_1 x + c_0$ entonces $f$ no es el polinomio cero y además $f(M) = \bec 0$. Ahora, mostremos que existe este polinomio de grado mínimo que anula la matriz.

\begin{teor}\label{teor:UnicidadPMin}
  Sea $M \in M_n(\F)$. Existe un único polinomio $f_M \in \F[x]$, de grado mayor a cero tal que
  \begin{enumerate}
    \item $f_M(M) = \bec 0$.
    \item El coeficiente líder de $f_M$ es la unidad $1_\F$.
    \item Si $g \in \F[x]$ cumple que $g(M) = 0$, entonces $g = 0$ o $\grad(g) \geq \grad(f)$. Más aún, $f_M$ divide a $g$.
  \end{enumerate}
\end{teor}
\begin{proof}
  Por lo ya visto, sabemos que existe al menos un polinomio no nulo que anula la matriz, así consideremos el conjunto $I$ de todos los polinomios no nulos tal que anulan a la matriz. Así, consideremos algún polinomio $f \in I$ de grado mínimo.

  En primer lugar, si $f$ tiene coeficiente líder $c$, entonces es fácil ver que que si $f(M) = \bec 0$ entonces $\frac{1}{c} f (M) = \bec 0$, así, el polinomio $f_M = \frac{1}{c} f$ anula a $M$, tiene grado mínimo y además tiene coeficiente líder $1$.

  Ahora probemos que se cumple la tercera propiedad. Sea $g \in I$, por el algoritmo de la división, tenemos que existen polinomios $q,r \in \F[x]$ tales que $g = qf_M+r$ y $\grad(f) > \grad(r)$. De esta forma, por la proposición \ref{prop:EvalPoly} y recordando que $g(M) = \bec 0$, entonces tenemos que
  \[
    g(M) = q(M) f_M(M) + r(M) = r(M) = \bec 0.
  \]
  Ahora, si $r \neq 0$ entonces eso implica que $r \in I$, pero como $\grad(r) < \grad(f_M)$, eso contradice que $f_M$ tenga grado mínimo. Así tenemos que $r = 0$, pero entonces $f$ divide a todos los elementos de $I$.
  
  Así, si existe otro polinomio $f'$ de mismo grado que $f_M$ en $I$ con coeficiente líder $1$, éste debe ser un múltiplo escalar de $f_M$, pero como ambos tienen el mismo coeficiente líder, entonces la única posibilidad, es que $f_M = f'$. Así tenemos que $f_M$ es el único polinomio de grado mínimo con coeficiente líder $1$ tal que anula a $M$.
\end{proof}

\begin{defi}
  Sea $M \in \M_n(\F)$ llamaremos el \emph{polinomio minimal} de $M$ al único polinomio $f_M$ de grado mínimo con coeficiente líder $1$ tal que $f_M(M) = \bec 0$.
\end{defi}

El polinomio minimal de una matriz tiene varias propiedades interesante a detallar, la primera es que el polinomio minimal es invariante ante la semejanza de matrices. Las demás propiedades se abordaran en la sección \ref{sec:Cayley-Hamilton}.

\begin{prop}\label{prop:InvMinPoly}
  Sea $M,N \in \M_n(\F)$ si $M \sim N$ entonces tienen el mismo polinomio minimal.
\end{prop}
\begin{proof}
  Notemos que si $f_M(M) = \bec 0$ y $f_N(N) = \bec 0$, entonces por la proposición \ref{prop:MPolySem} se debe cumplir que $f_N(M) = \bec 0$ y $f_N(M) = \bec 0$, pero eso implica, por el teorema \ref{teor:UnicidadPMin} que $f_M \mid f_N$ y $f_N \mid f_M$.

  Así, por propiedad de los polinomios, entonces debe existir una constante $c \in \F$ tal que $f_M = c f_N$, pero como $f_M$ y $f_N$ tiene el mismo coeficiente líder $1$, entonces tenemos que $c = 1$ y por tanto $f_M = f_N$.
\end{proof}



\section{Triangularización y diagonalización de matrices complejas.}

Ya con todas estas herramientas, podemos regresar a nuestro objetivo principal, el cual es saber cuando una matriz es diagonalizable o triangularizable. Primer demostremos algunos teoremas para un campo cualquiera, luego veremos algunos resultados para el campo de los complejos.

\begin{teor}
  Sea $M \in \M_n(\F)$, entonces la matriz $M$ es diagonalizable si y solo sí existe alguna base $B$ de $\F^n$ tal que todos los elementos de $B$ son vectores propios de $M$.
\end{teor}
\begin{proof}
  Supongamos que $M$ es diagonalizable, es decir, existen matrices $P,D \in \M_n(\F)$, donde $D$ es diagonal y $P$ es invertible, tal que $M = P^{-1} D P$. Ahora, si $P^{-1} = \bigl( v_1 \mid \cdots \mid v_n\bigr)$ entonces por teoremas conocidos, $B = \{v_1,\ldots, v_n\}$ forma una base de $\F^n$. 

  De igual forma, por propiedades de inversa de una matriz, sabemos que $P v_i = e_i$ para todo $i \in \{1,\ldots,n\}$, de esta forma, notemos que para todo $i \in \{1,\ldots,n\}$ se cumple que
  \[
    Mv_i = (P^{-1}DP)v_i = P^{-1} D e_i = P^{-1} (D_{ii}e_i) = D_{ii} P^{-1}e_i = D_{ii} v_i.
  \]
  De esta forma $D_{ii}$ es un valor propio de $M$ con vector propio $v_i$ para todo $i \in \{1,\ldots,n\}$, así todos los elementos de la base $B$ son vectores propios de $M$.

  Ahora, supongamos que existe una base $B = \{v_1,\ldots,v_n\}$ donde todos sus elementos son vectores propios de $M$. Sea $\lambda_i$ el valor propio asociado al vector propio $v_i$ para $i \in \{1,\ldots,n\}$. Así, definamos las matrices 
  \[ D = \begin{spmatrix}{c|c|c} \lambda_1 e_i & \cdots & \lambda_n e_n \end{spmatrix} 
    \Eqand
     P = \begin{spmatrix}{c|c|c} v_1 & \cdots & v_n \end{spmatrix}.
  \]
  
  Por propiedades de las matrices, tenemos que $P$ es invertible ya que sus columnas forman una base de $\F^n$, de esta forma, dado que $P^{-1}v_i = e_i$ para todo $i\in\{1,\ldots,n\}$ por propiedades de la inversa de una matriz, entonces notemos que
  \begin{align*}
    P^{-1} M P &= P^{-1} M \begin{spmatrix}{c|c|c} v_1 & \cdots & v_n \end{spmatrix} 
       =P^{-1}  \begin{spmatrix}{c|c|c} Mv_1 & \cdots & Mv_n \end{spmatrix} \\
      &=P^{-1}  \begin{spmatrix}{c|c|c} \lambda_1 v_1 & \cdots & \lambda_n v_n \end{spmatrix} 
       =\begin{spmatrix}{c|c|c} P^{-1}(\lambda_1 v_1) & \cdots & P^{-1}(\lambda_n v_n) \end{spmatrix} \\
      &=\begin{spmatrix}{c|c|c} \lambda_1 P^{-1} v_1 & \cdots & \lambda_n P^{-1} v_n \end{spmatrix} 
       =\begin{spmatrix}{c|c|c} \lambda_1 e_1 & \cdots & \lambda_n e_n \end{spmatrix} \\
      &= D.
  \end{align*}
  De esta forma $D \sim M$ y por tanto $M$ es diagonalizable.
\end{proof}

Este teorema, nos da una condición para saber cuando una matriz es diagonalizable. Entonces, lo que tenemos que hacer es, primero, encontrar todos las raíces del polinomio característico y luego obtener las bases de los espacios nulos de $M - \lambda I$ y juntar las bases, más adelante demostraremos que para valores propios distintos los espacios nulos son independientes, por tanto no hay que preocuparse que al juntar las bases estas no sean linealmente independientes. Si al juntar las bases obtenemos un conjunto de $n$ vectores, entonces la matriz es diagonalizable.

Ahora, qué sucede cuando una matriz no es diagonalizable, es decir que no se pueda hacer una base con los vectores  propios de $M$. Antes de responder esta pregunta, demostremos el siguiente lema.

\begin{lema}\label{lema:CleanColumn1}
  Sea $M \in \M_n(\F)$. La matriz $M$ tiene algún valor propio $\lambda$ si y solo sí es semejante a alguna matriz de la forma
  \[
    N = \begin{spmatrix}{c|c|c|c}
      \lambda e_1 & w_2 &  \ldots & w_n
    \end{spmatrix},
  \]
  donde $w_2, \ldots, w_n \in \F^{n-1}$.
\end{lema}
\begin{proof}
  Supongamos que $M$ tiene algún valor propio $\lambda$, de este modo existe vector no nulo $v$ tal que $Mv = \lambda v$. Ahora, por propiedades de los $\F$-espacios vectoriales, sabemos que debe existir vectores $v_2,\ldots,v_n \in \F^n$ tal que $B = \{v,v_2,\ldots,v_n\}$ es una base de $\F^n$.

  De esta forma, consideremos la matríz $P = \bigl( v \mid v_2 \mid \cdots \mid v_n \bigr)$, en primer lugar, notemos que $P$ es invertible ya que sus columnas forman una base y por propiedades de la inversa $P^{-1}v = e_1$. De este modo, sea $N = P^{-1}MP$, notemos que 
  \begin{align*}
    N &= P^{-1} M \begin{spmatrix}{c|c|c|c} v & v_2 &  \ldots & v_n \end{spmatrix} 
       = P^{-1} \begin{spmatrix}{c|c|c|c} Mv & Mv_2 &  \ldots & Mv_n \end{spmatrix} \\
      &= P^{-1} \begin{spmatrix}{c|c|c|c} \lambda v & Mv_2 &  \ldots & Mv_n \end{spmatrix} 
       =  \begin{spmatrix}{c|c|c|c} P^{-1}(\lambda v) & P^{-1}Mv_2 &  \ldots & P^{-1}Mv_n \end{spmatrix} \\
      &=  \begin{spmatrix}{c|c|c|c} \lambda P^{-1}v & P^{-1}Mv_2 &  \ldots & P^{-1}Mv_n \end{spmatrix} 
       =  \begin{spmatrix}{c|c|c|c} \lambda e_1 & P^{-1}Mv_2 &  \ldots & P^{-1}Mv_n \end{spmatrix} \\
      &=  \begin{spmatrix}{c|c|c|c} \lambda e_1 & w_2 &  \ldots & w_n \end{spmatrix}.
  \end{align*}
  Así, $N$ tiene la forma pedida y además $M \sim N$.

  Ahora si $M \sim N$ donde $N$ tiene la forma pedida, notemos que $Ne_1 = \lambda e_1$, eso muestra que $\lambda \in E(N)$, pero por el teorema \ref{teor:SemEspectro} tenemos que $E(M) = E(N)$ y por tanto $\lambda \in E(N)$.
\end{proof}

Respondiendo a la pregunta hecha con anterioridad, cuando una matriz no es diagonalizable, dependiendo del campo será diagonalizable o no. Con el lema anterior podemos ver que para que una matriz de $2\times 2$ sea triangularizable, esta debe tener al menos un vector propio, pero en los reales, existen matrices de $2\times 2$ tal que no tienen vectores propios ya que su polinomio característico no tiene raíces.

Debido a que en algunos campos una una matriz puede tener vectores propios o no, la teoría espectral se desarrolla principalmente sobre campos algebraicamente cerrados. Más específicamente, sobre $\C$. Un ejemplo es el siguiente teorema.

\begin{prop}\label{prop:MComplexHasEV}
  Sea $M \in  \M_n(\C)$, entonces $M$ tiene al menos un valor propio.
\end{prop}
\begin{proof}
  Sea $f \in \C[x]-\{0\}$ el polinomio minimal de $M$, por el teorema fundamental del cálculo, sabemos que debe haber alguna factorización $f = (x-\alpha_1)\cdots(x-\alpha_k)$, luego por la proposición \ref{prop:EvalPoly} tenemos que
  \[ f(M) = (M-\alpha_1 I)\cdots(M-\alpha_k I) = \bec 0.\]

  Ahora, dado que la multiplicación de matrices invertibles es invertible, entonces debe existir al menos un $\alpha_i$ con $i \in \{1,\ldots,k\}$ tal que $M-\alpha_i I$ no es invertible, ya que en caso contrario implicaría que $\bec 0$ es invertible. Pero por el teorema \ref{teor:PropVP}, esto implica que $\alpha_i$ es un valor propio de $M$.
\end{proof}

\begin{teor}\label{teor:ComplexTriang}
  Sea $M \in  \M_n(\C)$ entonces $M$ es triangularizable.
\end{teor}
\begin{proof}
  Probemos la proposición por inducción sobre $n$. Es claro que si $n = 1$ entonces $M$ ya es triangular superior, así supongamos que se cumple para algún natural $k$.

  Sea $M \in \M_{k+1}(\C)$, por la proposición \ref{prop:MComplexHasEV} existe $\lambda \in E(M)$, así, por el lema \ref{lema:CleanColumn1} entonces debe existir una matriz $Q \in \M_k(\C)$ y vector $v \in \F^k$ tal que $M \sim N$ donde 
  \[ N = \begin{spmatrix}{c|c}
      \lambda & v^t \\\hline
      \bec 0 & A
  \end{spmatrix} \]

  Pero por hipótesis de inducción existen matrices $U, Q \in \M_k(\C)$ donde $U$ es triangular superior y $Q$ es invertible tales que $A = Q^{-1}UQ$. Ahora, definamos las siguiente matríz $P$
  \[
    P = \begin{spmatrix}{c|c}
      1 & \bec 0 \\\hline
      \bec 0 & Q
  \end{spmatrix},
  \]
  Ahora, notemos que $P$ es una matriz de $(k+1)\times(k+1)$ invertible, ya que por multiplicación por bloques, se tiene que
  \[
    P \begin{spmatrix}{c|c}
      1 & \bec 0 \\\hline
      \bec 0 & Q^{-1}
  \end{spmatrix} = \begin{spmatrix}{c|c}
      1 & \bec 0 \\\hline
      \bec 0 & Q
  \end{spmatrix} \begin{spmatrix}{c|c}
    1 & \bec 0 \\\hline
    \bec 0 & Q^{-1}
\end{spmatrix} = \begin{spmatrix}{c|c}
  1 & \bec 0 \\\hline
  \bec 0 & I_k
\end{spmatrix} = I_{k+1}
      \implies
    P^{-1} = \begin{spmatrix}{c|c}
      1 & \bec 0 \\\hline
      \bec 0 & Q^{1}
  \end{spmatrix}.
  \]

  De esta forma, notemos que si definimos $T = P^{-1}NP$, entonces $M \sim T$, por transitividad de la semejanza y además se cumple que
  \begin{align*}
    T &= P^{-1}NP
       = \begin{spmatrix}{c|c}1 & \bec 0 \\\hline  \bec 0 & Q^{-1} \end{spmatrix}
         \begin{spmatrix}{c|c} \lambda & v^t \\\hline  \bec 0 & A \end{spmatrix}
         \begin{spmatrix}{c|c}1 & \bec 0 \\\hline  \bec 0 & Q \end{spmatrix} \\
      &= \begin{spmatrix}{c|c}1 & \bec 0 \\\hline  \bec 0 & Q^{-1} \end{spmatrix}
         \begin{spmatrix}{c|c}\lambda & v^tQ \\\hline  \bec 0 & AQ \end{spmatrix} 
       = \begin{spmatrix}{c|c}\lambda & v^tQ \\\hline  \bec 0 & Q^{-1}AQ \end{spmatrix} \\
      &= \begin{spmatrix}{c|c}\lambda & v^tQ \\\hline  \bec 0 & U \end{spmatrix}.
  \end{align*}
  Notemos que por la forma, $T$ es una matriz triangular superior. De esta forma tenemos que $M$ es triangularizable.

  De esta forma, por inducción, tenemos que para cualquier matriz cuadrada con entradas en los complejos es triangularizable.
\end{proof}



\subsection{Polinomio minimal de matrices complejas.}

Para concluir con este capítulo, mostraremos la forma del polinomio minimal para matrices compleja. Para ello necesitaremos recordar una propiedad importante de las matrices.

\begin{defi}
  Sea $M \in \M_n(\F)$, decimos que $M$ es \emph{nilpotente} si existe algún $d\in\N$ tal que $M^d = \bec 0$. Si $M$ es nilpotente, al mínimo $d \in \N$ tal que $M^d = \bec 0$ se le denomina \emph{grado del nilpotencia} de $M$.
\end{defi}

En primer lugar, notemos que si tenemos una matriz $T$ triangular no estricta, entonces $T$ no será nilpotente. Esto se puede comprobar fácilmente, para ello hay que recordar que si tenemos dos matrices triangulares superiores (inferiores) $A$ y $B$, entonces
  \[
    (AB)_{ii} = A_{ii} B_{ii}
  \]
para todo $i \in \{1,\ldots,n\}$. De este modo, si $T$ no es estricta, entonces para algún $j \in \{1,\ldots,n\}$ se debe cumplir que $T_{jj} \neq 0$ y por tanto $(T^k)_{ii} = (T_{ii})^k \neq 0$ para todo $k \in \N$.

De este modo, para que una matriz triangular sea nilpotente, tenemos que debe ser triangular estricta. La vuelta también es cierta y además podemos obtener una cota superior para su grado de nilpotencia.

\begin{lema}\label{lema:TriangNilp}
  Sea $T \in M_n(\F)$ una matriz triangular, entonces $T$ es nilpotente si y solo sí $T$ es triangular estricta. Además, si $T$ es nilpotente, entonces su grado de nilpotencia es menor o igual a $n$.
\end{lema}
\begin{proof}
  Por lo ya visto, sabemos que si $T$ es nilpotente entonces debe ser triangular estricta. De esta forma, solo falta ver que todas las matrices $T$ triangulares estrictas cumplen que $T^n = \bec 0$.

  Pero antes de ello, primero veamos que para cualesquiera matrices $A,B \in M_n(\F)$ y vectores $v,w \in \F^n$, entonces
  \begin{equation}\label{eq:auxProdBloq}
    \begin{spmatrix}{c|c}
      0 & v^t \\\hline
      \bec 0 & A
    \end{spmatrix}
    \begin{spmatrix}{c|c}
      0 & w^t \\\hline
      \bec 0 & B
    \end{spmatrix}
      = 
    \begin{spmatrix}{c|c}
      0 & v^tB \\\hline
      \bec 0 & AB
    \end{spmatrix}.
  \end{equation}

  Ahora, primero consideremos el caso donde $T$ es una matriz triangular superior estricta. Por inducción sobre $n$, notemos que si $n = 1$, entonces $T = 0$, por lo que es claro que $T^n = 0$. Así, supongamos que la propiedad se cumple para algún $k \in \N$.

  Sea $T \in \M_{k+1}(\F)$ una matriz triangular superior estricta, notemos que entonces $T$ tiene la siguiente forma
  \[
    T = \begin{spmatrix}{c|c}
      0 & v^t \\\hline
      \bec 0 & U
    \end{spmatrix},
  \]
  para alguna matriz triangular superior estricta $U \in \M_k(\F)$ y vector $v \in \F^k$. Ahora, por la ecuación \ref{eq:auxProdBloq} tenemos que
  \begin{align*}
    T^2 &= \begin{spmatrix}{c|c} 0 & v^t \\\hline \bec 0 & U \end{spmatrix} 
           \begin{spmatrix}{c|c} 0 & v^t \\\hline \bec 0 & U \end{spmatrix}
        = \begin{spmatrix}{c|c} 0 & v^tU \\\hline \bec 0 & U^2 \end{spmatrix} \\
    T^3 &= \begin{spmatrix}{c|c} 0 & v^tU \\\hline \bec 0 & U^2 \end{spmatrix} 
        \begin{spmatrix}{c|c} 0 & v^t \\\hline \bec 0 & U \end{spmatrix}
     = \begin{spmatrix}{c|c} 0 & v^tU^2 \\\hline \bec 0 & U^3 \end{spmatrix} \\
    &\vdotswithin{=} \\
    T^{k+1} &= \begin{spmatrix}{c|c} 0 & v^tU^{k-1} \\\hline \bec 0 & U^k \end{spmatrix} 
     \begin{spmatrix}{c|c} 0 & v^t \\\hline \bec 0 & U \end{spmatrix}
  = \begin{spmatrix}{c|c} 0 & v^tU^k \\\hline \bec 0 & U^{k+1} \end{spmatrix}.
  \end{align*}
  Ahora, por hipótesis de inducción tenemos que $U^k = U^{k+1} = \bec 0$ y de este modo $T^{k+1} = \bec 0$.

  Así, cualquier matriz $T \in M_n(\F)$  triangular superior estricta es nilpotente, y su grado de nilpotencia es menor o iguala a $n$.

  Ahora, si $T \in M_n(\F)$ es triangular inferior estricta, entonces $T^t$ es triangular superior estricta y por tanto $(T^t)^n = (T^n)^t = \bec 0$, lo que implica que $T^n =  \bec 0$. Lo que finalmente nos permite concluir la proposición.
\end{proof}

Para continuar, supongamos que tenemos una matriz triangular $T\in\M_n(\F)$. Por propiedades de la determinante sabemos que si $T = (t_{ij})$, entonces $\det(T) = t_{11} \cdots t_{nn}$.

De esta forma, si consideramos la matriz $T-t_{ii}I$ para alguna $i \in \{1,\ldots,n\}$, entonces, por propiedades de las matrices triangulares, tenemos que $T-t_{ii}I$ es triangular y además
\[
  \det(T-t_{ii}I) = (t_{11}-t_{ii})\cdots(t_{ii}-t_{ii})\cdots(t_{nn}-t_{ii}) = 0.
\]
Y por el teorema \ref{teor:PropVP}, eso quiere decir que $t_{ii} \in E(T)$. Ahora, de manera análoga, si $\lambda \in E(T)$ entonces 
  \[
    \det(T-t_{ii}I) = (t_{11}-\lambda)\cdots(t_{nn}-\lambda) = 0.
  \]
pero por propiedades de los campos, eso quiere decir que $\lambda = t_{ii}$ para algún $i \in \{1,\ldots,n\}$. Con esto, hemos mostrado la siguiente proposición

\begin{prop}\label{prop:EpecTriang}
  Sea $T \in \M_n(\F)$ una matriz triangular, entonces $E(T) = \{T_{ii} : i = 1,\ldots,n\}$. \qed
\end{prop}

Con todas estas propiedades ya demostradas, podemos demostrar el siguiente teorema.

\begin{teor}\label{teor:MinPolyTriang}
  Sea $T\in\M_n(\F)$ una matriz triangular, si $E(T) = \{\lambda_1,\ldots,\lambda_k\}$, entonces
    \[
      f_T = \prod_{i=1}^k (x-\lambda_i)^{n_i},
    \]
  donde $1 \leq n_i \leq n$ para cada $i \in \{1,\ldots,k\}$.
\end{teor}
\begin{proof}
  Sea $U = (T-\lambda_1 I)\cdots(T-\lambda_n I)$ y $T = (t_{ij})$. Primero, notemos que, por la proposición \ref{prop:EpecTriang}, para cada $i \in \{1,\ldots,n\}$ existe $j\in\{1,\ldots,k\}$ tal que $t_{ii} = \lambda_j$. De esta forma por propiedades de las matrices triangulares, se cumple que $U$ es triangular y además
  \begin{align*}
    U_{ii} &= (T-\lambda_1 I)_{ii}\cdots(T-\lambda_j I)_{ii}\cdots(T-\lambda_n I)_{ii}  \\
      &= (t_{ii}-\lambda_1)\cdots(t_{ii}-\lambda_j)\cdots(t_{ii}-\lambda_n)  \\
      &= 0.
  \end{align*}
  Así, tenemos que $U$ es una matriz triangular estricta y por el lema \ref{lema:TriangNilp} tenemos que $U^n = \bec 0$.

  Ahora, sea $f = (x-\lambda_1)^n\ldots(x-\lambda_k)^n$, notemos que $f(T) = U^n = \bec 0$, así, por el teorema \ref{teor:UnicidadPMin}, tenemos que $f_T \mid f$.
  
  Como $f$ se puede factorizar como un producto de polinomios de grado $1$, entonces, dado que $f_T$ divide a $f$, $f_T$ tiene coeficiente líder $1$ y por teoremas de divisibilidad de polinomios, se debe cumplir que
    \[
      f_T = \prod_{i=1}^k (x-\lambda_i)^{n_i},
    \]
donde $0 \leq n_i \leq n$ para cada $i \in \{1,\ldots,k\}$. Así supongamos que $n_r = 0$ para algún $r \in \{1,\ldots,k\}$.

Nuevamente por la proposición \ref{prop:EpecTriang} debe existir $s \in \{1,\ldots,n\}$ tal que $\lambda_r = t_{ss}$. Ahora, por propiedades de las matrices triangulares, se debe cumplir que
  \[
    [f_T(T)]_{ss} 
      = \prod_{\substack{i=1\\i\neq r}}^k (T-\lambda_i I)_{ss}^{n_i} 
      = \prod_{\substack{i=1\\i\neq r}}^k (t_{ss}-\lambda_i)^{n_i}.
  \]
Notemos que $t_{ss} \neq \lambda_i$ para todo $i \in \{1,\ldots,k\}-\{r\}$, entonces $[f_T(T)]_{ss} \neq 0$, pero eso contradice que $f_T(T) = \bec 0$. De este modo $1 \leq n_i \leq n$ para cada $i \in \{1,\ldots,k\}$.
\end{proof}

Ahora, como toda matriz compleja es triangularizable y el polinomio minimal es invariante ante la semejanza, el teorema anterior nos permite describir su polinomio minimal.

\begin{coro}
  Sea $M = \M_n(\C)$ y $E(M) = \{\lambda_1,\ldots,\lambda_k\}$, entonces $ f_M = \prod_{i=1}^k (x-\lambda_i)^{n_i}$, donde $1 \leq n_i \leq n$ para cada $i \in \{1,\ldots,k\}$.
\end{coro}
\begin{proof}
  Por el teorema \ref{teor:ComplexTriang}, sabemos que debe existir una matriz triangular $T \in \M_n(\C)$ tal que $M \sim T$. Ahora, por el teorema \ref{teor:SemEspectro} sabemos que $E(M) = E(T)$, así, por el teorema \ref{teor:MinPolyTriang} y la proposición \ref{prop:InvMinPoly}, finalmente obtenemos que
  \[
    f_M = f_T = \prod_{i=1}^k (x-\lambda_i)^{n_i},
  \]
donde $1 \leq n_i \leq n$ para cada $i \in \{1,\ldots,k\}$.
\end{proof}

 

\section{Proyecto: La fórma canónica de Jordan}

Para finalizar con la capítulo se presentará uno de los resultados más importantes de la teoría espectral. Para ello retomemos lo visto en la sección \ref{sec:TyDdeMat2x2}. Como se vio en esa sección, para el conjunto de matrices complejas de $2\times 2$ solo existen dos clases de equivalencias, las cuales son
\[
  D_{\alpha, \beta} = \begin{pmatrix}
    \alpha & 0 \\ 0 & \beta
  \end{pmatrix}
    \Eqand
  T_\gamma \begin{pmatrix}
      \gamma & 1 \\ 0 & \gamma
    \end{pmatrix},
\]
con $\alpha, \beta, \gamma \in \C$. Con lo visto en el capítulo, es fácil ver que los valores de $\alpha$, $\beta$ y $\gamma$ corresponden a los valores propios de la matriz. Además, por los teoremas vistos, se puede ver que si una matriz es semejante a $D_{\alpha, \beta}$ es por que sus valores propios son distintos o el espacio propio es de dimensión $2$.

Ahora, lo que realmente nos interesa, es cuando una matriz es semejante a $T_\gamma$. Notemos que en este caso la matriz tiene un único valor propio y la dimensión de su espacio propio es 1. En este caso la matriz no es triangularizable, pero tiene una forma muy particular, la forma de la matriz $T_\gamma$ no es pura casualidad, esta matriz es conocida como bloque de Jordan de tamaño 2 con valor propio $\gamma$. Los bloques de Jordan son un tipo muy especial de matrices y serán fundamentales para esta sección.

\begin{defi}
  Se denomina \emph{bloque de Jordan} de tamaño $n$ con valor propio $\lambda \in \F$ a la matriz de $n\times n$ dada por
    \[
      J_{n,\lambda} = \begin{pmatrix}
        \lambda & 1       & 0       & \cdots & 0       & 0       & 0       \\
        0       & \lambda & 1       & \cdots & 0       & 0       & 0       \\
        0       & 0       & \lambda & \cdots & 0       & 0       & 0       \\
        \vdots  & \vdots  & \vdots  & \ddots & \vdots  & \vdots  & \vdots  \\
        0       & 0       & 0       & \cdots & \lambda & 1       & 0       \\
        0       & 0       & 0       & \cdots & 0       & \lambda & 1       \\
        0       & 0       & 0       & \cdots & 0       & 0       & \lambda
      \end{pmatrix}.
    \]
\end{defi}

Hay varias cosas que notar, la primera es que si $n=1$ entonces toda matriz de $1\times 1$ se considera un bloque de Jordan. La segunda propiedad importante, es que el bloque de Jordan $J_{n,\lambda}$ tiene como único valor propio a $\lambda$. La última propiedad interesante es que $(J_{n,\lambda}-\lambda I )^n = \bec 0$, por el teorema de Cayley-Hamilton. Esta última propiedad va más allá, resulta que para todo bloque de Jordan $J$ de tamaño $n$, su polinomio minimal es $f_J = (x-\lambda)^n$.

Ahora, como vimos en la sección \ref{sec:TyDdeMat2x2}, toda matriz de $2\times 2$ es semejante a una matriz compuesta de solo bloques de Jordan. En el caso de $D_{\alpha, \beta}$ son los bloques $J_{1,\alpha} = \alpha$ y $J_{1,\beta} = \beta$ y en el caso de $T_\gamma$ es claro que $J_{2,\gamma} = T_\gamma$. A esta matriz compuesta de bloques de Jordan es la que llamaremos su forma canónica de Jordan.

\begin{defi}
  Sea $M \in \M_n(\F)$ decimos que la matriz $J \in \M_n(\F)$ es su \emph{forma canónica de Jordan} si $M \sim J$ y $J$ está compuesta solo de bloques de Jordan, es decir, es de la forma
    \[
      J = \begin{spmatrix}{c|c|c|c}
        J_{n_1,\lambda_1} & \bec 0 & \cdots & \bec 0  \\ \hline
        \bec 0 & J_{n_2,\lambda_2} & \cdots & \bec 0  \\ \hline
        \vdots & \vdots & \ddots & \vdots  \\ \hline
        \bec 0 & \bec 0 & \cdots  & J_{n_k,\lambda_k} 
      \end{spmatrix}.
    \]
\end{defi}

Con toda esta introducción ya podemos formular el que podría ser uno de los teoremas más importantes de la teoría espectral.

\begin{teor}
  Para toda matríz $M \in \M_n(\C)$ existe su forma canónica de Jordan, además esta su forma canónica de Jordan es única salvo el orden de los bloques. Por último, dos matrices $M,N \in \M_n(\C)$ son semejantes si y solo si comparten la misma forma canónica de Jordan.
\end{teor}

Este teorema implica dos propiedades muy importantes. La primera es que por fin tenemos un ``representante'' de las clases de equivalencia, por lo que podemos determinar de manera simple cuando dos matrices son equivalentes. La segunda es que nos da una expresión muy sencilla para trabajar con las matrices, ya que la fórma canónica de Jordan es casi diagonal, ya que solo tiene elementos distintos de cero en la diagonal principal y tiene unos en la diagonal por encima de ésta.

El único inconveniente, es que la demostración de este teorema es bastante compleja. En esta sección no se demostrará formalmente el teorema, pero sí indagaremos en las propiedades e ideas más importantes para entender el teorema así como el proceso de encontrar la forma canónica de Jordan de una matriz, aunque sin entrar en mucho detalle de las demostraciones.



\subsection{Los espacios invariantes}

El primer tema que hay que entender son los espacios invariantes a una matriz. Para entender la idea principal consideremos el siguiente ejemplo.

Sea $M \in \M_n(\F)$ y $\lambda \in E(M)$, pensemos en el subespacio $E_\lambda$, veamos que para cualquier $v \in E_\lambda$ se cumple que
\[
  Mv = \lambda v \implies M(Mv) = \lambda(Mv),
\]
pero esto implica que $Mv \in E_\lambda$. Esto en pocas palabras quiere decir que si $M[E_\lambda] = \{Mv : v \in E_\lambda\}$ entonces $M[E_\lambda] \subseteq E_\lambda$. Esta propiedad se le conoce como invarianza bajo la matriz $M$ y nos será importante para esta parte.

\begin{defi}
  Sea $M \in \M_n(\F)$ y $W$ un subespacio de $\F^n$. Si definimos $M[W] = \{Mv : v \in W\}$, entonces decimos que $W$ es \emph{invariante bajo $M$} o \emph{$M$-invariante} si $M[W] \subseteq W$.
\end{defi}

¿Por que nos interesan los espacios invariantes? Para contestar esto, recordemos que una matriz $M$ es diagonalizable si y solo si $ E_{\lambda_1} \oplus \cdots \oplus E_{\lambda_k} = \F^n$, donde $E(M) = \{\lambda_1, \cdots, \lambda_k\}$, notemos que esto implica que $\F^n$ se puede descomponer como suma directa de espacios invariantes.

Ahora supongamos que tenemos el otro caso, es decir que se tienen $W_1, \cdots, W_k$ subespacios invariantes bajo $M$ tal que $\F^n = W_1 \oplus \cdots \oplus W_k$. Resulta que al realizar el cambio de base, la matriz resultante está compuesta por bloques, de manera muy semejante a la forma canónica de Jordan. Esto se puede ver con el siguiente teorema.

\begin{teor}\label{teor:InvandBloq}
  Sea $M \in \M_n(\F)$ y sean $W_1, \ldots, W_k$ subespacios $M$-invariantes de $\F^n$ tal que $W_1 \oplus \cdots \oplus W_k = \F^n$, entonces $M \sim A$ donde $A$ está compuesta por bloques, es decir, $A$ es de la forma
    \[
      A = \begin{spmatrix}{c|c|c|c}
        A_1 & \bec 0 & \cdots & \bec 0  \\ \hline
        \bec 0 & A_2 & \cdots & \bec 0  \\ \hline
        \vdots & \vdots & \ddots & \vdots  \\ \hline
        \bec 0 & \bec 0 & \cdots  & A_k 
      \end{spmatrix}.
    \]
\end{teor}
\begin{proof}
  Sea $B_i = \{v_{i1}, \ldots, v_{im_i}\}$ una base de $W_i$ para cada $i \in \{1,\ldots,k\}$, consideremos la matriz $P$ dada por
    \[ P = \begin{spmatrix}{c|c|c|c}
      P_1 & P_2 & \cdots & P_k
    \end{spmatrix},
    \]
  donde $P_i = \bigl( v_{i1} \mid \cdots \mid v_{im_i} \bigr)$. Dado que $\F^n = W_1 \oplus \cdots \oplus W_k$, entonces las columnas de $P$ conforman una base de $\F^n$, eso implica que $P$ es invertible. Así definamos $A = P^{-1}MP$, es claro que $M \sim A$, por lo que solo falta demostrar que $A$ tiene la forma pedida.

  En primer lugar, por definición, tenemos que $A = P^{-1} M \bigl( P_1 \mid \cdots \mid P_k \bigr)$, entonces, por multiplicación en bloques, tenemos que
    \[
      A = \begin{spmatrix}{c|c|c|c}
        P^{-1} M P_1 & P^{-1} M P_2 & \cdots & P^{-1} M P_k
      \end{spmatrix}.
    \]
  De esta forma, estudiemos los bloques $P^{-1} M P_i $ con $i \in \{1,\ldots, k\}$.

  Primero, para todo $i \in \{1,\ldots, k\}$, notemos que si $v \in B_i$, dado que $W_i$ es invariante bajo $M$, entonces $Mv \in W_i = \inner{B_i}$, esto quiere decir que existen $c_1,\ldots,c_{m_i} \in \F$ tales que
    \[ Mv = c_1 v_{i1} + \cdots + c_{m_i} v_{im_i}. \]
  Ahora, notemos que por propiedades de la inversa y construcción, para todo $j\in\{1,\ldots,m_i\}$ se cumple que
  \[
    P^{-1}v_{ij} = e_{s_i + j -1 },
  \]
  donde $s_i$ es el índice de columna del vector $v_{i1}$ en la matriz $P$. Así, dado que $Mv = c_1 v_{i1} + \cdots + c_{m_i} v_{im_i}$ y en la matriz $P$, entonces tenemos que
    \begin{align*}
      P^{-1} M v &= c_1 P^{-1} v_{i1} + \cdots + c_{m_i} P^{-1} v_{im_i} \\
        &= c_1 e_{s_i} + \cdots + c_{m_i} e_{s_i + m_i-1}. 
    \end{align*}
  Resumiendo todo lo dicho, si $v \in B_i$ entonces el vector $P^{-1}Mv$ tiene ceros excepto en las filas correspondientes a las posiciones de columna de $v_{i1},\ldots,v_{im_i}$ en la matriz $P$, es decir
    \[
      P^{-1}Mv = \begin{spmatrix}{c}
        \bec 0_{m_1} \\ \hline
        \vdots \\\hline
        \bec 0_{m_{i-1}} \\\hline
        c_1 \\
        \vdots \\
        c_{m_i} \\ \hline
        \bec 0_{m_{i+1}} \\\hline
        \vdots \\\hline
        \bec 0_{m_k} 
      \end{spmatrix}.
    \]
  
  Así, consideremos el bloque $P^{-1} M P_i $ con $i \in \{1,\ldots,k\}$, por multiplicación por bloques y recordando que $P_i = \bigl( v_{i1} \mid \cdots \mid v_{im_i} \bigr)$, entonces tenemos que
  \[
    P^{-1}MP_i = \begin{spmatrix}{c|c|c}
      P^{-1}Mv_{i1} & \cdots & P^{-1}Mv_{im_i}
    \end{spmatrix}.
  \]
  Notemos que por lo demostrado, para todo $j \in \{1,\ldots,m_i\}$ se tiene que el vector $P^{-1}Mv_{ij}$ tiene ceros en todas sus entradas, excepto en las filas correspondientes a las posiciones de columna de $v_{i1},\ldots,v_{im_i}$ en la matriz $P$, entonces se cumple que existe una matriz $A_i \in \M_{m_i}(\F)$ tal que
  \[
    P^{-1}MP_i = \begin{spmatrix}{c}
      \bec 0_{m_1} \\ \hline
      \vdots \\\hline
      \bec 0_{m_{i-1}} \\\hline
      A_i \\ \hline
      \bec 0_{m_{i+1}} \\\hline
      \vdots \\\hline
      \bec 0_{m_k} 
    \end{spmatrix}.
  \]
  De esta forma, podemos concluir que $A$ está compuesta por bloques, ya que tiene la forma
  \[
    A = \begin{spmatrix}{c|c|c|c}
      A_1 & \bec 0 & \cdots & \bec 0  \\ \hline
      \bec 0 & A_2 & \cdots & \bec 0  \\ \hline
      \vdots & \vdots & \ddots & \vdots  \\ \hline
      \bec 0 & \bec 0 & \cdots  & A_k 
    \end{spmatrix}. \qedhere
  \]
\end{proof}

Notemos que el poder encontrar espacios invariantes independientes, es indispensable para nuestro objetivo. Si queremos llevar una matriz a su forma de Jordan, tenemos que encontrar los espacios invariantes necesarios.



\subsection{Vectores propios generalizados} \label{subsec:VecPropGen}

Ya vimos que para llevar una matriz a su forma canónica de Jordan debemos encontrar espacios invariantes que nos lo permitan. Así pensemos un poco por adelantado, supongamos que tenemos una matriz $M \in \M_5(\C)$ cuya fórma canónica de Jordan $J$ está dada por
\[
  J = \begin{spmatrix}{ccc|cc}
    \lambda & 1 & 0 & 0 & 0 \\
    0 & \lambda & 1 & 0 & 0 \\
    0 & 0 & \lambda & 0 & 0 \\\hline
    0 & 0 & 0 & \mu & 1 \\
    0 & 0 & 0 & 0 & \mu 
  \end{spmatrix}.
\]

Una pregunta que nos podríamos hacer es: ¿cuál es la base que me manda $M$ a $J$? En otras palabras, qué tiene que cumplir $P$ para que $P^{-1}MP = J$. Si $P = \bigl( v_1 \mid v_2 \mid v_3 \mid v_4 \mid v_5 \bigr)$ entonces, por multiplicación en bloques, veamos que
\begin{align*}
  MP &= PJ,  \\
  M\begin{spmatrix}{c|c|c|c|c} v_1 & v_2 & v_3 & v_4 & v_5 \end{spmatrix}
    &= P\begin{spmatrix}{c|c|c|c|c}
        \lambda & 1 & 0 & 0 & 0 \\
        0 & \lambda & 1 & 0 & 0 \\
        0 & 0 & \lambda & 0 & 0 \\
        0 & 0 & 0 & \mu & 1 \\
        0 & 0 & 0 & 0 & \mu 
      \end{spmatrix} , \\
  \begin{spmatrix}{c|c|c|c|c} Mv_1 & Mv_2 & Mv_3 & Mv_4 & Mv_5 \end{spmatrix}
    &= \begin{spmatrix}{c|c|c|c|c} \lambda v_1 & \lambda v_2 + v_1 & \lambda v_3 + v_2 & \mu v_4 & \mu v_5 + v_4 \end{spmatrix}. \tagthis \label{eq:ExampleJordan}
\end{align*}

Ahora, analicemos cuidadosamente cada una de estas propiedades. En primer lugar, si analizamos la demostración del teorema \ref{teor:InvandBloq} podemos deducir que $B_1 = \{v_1, v_2, v_3\}$ y $B_2 = \{v_3, v_4\}$ son las bases de dos espacios vectoriales invariantes independientes. Así analicemos el comportamiento de cada uno de estas bases para deducir el espacio invariante del que vienen.

Comencemos con $B_1$. Si revisamos la ecuación \eqref{eq:ExampleJordan} podemos ver $v_1$ es vector propio, por propiedades conocidas, eso implica que $v_1 \in E_\lambda = \ker(M-\lambda I)$. Ahora, nuevamente revisando la ecuación \eqref{eq:ExampleJordan}, notemos que $v_2$ cumple que
  \begin{align*}
    Mv_2 &= \lambda v_2 + v_1, \\
    (M_2 - \lambda I)v_2 &= v_1, \\
    (M_2 - \lambda I)^2 v_2 &= (M_2 - \lambda I)v_1 \\
      &= \bec 0,
  \end{align*}  
de esta forma $v_2 \in \ker(M-\lambda I)^2$. Por último, repitiendo el mismo proceso, podemos ver que $v_3 \in \ker(M-\lambda I)^3$ y además
  \[
    (M-\lambda I)^2v_3 = (M-\lambda I)v_2 = v_1.
  \]
Análogamente con $B_2$, es posible comprobar que $v_4 \in \ker(M-\mu I)$, $v_4 \in \ker(M-\mu I)^2$ y además se cumple la identidad
\[
  (M-\mu I) v_4 = v_5.
\]

Resumamos todo lo que conocemos hasta ahora, en primer lugar, si los vectores $\{v_1,\ldots,v_k\}$ están asociados a un bloque de Jordan de tamaño $k$ y valor $\lambda$, estos deben cumplir dos propiedades. La primera es que $v_i \in \ker(M-\lambda I)^i$ para todo $i \in \{1,\ldots,k\}$ y la segunda es que cumplen la siguiente identidad
\[
  (M-\lambda I)^{k-1}v_k = (M-\lambda I)^{k-2}v_{k-1} = \cdots = (M-\lambda)v_2 = v_1.
\]

La primera propiedad es la que nos va a indicar cuáles son los espacios invariantes que nos permitirán llevar una matriz a su forma canónica de Jordan y es la que estudiaremos en lo que resta de esta parte. La segunda propiedad es la que nos indicará la forma de los bloques de Jordan, esa lo estudiaremos en la siguiente parte.

\begin{defi}
  Sean $M \in \M_n(\F)$ y $\lambda \in \F$, un vector $v \in \F^n$ es llamado un \emph{vector propio generalizado} de $M$ correspondiente a $\lambda$ si $(M-\lambda I)^m v = \bec 0$ para algún entero positivo $m$.
\end{defi}

Notemos que por lo revisado con anterioridad, todos los vectores asociados a un bloque de Jordan son vectores propios generalizados. Además, si $m$ es el menor número tal que $(M-\lambda I)^mv = \bec 0$, entonces $(M-\lambda I)^{m-1}v$ es un vector propio y por tanto $\lambda$ es un valor propio. Así todo vector propio generalizado está asociado a un valor propio.

De igual forma, es natural pensar que los subespacios invariantes que buscamos estén relacionados con los vectores propios generalizados. Así, pensemos en dos vectores propios generalizados $v$ y $w$ de $M$ asociados al valor propio $\lambda$, notemos que deben existir $r, s \in \N$ tal que $(M-\lambda I)^r v = \bec 0$ y $(M-\lambda I)^s w = \bec 0$, así notemos que para todo $c\in\F$ se cumple que
\[
  (M-\lambda I)^{r+s}(v+cw) = (M-\lambda I)^s(M-\lambda I)^r v + c (M-\lambda I)^r (M-\lambda I)^sv = \bec 0.
\]
Esto implica que el conjunto de todos los vectores propios generalizados forman un subespacio vectorial, así podemos justificar la siguiente definición.

\begin{defi}
  Sea $M \in \M_n(\F)$ y $\lambda \in E(M)$. El \emph{espacio propio generalizado} de  $T$ correspondiente a $\lambda$ es definido como $K_\lambda = \{v \in \F^n : (M-\lambda I)^m v = \bec 0, m \in \N\}$.
\end{defi}

Ahora, veamos algunas de las propiedades del espacio propio generalizado. En primer lugar, notemos que $v \in \ker(I-\lambda)^m$ para algún $m \in \N$ si y solo si $v \in K_\lambda$, esto es claro por la definición. La segunda propiedad importante, es que $K_\lambda$ es un conjunto $T$-invariante, para demostrar esto pensemos en algún $v \in K_\lambda$, sabemos que existe $m\in\N$ tal que $(M-\lambda I)^mv = \bec 0$, así, por la proposición \ref{prop:EvalPoly}, veamos que
\[ (M-\lambda I)^m (Mv) = M(M-\lambda I)^m v = M\bec 0 = 0,\]
así es claro que $M[K_\lambda] \subseteq K_\lambda$. De este modo, hemos demostrado el siguiente teorema.

\begin{teor}
  Sea $M \in \M_n(\F)$ y $\lambda \in E(M)$ entonces $K_\lambda$ es un subespacio $M$-invariante que contiene a $E_\lambda$. \qed
\end{teor}

Ahora, para la última propiedad importante, se puede probar de manera simple que para toda $m \in \N$ se cumple que
\[
  \{0\} \subseteq \ker(M-\lambda I) \subseteq \ker(M-\lambda I)^2 \subseteq \cdots \subseteq \ker(M-\lambda I)^m \subseteq K_\lambda,
\]
dado que la dimensión no puede crecer indefinidamente, debe existir algún $m_\lambda$ tal que para todo $m \geq m_\lambda$ se cumple que $\ker(M-\lambda I)^{m_\lambda} = \ker(M-\lambda I)^m$, entonces esto implica que existe $m_\lambda \in \N$ tal que
\[ K_\lambda = \ker(M-\lambda I)^{m_\lambda}.\]
Este número $m_\lambda$ no siempre se puede encontrar para cualquier campo, pero en los complejos este coincide con la multiplicidad algebraica. Todas estas propiedades se puede resumir con el siguiente teorema.

\begin{teor}
  Sea $M \in \M_n(\C)$ y $\lambda \in E(M)$, si $m$ es la multiplicidad algebraica de $\lambda$ entonces
\[  \dim(K_{\lambda}) = m \Eqand K_\lambda = \ker(T-\lambda I)^m. \]
\end{teor}

Este último teorema nos da una forma sencilla de calcular $K_\lambda$ y como ya uno puede estar suponiendo, los espacios propios generalizados son los subespacios invariantes que nos permitirán encontrar la forma canónica de Jordan.

\begin{teor}
  Sea $M \in \M_n(\C)$ y $E(M) = \{\lambda_1, \cdots, \lambda_k\}$, entonces $K_{\lambda_1} \oplus \cdots \oplus K_{\lambda_k} = \F^n$.
\end{teor}

Aunque ya tengamos los espacios invariantes independientes, esto aún no nos permite tener la forma canónica de Jordan, simplemente nos permite encontrar una matriz semejante a $M$ compuesta por bloques. Como observación final, cada uno de estos bloques estará asociado a un solo valor propio, esto nos permitirá simplificar la búsqueda de la forma canónica de Jordan, ya que solo tenemos que concentrarnos en las matrices con un único valor propio.

\begin{example}
  Consideremos la matriz $A$ de $4\times 4$ con entradas complejas dada por
    \[
      A = \begin{pmatrix}
        2 & -1 & 0 & 1 \\
        0 & 3 & -1 & 0 \\
        0 & 1 & 1 & 0 \\
        0 & -1 & 0 & 3
      \end{pmatrix}.
    \]
  Encontremos una matriz semejante a $A$ compuesta por bloques usando los espacios propios generalizados.

  \examplesolution

  En primer lugar calculemos los valores propios de la matriz, así como sus multiplicidades algebraicas, usando el polinomio característico. Veamos que
  \[
    \det(A-xI) = \det\begin{pmatrix}
      2-x & -1 & 0 & 1 \\
      0 & 3-x & -1 & 0 \\
      0 & 1 & 1-x & 0 \\
      0 & -1 & 0 & 3-x
    \end{pmatrix}
      = (x-2)^3 (x-3).
  \] 

  De esta forma $E(A) = \{2,3\}$, $K_2 = \ker(M-2I)^3$ y $K_2 = \ker(M-3I)$. Ahora calculemos una base para $K_2$ y $K_2$, por eliminación gaussiana tenemos que
    \begin{align*}
      (M-2I)^3 &= \begin{pmatrix} 0 &-2 & 1 & 1 \\ 0 & 0 & 0 & 0 \\ 0 & 0 & 0 & 0 \\ 0 & -2 & 1 & 1 \end{pmatrix}
        \xrightarrow{\text{FERR}} \begin{pmatrix} 0 & 1 & -1/2& -1/2\\ 0 & 0 & 0 & 0 \\ 0 & 0 & 0 & 0 \\ 0 & 0 & 0 & 0 \end{pmatrix}, \\
      M-3I &= \begin{pmatrix} -1 & -1 & 0 & 1 \\ 0 & 0 & -1 & 0 \\ 0 & 1 & -2 & 0 \\ 0 & -1 & 0 & 0 \end{pmatrix}
        \xrightarrow{\text{FERR}} \begin{pmatrix} 1 & 0 & 0 & -1 \\ 0 & 1 & 0 & 0 \\ 0 & 0 & 1 & 0 \\ 0 & 0 & 0 & 0 \end{pmatrix}.
    \end{align*}
  Así podemos ver que $B_2 = \{ (1, 0, 0, 0)^t, (0, 1, 2, 0)^t, (0, 1, 0, 2) \}$ y $B_3 = \{(1, 0, 0, 1)^t\}$ son bases para $K_2$ y $K_3$ con respecto. Por los teoremas vistos claro que $B_2 \cup B_3$ forma una base de $\F^n$ así sea $P$ la matriz cuyas columnas son los elementos de $B_2$ y $B_3$, entonces $P$ es invertible y por tanto
  \[
    P = \begin{pmatrix} 1 & 0 & 0 & 1 \\ 0 & 1 & 1 & 0 \\ 0 & 2 & 0 & 0 \\ 0 & 0 & 2 & 1 \end{pmatrix}
      \Eqand
    P^{-1} = \begin{pmatrix} 1 & 2 & -1 & -1 \\ 0 & 0 & 1/2 & 0 \\ 0 & 1 & -1/2 & 0 \\ 0 &-2 & 1 & 1 \end{pmatrix}.
  \]
  
  Para finalizar, haciendo las cuentas, notemos que $P^{-1}AP$ es una matriz por bloques.
  \[
    P^{-1}AP = \begin{pmatrix} 2 & -1 & 1 & 0 \\ 0 & 3/2 & 1/2 & 0 \\ 0 & -1/2 & 5/2 & 0 \\ 0 & 0 & 0 & 3 \end{pmatrix}.
  \]
\end{example}



\subsection{Ciclos de vectores propios}

Ya hemos visto que la base que nos permite encontrar la fórma canónica de Jordan está compuesta de vectores propios generalizados. Pero no son cualesquiera, sino que cumplen una condición muy específica, estos cumplen que 
\[
  (M-\lambda I)^{k-1}v_k = (M-\lambda I)^{k-2}v_{k-1} = \cdots = (M-\lambda)v_2 = v_1,
\]
para algún $k$. Esto nos da el motivo para la siguiente definición.

\begin{defi}
  Sea $M \in \M_n(\C)$, $\lambda \in E(M)$ y $v \in K_\lambda$. Si $\ell$ es el entero positivo más pequeño para el cual $(M-\lambda I)^\ell v = \bec 0$, entonces el conjunto
    \[
      C_v = \{ (M-\lambda I)^{\ell-1}v, (M-\lambda I)^{\ell-2}v, \cdots, (M-\lambda I)v, v \}.
    \]
  es conocido como un \emph{ciclo} o \emph{cadena de Jordan} de vectores propios  $v$ asociado al valor $\lambda$.  Los vectores $(M-\lambda I)^{\ell-1}v$ y $v$ son llamados el \emph{vector inicial} y el \emph{vector final} del ciclo, respectivamente. Además $\ell$ es conocido como la \emph{longitud del ciclo}.
\end{defi}

Con esto tenemos el último ingrediente para encontrar la fórma canónica de Jordan de una matriz, todo lo que tenemos que hacer es encontrar bases de $K_\lambda$ que sean ciclos de vectores propios, no necesariamente ocurre que haya un solo ciclo, puede ser más de uno, lo importante es que forme una base.

Existen varias cuestiones: ¿cuándo los ciclos son linealmente independientes?, ¿dos ciclos distintos siempre son linealmente independientes? y ¿siempre existe una base compuesta de ciclos de vectores propios? 

Para la primera pregunta, los ciclos siempre son linealmente independientes. De este modo no es necesario comprobar la independencia lineal de un ciclo.

\begin{teor}
  Cada ciclo de vectores propios generalizados de una matriz $M\in\M_n(\C)$ es linealmente independiente.
\end{teor}

Para la segunda pregunta, para que la unión de ciclos sean linealmente independientes se necesita que sus vectores iniciales linealmente independientes.

\begin{teor} \label{teor:IndepCiclos}
  Sean $M \in \M_n(\C)$ y $\lambda\in E(M)$. Si $C_{v_1}, C_{v_2}, \cdots, C_{v_k}$ son ciclos de vectores propios generalizados de $M$ correspondiente a $\lambda$ tal que los vectores iniciales de los $C_{v_i}$ son distintos y conforman un conjunto linealmente independiente. Entonces los $C_{v_i}$ son disjuntos dos a dos y su unión $C = \bigcup_{i=1}^k C_{v_i}$ es un conjunto linealmente independiente.
\end{teor}

 Y para la última pregunta, el espacio $K_\lambda$ siempre tiene una base que consiste en una unión disjunta de ciclos de vectores propios.

\begin{teor}
  Sea $M \in \M_n(\C)$ y $\lambda \in E(M)$, entonces $K_\lambda$ tiene una base que consiste en una unión de ciclos disjuntos de vectores propios generalizados correspondientes a $\lambda$.
\end{teor}

Las demostraciones son un poco complejas y no son tan necesarias para entender el tema. Lo importante es saber que siempre se puede encontrar una base de $K_\lambda$ que consiste en una unión de ciclos disjuntos. En la siguiente parte, abundaremos un poco más en el proceso para encontrar los ciclos.



\subsection{El diagrama de puntos}

Con lo que tenemos hasta ahora ya es visible que toda matriz tiene una forma canónica de Jordan. Lo último que falta sería tener la unicidad y un proceso para encontrar la base que me permite llevar una matriz a su forma canónica de Jordan.

Para ello pensemos en una matriz $M\in\M_n(\C)$ y $\lambda \in E(M)$, para continuar adoptemos la siguiente convención, en primer lugar, si $B_\lambda$ es la base de $K_\lambda$ que está compuesta de ciclos de vectores propios, consideremos que los ciclos de tal manera que la longitud de los ciclos aparezca en orden decreciente. Es decir, si $B_\lambda = C_1 \cup C_2 \cup \cdots \cup C_k$ donde cada $C_i$ es un ciclo de longitud $\ell_i$ entonces $\ell_1 \geq \ell_2 \geq \cdots \geq \ell_k$.

Ahora, ¿cómo se ve esto en la matriz? Consideremos el caso donde $B_\lambda = C_1 \cup C_2 \cup C_3 \cup C_4$ donde $C_1$ es un ciclo de tamaño 3, $C_2$ es un ciclo de tamaño 3 y $C_3$ es un ciclo de tamaño 2 y $C_4$ es un ciclo de tamaño 1, haciendo el mismo análisis del inicio de la subsección \ref{subsec:VecPropGen}, pero a la inversa, podemos determinar que el bloque correspondiente al subespacio $K_\lambda$ sería 
\[
  A_\lambda = \begin{pmatrix}\lambda & 1 & 0 & 0 & 0 & 0 & 0 & 0 & 0 \\ 0 &\lambda & 1 & 0 & 0 & 0 & 0 & 0 & 0 \\  0 & 0 &\lambda & 0 & 0 & 0 & 0 & 0 & 0 \\  0 & 0 & 0 & \lambda & 1 & 0 & 0 & 0 & 0 \\ 0 & 0 & 0 & 0 & \lambda & 1 & 0 & 0 & 0 \\ 0 & 0 & 0 & 0 & 0 & \lambda & 0 & 0 & 0 \\ 0 & 0 & 0 & 0 & 0 & 0 & \lambda & 1 & 0 \\ 0 & 0 & 0 & 0 & 0 & 0 & 0 &\lambda & 0 \\ 0 & 0 & 0 & 0 & 0 & 0 & 0 & 0 & \lambda \end{pmatrix}
  = \begin{spmatrix}{c|c|c|c}
    J_{3,\lambda} & \bec 0 & \bec 0 & \bec 0 \\\hline
    \bec 0 & J_{3,\lambda} & \bec 0 & \bec 0 \\\hline
    \bec 0 & \bec 0 & J_{2,\lambda} & \bec 0 \\\hline
    \bec 0 & \bec 0 & \bec 0 & J_{1,\lambda} 
  \end{spmatrix}.
\]
De esta forma, el número y tamaño de ciclos, determina directamente como se verán los bloques de Jordan.

De esta forma, es primordial saber cómo son los ciclos que conforman la base de cada $K_\lambda$. Para que la visualización y cálculo de los ciclos sea más sencilla utilizamos una matriz de puntos llamado \emph{diagrama de puntos}. Supongamos que tenemos una base $B_\lambda = C_1 \cup \cdots \cup C_k$ de $K_\lambda$ compuesta de ciclos en orden decreciente por su longitud, además la longitud de $C_i$ es $\ell_i$. Entonces, el diagrama de puntos de $K_\lambda$ estará configurado de acuerdo a las siguiente reglas
\begin{enumerate}
  \item La matriz consiste de $k$ columnas (una columna por cada ciclo).
  \item De izquierda a derecha, la columna $i$-ésima consiste de $\ell_i$ puntos que corresponden a los vectores del ciclo $C_i$, comenzado con el vector inicial en la primera fila y continuando hacia abajo hasta el vector final.
\end{enumerate}

De esta forma, si $v_i$ corresponde al vector final del ciclo $C_i$, entonces el diagrama de puntos de $K_\lambda
$ sería algo así:
\[
  \begin{array}{llll} 
    \bullet (M-\lambda I)^{\ell_1-1}v_1 & \bullet (M-\lambda I)^{\ell_2-1}v_2 & \cdots & \bullet (M-\lambda I)^{\ell_k-1}v_k  \\
    \bullet (M-\lambda I)^{\ell_1-2}v_1 & \bullet (M-\lambda I)^{\ell_2-2}v_2 & \cdots & \bullet (M-\lambda I)^{\ell_k-2}v_k  \\
    \vdots & \vdots & \ddots & \vdots \\
    \vdots & \vdots & \cdots & \bullet (M-\lambda I)v_k \\
    \vdots & \vdots & \cdots & \bullet v_k \\
    \vdots & \bullet (M-\lambda I)v_2 \\
    \vdots & \bullet v_2 \\
    \bullet (M-\lambda I)v_1 \\
    \bullet v_1 
  \end{array}
\]

Como ya mencionamos, el numero de columnas es el mismo que el de ciclos, en este caso $k$. El número de puntos de un renglón es menor que el de las columnas que la preceden, ya que los ciclos están ordenados de manera decreciente por su longitud, de esta forma el número de renglones es el mismo que el del primer ciclo $\ell_1$. 

Ahora, denotemos como $r_i$ al número de puntos en el renglón $i$-ésimo, con $i \in \{1,\ldots,\ell_1\}$, notemos que por construcción $r_1 \geq r_2 \geq \cdots \geq r_{\ell_1}$.

\begin{example}
  Sea $B_\lambda = C_1 \cup C_2 \cup C_3 \cup C_4$ una base por ciclos de $K_\lambda$ donde $C_1$ es un ciclo de tamaño 3, $C_2$ es un ciclo de tamaño 3 y $C_c$ es un ciclo de tamaño 2 y $C_4$ es un ciclo de tamaño 1. Encontremos su diagrama de puntos.

  \examplesolution

  En primer lugar, dado que tenemos 4 ciclos, entonces tendremos 4 columnas. La primera columna debe tener el mismo número de puntos que la longitud ciclo $C_1$, la segunda el mismo que la longitud ciclo $C_2$ y así consecutivamente. Entonces debe haber 3 puntos en la primera y segunda columna, 2 en la tercera y 1 en la cuarta, así el diagrama sería el siguiente.
  \[
    \begin{array}{ccccc}
      & C_1 & C_2 & C_3 & C_4 \\
      r_1 & \bullet & \bullet & \bullet & \bullet \\
      r_2 & \bullet & \bullet & \bullet\\
      r_3 & \bullet & \bullet \\
      r_4 & \bullet & \bullet 
    \end{array}
  \]
  Además podemos ver que hay 4 puntos en el primer renglón, 3 en el segundo y 2 en el tercero y cuarto. Así $r_1 = 4$, $r_2 = 3$ y $r_3 = r_4 = 2$.
\end{example}

\begin{example}
  Supongamos que el subespacio $K_\lambda$ tiene el siguiente diagrama de puntos.
  \[ \begin{array}{ccccc}
    \bullet & \bullet & \bullet & \bullet \\
    \bullet & \bullet & \\
    \bullet & 
  \end{array} \]
  Describamos cómo debe ser la base por ciclos de vectores propios generalizados de $K_\lambda$ y la forma del bloque asociado con $K_\lambda$.

  \examplesolution

  En primer lugar, dado que hay 4 columnas, entonces debe existir $C_1$, $C_2$, $C_3$ y $C_4$ ciclos disjuntos de vectores propios generalizados tal que $B_\lambda = C_1\cup C_2\cup C_3\cup C_4$ es una base de $K_\lambda$. Más aún, midiendo el número de puntos en cada columna tenemos que $C_1$ tiene longitud 3, $C_2$ tiene longitud 2 y finalmente $C_3$ y $C_4$ tienen longitud 1.

  Por el análisis de las matrices por bloques de Jordan hecho con anterioridad, es fácil ver que el bloque correspondiente a $K_\lambda$ debe ser
  \[
    A_\lambda = \begin{pmatrix}\lambda & 1 & 0 & 0 & 0 & 0 & 0 \\ 0 & \lambda & 1 & 0 & 0 & 0 & 0 \\ 0 & 0 & \lambda & 0 & 0 & 0 & 0 \\ 0 & 0 & 0 & \lambda & 1 & 0 & 0 \\ 0 & 0 & 0 & 0 & \lambda & 0 & 0 \\ 0 & 0 & 0 & 0 & 0 & \lambda & 0 \\ 0 & 0 & 0 & 0 & 0 & 0 & \lambda \end{pmatrix}
    = \begin{spmatrix}{c|c|c|c}
      J_{3,\lambda} & \bec 0 & \bec 0 & \bec 0 \\\hline
      \bec 0 & J_{2,\lambda} & \bec 0 & \bec 0 \\\hline
      \bec 0 & \bec 0 & J_{1,\lambda} & \bec 0 \\\hline
      \bec 0 & \bec 0 & \bec 0 & J_{1,\lambda} 
    \end{spmatrix}.
  \]
\end{example}

Ahora, aunque el diagrama de puntos nos permite visualizar la longitud y número de ciclos, no sería de mucha ayuda si tuviésemos que calcular los ciclos para construirlo. En otras palabras, deseamos buscar un método para construir el diagrama sin tener que recurrir a la construcción de los ciclos.

Para ello pensemos en que tiene en común los vectores en el mismo renglón. Notemos que todos los elementos del primer renglones son vectores propios, es decir están en $\ker(M-\lambda I)$, los vectores en la primera y segunda fila son todos vectores en que están en $\ker(M-\lambda I)^2$, los vectores de la primera a la tercera fila están en $\ker(M-\lambda I)$, y así consecutivamente.
\[
  \begin{array}{llll@{\qquad}l} 
    \bullet (M-\lambda I)^{\ell_1-1}v_1 & \bullet (M-\lambda I)^{\ell_2-1}v_2 & \cdots & \bullet (M-\lambda I)^{\ell_k-1}v_k  & \text{Vectores en } \ker(M-\lambda I)\\
    \bullet (M-\lambda I)^{\ell_1-2}v_1 & \bullet (M-\lambda I)^{\ell_2-2}v_2 & \cdots & \bullet (M-\lambda I)^{\ell_k-2}v_k  & \text{Vectores en } \ker(M-\lambda I)^2 \\
    \vdots & \vdots & \ddots & \vdots &  \hspace{5em} \vdots \\
    \vdots & \vdots & \cdots & \bullet (M-\lambda I)v_k  & \hspace{5em}\vdots  \\
    \vdots & \vdots & \cdots & \bullet v_k  & \hspace{5em}\vdots  \\
    \vdots & \bullet (M-\lambda I)v_2  & & & \hspace{5em}\vdots \\
    \vdots & \bullet v_2  & &  & \hspace{5em}\vdots \\
    \bullet (M-\lambda I)v_1 & & & & \hspace{5em}\vdots \\
    \bullet v_1 & & & & \text{Vectores en } \ker(M-\lambda I)^{\ell_1}\\
  \end{array}
\]

Pero esta relación va más allá, resulta que si tomamos todos los vectores que están en los primeros $r$ renglones estos forman una base de $\ker(M-\lambda I)^r$.
\[
  \begin{array}{llll@{\quad}l} 
    \bullet (M-\lambda I)^{\ell_1-1}v_1 & \bullet (M-\lambda I)^{\ell_2-1}v_2 & \cdots & \bullet (M-\lambda I)^{\ell_k-1}v_k  & \\
    \vdots & \vdots & \ddots & \vdots &
        \raisebox{0.35em}{\smash{$ \left.\rule{0pt}{1cm}\right\}$ Base de $\ker(M-\lambda I)^r$ }}\\
    (M-\lambda I)^{\ell_1-r}v_1 & (M-\lambda I)^{\ell_2-r}v_2 & \cdots & \bullet (M-\lambda I)^{\ell_k-r}v_k  &   \\
    \vdots & \vdots & \ddots & \vdots &  \\
    \vdots & \vdots & \cdots & \bullet (M-\lambda I)v_k  &   \\
    \vdots & \vdots & \cdots & \bullet v_k  &   \\
    \vdots & \bullet (M-\lambda I)v_2  & & &  \\
    \vdots & \bullet v_2  & &  &  \\
    \bullet (M-\lambda I)v_1 & & & &  \\
    \bullet v_1 & & & & \\
  \end{array}
\]
Recordando que $n = \dim\ker(M) + \dim\Im(M)$ para toda matriz $M \in \M_n(\F)$, por el teorema de la dimensión y que $\rango(M) = \dim \Im(M)$, este resultado nos permite formular el siguiente teorema.

\begin{teor}
  Sea $M \in \M_n(\C)$ y $\lambda \in E(M)$, si $r_i$ el número de puntos en el $i$-ésimo renglón del diagrama de puntos de $K_\lambda$, entonces
  \begin{enumerate}
    \item $r_1 + \cdots + r_i = \dim \ker(M-\lambda I)^i $
    \item $r_1 = n - \rango(M-\lambda I)$.
    \item $r_i = \rango(M-\lambda)^{i-1} - \rango(M-\lambda)^i$ si $i>1$.
  \end{enumerate}
\end{teor}

Con este teorema no tenemos que calcular los ciclos de $K_\lambda$ para obtener su diagrama de puntos. Más aún, este teorema nos permite ver que el diagrama de puntos es único para una matriz y dado que el diagrama de puntos indica cual va a ser la forma de la forma canónica de Jordan, entonces tenemos finalmente que toda matriz compleja tiene una única forma canónica de Jordan, salvo el orden de los bloques.

\begin{example}
  Sea $A$ la matriz de $4\times 4$ de entradas complejas dada por 
    \[
      \begin{pmatrix}
      2 & -1 & 0 & 1 \\
      0 & 3 & -1 & 0 \\
      0 & 1 & 1 & 0 \\
      0 & -1 & 0 & 3 
    \end{pmatrix}.
    \]
  Encuentre el diagrama de puntos de cada uno de sus espacios propios generalizados. Posteriormente encuentre su forma canónica de Jordan y la base que manda la matriz a su forma canónica de Jordan.

  \examplesolution

  Primero encontremos los valores propios de $A$, usando el polinomio característico tenemos que
    \[
      \det(M-xI) = \det\begin{pmatrix}
        2-x & -1 & 0 & 1 \\
        0 & 3-x & -1 & 0 \\
        0 & 1 & 1-x & 0 \\
        0 & -1 & 0 & 3-x
      \end{pmatrix}
      = (x-2)^3(x-3).
    \]

  Así tenemos que $E(A) = \{2,3\}$. Ahora, dado que $\dim(K_2) = 3$ entonces el diagrama de puntos de $K_2$ tiene 3 puntos, así, solo debemos encontrar los rangos de las matrices $(M-2I)^r$ hasta que los renglones sumen 3. Por eliminación gaussiana notemos que
  \[
    \rango(A-2I) = 2 
      \Eqand
    \rango(A-2I)^2 = 1.
  \]
  De este modo tenemos que $r_1  = 4 - \rango(A-2I) = 2$ y $r_2 = \rango(A-2I) - \rango(A-2I)^2 = 1$. De aquí el diagrama de puntos de $K_2$ es 
  \[
    \begin{array}{cc}
      \bullet & \bullet\\
      \bullet
    \end{array}
  \]

  Ahora, para $K_3$, dado que $\dim(K_3) = 1$ entonces su diagrama de puntos está únicamente dado por un solo punto. De esta forma, juntando con todo lo que sabemos hasta ahora, podemos determinar que la forma canónica de Jordan $J$ de $A$ es
  \[
    J = \begin{pmatrix} 2 & 1 & 0 & 0 \\ 0 & 2 & 0 & 0 \\ 0 & 0 & 2 & 0 \\ 0 & 0 & 0 &3 \end{pmatrix}
    = \begin{spmatrix}{c|c|c}
      J_{2,2} & \bec 0 & \bec 0\\ \hline
      \bec 0 & J_{1,2} & \bec 0\\ \hline
      \bec 0 & \bec 0 & J_{1,3}
    \end{spmatrix}.
  \]

  Por último, encontremos la base que permite mandar la matriz a su forma canónica de Jordan. Para $K_3$ es fácil, dado que $K_3 = E_3$ entonces basta con tomar algún elemento no nulo de $\ker(M-3I)$, por ejemplo consideremos la base de $K_3$ dada por $B_3 = \{(1,0,0,1)^t\}$.

  Ahora, para encontrar una base $B_2$ de $K_2$ tenemos que ponerle nombres a los puntos de su diagrama de puntos, por ejemplo:
  \[
    \begin{array}{ll}
      \bullet (M-2I)v_1 & \bullet v_2\\
      \bullet v_1
    \end{array}
  \]
  Lo primero que hay que ver es que $v_1 \in \ker(M-2I)^2$ pero $v_1 \notin \ker(M-2I)$, así busquemos una base para $\ker(M-2I)^2$, por eliminación gaussiana tenemos que
    \[
      (A-2I)^2 = \begin{pmatrix}0&-2&1&1\\ 0&0&0&0\\ 0&0&0&0\\ 0&-2&1&1\end{pmatrix}
        \xrightarrow{\text{FERR}}
        \begin{pmatrix}0&1&-1/2&-1/2\\ 0&0&0&0\\ 0&0&0&0\\ 0&0&0&0\end{pmatrix}.
    \]
  De aquí tenemos que $\{ (1, 0, 0, 0)^t, (0,1,2,0)^t, (0,1,0,2)^t \}$ es una base de $\ker(M-2I)^2$, ahora veamos cuales no pertenecen a $\ker(M-2I)$, notemos que
  \[
    (M-2I) \begin{pmatrix}
      1 \\ 0 \\ 0 \\ 0
    \end{pmatrix} = \bec 0,
        \qquad
    (M-2I) \begin{pmatrix}
      0 \\ 1 \\ 2 \\ 0
    \end{pmatrix} = \begin{pmatrix}
      -1 \\ -1 \\ -1 \\ -1
    \end{pmatrix} 
      \Eqand
      (M-2I) \begin{pmatrix}
        0 \\ 1 \\ 0 \\ 2
      \end{pmatrix} = \begin{pmatrix}
        1 \\ 1 \\ 1 \\ 1
      \end{pmatrix}.
  \]
  Entonces solo el primer vector pertenece a $\ker(M-2I)$, así podemos usar cualquiera de los otros dos para ser $v_1$, en este caso supongamos que $v_2 = (0,1,2,0)^t$, notemos entonces que por lo calculado
    \[
      (M-2I)v_1 = \begin{pmatrix}
        -1 \\ -1 \\ -1 \\ -1
      \end{pmatrix}.
    \]

  Ahora, dado que $v_2$ y $(M-2I)v_1$ forman una base de $\ker(M-2I)$, entonces basta con encontrar un vector linealmente independiente de $(M-2I)v_1$ que se encuentre en $\ker(M-2I)$. Para ello calculemos una base de $\ker(M-2I)$, por eliminación gaussiana tenemos que
    \[
      A-2I = \begin{pmatrix} 0 & -1 & 0 & 1 \\ 0 & 1 & -1 & 0 \\ 0 & 1 & -1 & 0 \\ 0 & -1 &0 & 1 \end{pmatrix}
        \xrightarrow{\text{FERR}}
        \begin{pmatrix} 0 & 1 & 0 & -1 \\ 0 & 0 & 0 & 0 \\ 0 & 0 & 1 & -1 \\ 0 & 0 & 0 & 0 \end{pmatrix}.
    \]
  De aquí tenemos que $\{(1,0,0,0)^t, (0,1,1,1)^t\}$ forma una base de $\ker(M-2I)$, notemos que $(1,0,0,0)^t$ es linealmente independiente de $(M-2I)v$, asi escojamos $v_2 = (1,0,0,0)^t$. De esta forma tenemos que el diagrama de puntos queda de la siguiente forma
  \[
    \begin{array}{ll}
      \bullet \begin{pmatrix} -1 \\ -1 \\ -1 \\ -1 \end{pmatrix} & \bullet \begin{pmatrix}1 \\ 0 \\ 0 \\ 0\end{pmatrix}\\
      \bullet \begin{pmatrix} 0 \\ 1 \\ 2 \\ 0 \end{pmatrix}
    \end{array}
  \]
  No es necesario comprobar la independencia lineal, esto se debe al teorema \ref{teor:IndepCiclos}, dado que pedimos que los inicios de los ciclos sean linealmente independientes.

  Ahora, es bastante simple que si $P$ es la matriz cuyas columnas son las bases calculadas, entonces $P$ es invertible y además
  \[
    P = \begin{pmatrix}
      0 & -1 & 1 & 1 \\
      1 & -1 & 0 & 0 \\
      2 & -1 & 0 & 0 \\
      0 & -1 & 0 & 1 
    \end{pmatrix}
    \Eqand
    P^{-1} = \begin{pmatrix} 0 & -1 & 1 & 0 \\ 0 & -2 & 1 & 0 \\ 1 & 0 & 0 & -1 \\ 0 & -2 & 1 & 1 \end{pmatrix}.
  \]
  De este modo, haciendo el cálculo final, obtenemos que
  \[
    P^{-1}AP = \begin{pmatrix} 2 & 0 & 0 & 0 \\ 1 & 2 & 0 & 0 \\ 0 & 0 & 2 & 0 \\ 0 & 0 & 0 & 3 \end{pmatrix} = J.
  \]
\end{example}

\begin{example}
  Sea $T$ la matriz de $6\times 6$ de entradas complejas dada por 
  \[
    T = \begin{pmatrix}
      0 & 1 & 0 & 0 & 0 & 0 \\
      0 & 0 & 0 & 2 & 0 & 0 \\
      0 & 0 & 0 & 0 & 0 & 1 \\
      0 & 0 & 0 & 0 & 0 & 0 \\
      0 & 0 & 0 & 0 & 0 & 0 \\
      0 & 0 & 0 & 0 & 0 & 0
    \end{pmatrix}.
  \]
  Calculemos su forma canónica de Jordan y la base que manda la matriz a su forma canónica de Jordan.

  \examplesolution

  En primer lugar, es fácil ver que $E(T) = \{0\}$, así solo hay que calcular un diagrama de puntos. Primero veamos que $\dim(K_0) = 6$, por lo que calculemos los rangos de $T^r$ hasta que los renglones sumen 6. Notemos que
  \[ \rango(T) = 3 \Eqand \rango(T^2) = 1.\]
  así tenemos que $r_1 = 6-3 = 3$, $r_2 = 3-1 = 1$ y es fácil ver que $r_3 = 1$, ya que la suma de los puntos en los renglones debe ser 6. Así el diagrama de puntos es el siguiente.
  \[\begin{array}{ccc}
    \bullet & \bullet & \bullet \\
    \bullet & \bullet \\
    \bullet
  \end{array}\]
  Así es fácil ver que la forma canónica de Jordan $J$ de $T$ es 
  \[J =  \begin{pmatrix} 0 & 1 & 0 & 0 & 0 & 0 \\ 0 & 0 & 1 & 0 & 0 & 0 \\ 0 & 0 & 0 & 0 & 0 & 0 \\ 0 & 0 & 0 & 0 & 1 & 0 \\ 0 & 0 & 0 & 0 & 0 & 0 \\ 0 & 0 & 0 & 0 & 0 & 0 \end{pmatrix} 
    = \begin{spmatrix}{c|c|c}
      J_{3,0} & \bec 0 & \bec 0 \\\hline
      \bec 0 & J_{2,0} & \bec 0 \\\hline
      \bec 0 & \bec 0 & J_{1,0}
    \end{spmatrix}. \]
  
  Ahora, para encontrar la base que me manda la matriz a su forma canónica de Jordan, notemos que $K_0 = \F^n$, así una base de $K_0$ es la canónica. Notemos que en el diagrama de puntos hay un valor que está en $\ker(T^3)$ pero no está en $\ker(T^2)$, así busquemos que valores de la base canónica lo cumplen. Notemos que
    \[
      T^2 = \begin{pmatrix} 0 & 0 & 0 & 2 & 0 & 0 \\ 0 & 0 & 0 & 0 & 0 & 0 \\ 0 & 0 & 0 & 0 & 0 & 0 \\ 0 & 0 & 0 & 0 & 0 &0 \\ 0 & 0 & 0 & 0 & 0 & 0 \\ 0 & 0 & 0 & 0 & 0 & 0 \end{pmatrix},
    \]
  de aquí se puede ver que solo $e_4$ cumple que $T^2e_4 \neq 0$. Así calculemos el ciclo de $e_4$, notemos que
  \[
    Te_4 = 2e_2 \Eqand T^2e_4 = 2e_1.
  \]
  De aquí tenemos que el primer ciclo es $C_1 = \{2e_1, 2e_2, e_4\}$. 

  Revisando nuevamente el diagrama de puntos, vemos que para el segundo ciclo, su vector final está en $\ker(T^2)$ pero no está en $\ker(T)$, así calculemos una base de $\ker T^2$. Dado que ya teníamos calculado $T^2$  es fácil ver que  $ \{ e_1, e_2, e_3, e_5, e_6 \} $ es una base de $\ker(T^2)$, ahora $e_1, e_3, e_5$ están en $\ker(T)$ por lo que solo quedan $e_2$ y $e_6$, pero notemos que $e_2 \in \inner{C_1}$ por lo que también queda descartado, dado que queremos que sea linealmente independiente con $C_1$, así el único candidato es $e_6$, veamos que $T(e_6) = e_3$ así el segundo cíclo está dado por $C_2 = \{e_3, e_6\}$.

  Ahora, es fácil ver que $C_1$ y $C_2$ son linealmente independientes, además el único vector de la base canónica que es linealmente independiente de  $C_1 \cup C_2$ es $e_5 \in \ker(T)$, así, se puede deducir que el último ciclo es $C_3 = \{e_5\}$. De este modo, el diagrama de puntos nos quedó de la siguiente manera 
  \[\begin{array}{lll}
    \bullet 2e_1 & \bullet e_3 & \bullet e_5 \\
    \bullet 2e_2 & \bullet e_6 \\
    \bullet e_4
  \end{array}\]

  Sea $P$ la matriz cuyas columnas son la base de $K_0$ dada por $C_1 \cup C_2 \cup C_3$, entonces $P$ es invertible y además
  \[
    P =  \begin{pmatrix}
      2 & 0 & 0 & 0 & 0 & 0 \\
      0 & 2 & 0 & 0 & 0 & 0 \\
      0 & 0 & 0 & 1 & 0 & 0 \\
      0 & 0 & 1 & 0 & 0 & 0 \\
      0 & 0 & 0 & 0 & 0 & 1 \\
      0 & 0 & 0 & 0 & 1 & 0
    \end{pmatrix}
    \Eqand
    P^{-1} =  \begin{pmatrix} 1/2 & 0 & 0 & 0 & 0 & 0 \\ 0 & 1/2 & 0 & 0 & 0 & 0 \\ 0 & 0 & 0 & 1 & 0 & 0 \\ 0 & 0 & 1 & 0 & 0 & 0 \\  0 & 0 & 0 & 0 & 0 & 1 \\ 0 & 0 & 0 & 0 & 1 & 0 \end{pmatrix}.
  \]
  Finalmente, realizando una cuenta rutinaria, podemos comprobar que
    \[
      P^{-1}TP 
        = \begin{pmatrix} 0 & 1 & 0 & 0 & 0 & 0 \\ 0 & 0 & 1 & 0 & 0 & 0 \\ 0 & 0 & 0 & 0 & 0 & 0 \\ 0 & 0 & 0 & 0 & 1 &0 \\ 0 & 0 & 0 & 0 & 0 & 0 \\ 0 & 0 & 0 & 0 & 0 & 0 \end{pmatrix} = J.
    \]
\end{example}




% =============================================================================
% CAPÍTULO 2
% =============================================================================

\chapter{Espacios con producto interno}


En este capítulo nuestro objetivo principal es estudiar a los espacios vectoriales donde existe la noción de ángulo o longitud. Para estudiar estos conceptos geométricos se hace uso de una función especial llamado producto interno.

Un ejemplo de producto interno es el conocido producto punto, definido en el espacios $R^2$ que nos permite obtener la longitud de un vector o el ángulo entre dos vectores.

Nuestro objetivo en este capítulo será, generalizar esta idea sobre los espacios vectoriales con campo en los reales o complejos y estudiar sus principales propiedades.

\section{Producto interno}

\begin{defi}
  Sea $\K$ el campo de los números reales o de los complejos y $V$ un $\K$-espacio vectorial. Un \emph{producto interno} sobre $V$ es una función $\inner{\cdot, \cdot}\colon V\times V \to \K$ tal que para cualesquiera $v,u,w \in V$ y $\lambda \in \K$ se cumple que:
    \begin{enumerate}
      \item (Linealidad izquierda) $\inner{v+\lambda w, u} = \inner{v,u} + \lambda\inner{w,u}$.
      \item (Hermiticidad) $\inner{v,w} = \overline{\inner{w,v}}$, donde la barra indica la conjugación compleja.
      \item (Definida positiva) $\inner{v,v} > 0$ si $v\neq 0$.
    \end{enumerate}
\end{defi}

Notemos que la primera propiedad indica que $\inner{\cdot, v}$ es una transformación lineal para todo $v \in V$, eso quiere decir que $\inner{v+w,u} = \inner{v,u} + \inner{w,u}$ (separa sumas), $\inner{\lambda v, w} = \lambda \inner{v,w}$ (saca escalares) y $\inner{0,v} = 0$ para cualesquiera $v,u,w \in V$ y $\lambda \in \K$.

Ahora, para la segunda entrada decimos que es \emph{lineal conjugada por la derecha} ya que por 1 y 2 se tiene que
  \[
    \inner{u, v + \lambda w} = \overline{\inner{v + \lambda w, v}} = \overline{\inner{u,v}} +  \overline{\lambda \inner{u,w}} = \inner{v,u} + \bar\lambda \inner{w,u}.
  \]

Notemos que si $\K=\R$ entonces el conjugado está de más, en este caso la propiedad 2 es llamada \emph{simetría}. Sin embargo el conjugado es necesario para los complejos, ya que sin éste, para $v\neq 0$, se tendría que 
  \[
    \inner{v,v}>0 \Eqand \inner{iv, iv} = i^2 \inner{v,v} = -\inner{v,v}>0.
  \]

Para continuar, por convención, a lo largo de todo el capítulo $\K$ será el campo de los números reales o de los complejos a menos que se indique lo contrario.

Otra convención, es que los espacios $\K^n$ tienen un producto interno al que usualmente le llamamos \emph{canónico}, este es el producto interno $\bcdot\colon  \K^n \times \K^n$ dado por $v \bcdot w = v^t \overline{w}$. Por último, también se manejará la siguiente notación para matrices.

\begin{defi}
  Sea $M \in \M_{n\times m}(\C)$ dada por $M = (m_{ij})$, definimos su \emph{conjugada} como la matriz dada por $\overline{M} = (\overline{m_{ij}})$. De igual forma, definimos su \emph{traspuesta conjugada} o \emph{adjunta} como la matriz dada por $M^* = \overline{M}^t$.
\end{defi}

\begin{example}
  Sean $v = (x_1, y_1)$ y $w = (x_2, y_2)$ elementos de $\R^2$, definamos el producto $\inner{\cdot,\cdot}\colon \R^2 \times \R^2 \to \R$ dado por
    \[ \inner{v,w} = x_1 x_2 - y_1x_2 - x_1y_2 + 4y_1y_2. \]
  Demostremos que es un producto interno sobre $\R^2$.

  \examplesolution

  La demostración de las propiedades 1 y 2 son fáciles de comprobar. Basta con hacer los cálculos pertinentes, otra forma es ver que 
    \[
      \inner{v,w} = v^t \begin{pmatrix}
        1 & -1 \\ -1 & 4
      \end{pmatrix} w.
    \]
  En esta forma es fácil verificar la linealidad izquierda y la simetría (dado que la matriz es simétrica). Ahora, para ver que es definida positiva basta con verificar que
   \[
     \inner{v,v} = (x_1 - y_1)^2 + 3 y_1^2,
   \]
   es claro que si $v \neq 0$ entonces $\inner{v,v} > 0$.
\end{example}

\begin{defi}
  Un \emph{espacio con producto interno} $V$ es un espacio vectorial real o complejo con un producto interno definido sobre ese espacio.
\end{defi}

Un espacio con producto interno real a menudo es llamado un \emph{espacio euclidiano} mientras que uno complejo es referido como un \emph{espacio unitario}.



\section{La geometría de un espacio con producto interno}

Como mencionamos al principio, un producto interno nos ayudará a definir los conceptos de distancia y ángulo dentro de un espacio vectorial con producto interno. En esta sección abordaremos con más detalle estos conceptos geométricos.


\subsection{Norma de un vector}

\begin{defi}
  Sea $V$ un espacio con producto interno, definimos la \emph{norma} de un vector $v \in V$ como $\norm{v} = \sqrt{\inner{v,v}}$.
\end{defi}

La norma de un vector puede pensarse como su \emph{magnitud}. Además, la norma nos permite definir una noción de distancia entre dos vectores, si pensamos en un vector como un ``camino'' entonces para ir de un vector $v$ al vector $w$ tenemos que ``caminar'' un vector $w-v$ ya que $w = v + (w-v)$, así podemos pensar en $\norm{w-v}$ como la distancia del vector $v$ al vector $w$.

El siguiente teorema demostrará algunas propiedades de la norma que intuitivamente nos permiten considerarla como la magnitud de un vector.

\begin{teor} \label{teor:propNorm}
  Sea $V$ un espacio con producto interno, entonces para cualesquiera $v,w \in V$ y $\lambda \in \K$ se satisfacen las siguientes propiedades:
  \begin{enumerate}
    \item $\norm{v}>0$ para $v \neq 0$.
    \item $\norm{\lambda v} = \abs{\lambda}\norm{v}$.
    \item $\norm{v \pm w}^2 = \norm{v}^2 \pm 2 \Re(\inner{v,w}) + \norm{w}^2$.
    \item Desigualdad de Cauchy-Schwartz: $\abs{\inner{v,w}} \leq \norm{v} \norm{w}$.
    \item Desigualdad del triángulo: $\norm{v+w} \leq \norm{v} + \norm{w}$.
  \end{enumerate}
\end{teor}
\begin{proof}~
  \begin{enumerate}
    \item Si $v \neq 0$ por definición $\inner{v,v}>0$ y por propiedades de las desigualdades tenemos que $\norm{v} = \sqrt{\inner{v,v}}>0$.
    
    \item Por la linealidad izquierda y linealidad conjugada derecha tenemos que 
      \[ \norm{\lambda v} = \sqrt{\inner{\lambda v, \lambda v}} = \sqrt{\lambda \bar\lambda \inner{v,v}} = \sqrt{\abs{\lambda}^2} \sqrt{\inner{v,v}} = \abs{\lambda}\norm{v}. \]
    
    \item Aplicando la linealidad izquierda y linealidad conjugada derecha, tenemos que
      \begin{align*}
        \norm{v \pm w}^2 &= \inner{v\pm w, v\pm w} \\
          &= \inner{v,v\pm w} \pm \inner{w, v\pm w} \\
          &= (\inner{v,v} \pm \inner{v,w}) \pm (\inner{w,v} \pm \inner{w,w}) \\
          &= \norm{v}^2 \pm(\inner{v,w} + \inner{w,v}) + \norm{w}^2.
      \end{align*}

      Ahora, por hermiticidad y propiedades de los complejos tenemos que 
        \[
          \inner{v,w} + \inner{w,v} = \inner{v,w} + \overline{\inner{v,w}} = 2\Re(\inner{v,w}).
        \]
      De esta forma podemos concluir que $\norm{v\pm w}^2 = \norm{v}^2 \pm 2\Re(\inner{v,w}) + \norm{w}^2$.

    \item Notemos que si $v = 0$ entonces claramente se sostiene que $\inner{v,w} = 0 = \norm{v} \norm{w}$, de esta forma consideremos el caso donde $v \neq 0$. Ahora, dado que $v \neq 0$ entonces $\norm{v}^2 = \inner{v,v} > 0$ por definición, así consideremos el vector $u$ dado por
      \[
        u = w - \frac{\inner{w,v}}{\norm{v}^2} v.
      \]
    En primer lugar, por linealidad izquierda y definición, se cumple que 
      \begin{align*}
        \inner{u, v} &=  \Inner{w - \frac{\inner{w,v}}{\norm{v}^2} v, v} 
           = \inner{w,v} - \frac{\inner{w,v}}{\norm{v}^2}  \inner{v,v} \\
          &= \inner{w,v} - \frac{\inner{w,v}}{\norm{v}^2}  \norm{v}^2 
           = \inner{w,v} - \inner{w,v} \\
          &= 0.
      \end{align*}
    Así, para todo $\lambda \in \K$ tenemos que $\inner{u,\lambda v} = \bar \lambda \inner{u,v} = 0$. De este modo, dado que $\inner{w,v} \inner{v,w} =\overline{\inner{v,w}}  \inner{v,w} = \abs{\inner{v,w}}^2$, por hermiticidad y propiedades de los complejos, entonces veamos que 
      \begin{align*}
        \norm{u}^2 &= \inner{u, u}
           = \Inner{u, w - \frac{\inner{w,v}}{\norm{v}^2} v} \\
          &= \inner{u,w} + \Inner{u, - \frac{\inner{w,v}}{\norm{v}^2} v} 
           = \inner{u,w} \\
          &= \Inner{w - \frac{\inner{w,v}}{\norm{v}^2} v, w} 
           = \inner{w,w} - \frac{\inner{w,v}}{\norm{v}^2} \inner{v,w} \\
          &= \norm{w}^2 - \frac{\abs{\inner{v,w}}^2}{\norm{v}^2}.
      \end{align*}
    
    Dado que $\norm{u}^2 \geq 0$ por la propiedad 1, entonces podemos concluir que
      \[0 \leq \norm{w}^2 - \frac{\abs{\inner{v,w}}^2}{\norm{v}^2}
          \implies
        \abs{\inner{v,w}}^2 \leq \norm{v}^2 \norm{w}^2.
      \]
    Lo que por propiedades de las desigualdades, finalmente implica que $\abs{\inner{v,w}} \leq \norm{v} \norm{w}$.

    \item Por propiedades de los complejos tenemos que $\Re(\inner{v,w}) \leq \abs{\inner{v,w}}$, pero por la desigualdad de Cauchy-Schwartz tenemos que $\abs{\inner{v,w}} \leq \norm{v} \norm{w}$. De esta forma, por la propiedad 2 tenemos que 
      \begin{align*}
        \norm{v+w}^2 &= \norm{v}^2 + 2 \Re(\inner{v,w}) + \norm{w}^2 \\
          &\leq \norm{v}^2 + 2\norm{v} \norm{w} + \norm{w}^2 \\
          &= (\norm{v} + \norm{w})^2.
      \end{align*}
    Y por propiedades de las desigualdades, podemos concluir que $\norm{v+w} \leq \norm{v}+\norm{w}$. \qedhere
  \end{enumerate}
\end{proof}

\begin{coro} \label{coro:SchwartzEq}
  La igualdad en desigualdad de Cauchy-Schwartz se da si solo si los vectores son linealmente dependientes.
\end{coro}

\begin{proof}
  Supongamos que $v$ y $w$ son linealmente dependientes. Si $v = 0$, por la prueba de la propiedad 4 del teorema \ref{teor:propNorm} ya vimos que se cumple la igualdad, así, supongamos que $v \neq 0$.
  
  Dado que $v \neq 0$, entonces por independencia lineal debe existir $\lambda \in \K$ tal que $w = \lambda v$, de este modo, por el teorema \ref{teor:propNorm}, tenemos que
    \[
      \abs{\inner{v,w}} = \abs{\inner{v,\lambda v}} = \abs{\lambda\inner{v,v}} = \abs{\lambda} \abs{\inner{v,v}} = \abs{\lambda} \norm{v}^2 = \norm{v} (\abs{\lambda} \norm{v}) = \norm{v} \norm{\lambda v} = \norm{v} \norm{w}.
    \]
  
  Ahora, supongamos que se da la igualdad. Es trivial que $v$ y $w$ son linealmente dependientes cuando $v = 0$, pero si $v \neq 0$, entonces por la prueba de la propiedad 4 del teorema \ref{teor:propNorm} tenemos que el vector $u = w - (\inner{w,v}/\norm{v}^2)v$ cumple que $\norm{u} = 0$, pero esto implica que $u = 0$ y por tanto
    \[ w = \frac{\inner{w,v}}{\norm{v}^2} v, \]
  lo que muestra que $u$ y $v$ son linealmente dependientes.
\end{proof}

Ahora, hemos visto que el producto interno define el concepto de distancia, pero este proceso también funciona a la inversa, dado ciertas distancias es posible conocer el producto interno de dos vectores, esta propiedad es conocida como identidad de polarización. 

\begin{teor}[Identidad de polarización]
  Sea $V$ un espacio con producto interno, entonces para cualesquiera $v,w \in V$ se cumple que:
    \begin{enumerate}
      \item Si $\K = \R$ entonces
        \[ \inner{v,w} = \frac{1}{4} \norm{v+w}^2 - \frac{1}{4} \norm{v-w}^2.\]
      \item Si $\K = \C$ entonces
      \[ \inner{v,w} = \frac{1}{4} \norm{v+w}^2 - \frac{1}{4} \norm{v-w}^2 + \frac{i}{4}\norm{v+iw}^2 - \frac{i}{4}\norm{v-iw}^2.\]
    \end{enumerate}
\end{teor}
\begin{proof}
  Por el teorema \ref{teor:propNorm} tenemos que $\norm{v \pm w}^2 = \norm{v}^2 \pm 2 \Re(\inner{v,w}) + \norm{w}^2$, de este modo
    \begin{align*}
      \norm{v+w}^2 - \norm{v-w}^2
        &= (\norm{v}^2 + 2 \Re(\inner{v,w}) + \norm{w}^2) - (\norm{v}^2 - 2 \Re(\inner{v,w}) + \norm{w}^2) \\
        &= 4\Re(\inner{v,w}). \tagthis\label{eq:IdPolRe}
    \end{align*}
  Notemos que si $\K = \R$, entonces $\Re(\inner{v,w}) = \inner{v,w}$, de este modo tenemos que
    \[
      \inner{v,w} = \frac{1}{4} \norm{v+w}^2 - \frac{1}{4} \norm{v-w}^2.
    \]
  
  Ahora, si $\K = \C$, por el teorema \ref{teor:propNorm} y notando que $\Re(-i\inner{v,w}) = \Im(\inner{v,w})$, por propiedades de complejos, entonces tenemos que 
    \begin{align*}
      i\norm{v \pm iw}^2
        &= i\norm{v}^2 \pm 2 i\Re(\inner{v,iw}) + i\norm{w}^2 \\
        &= i\norm{v}^2 \pm 2 i\Re(-i\inner{v,w}) + i\norm{w}^2 \\
        &= i\norm{v}^2 \pm 2 i\Im(\inner{v,w}) + i\norm{w}^2, \\
      i\norm{v + iw}^2 - i\norm{v - iw}^2
        &= (i\norm{v}^2 + 2 i\Im(\inner{v,w}) + i\norm{w}^2) - (i\norm{v}^2 - 2 i\Im(\inner{v,w}) + i\norm{w}^2 ) \\
        &= 4i\Im(\inner{v,w}).  \tagthis \label{eq:IdPolIm}
    \end{align*}
  

  De esta forma, dado que $\inner{v,w} = \Re(\inner{v,w}) + i\Im(\inner{v,w})$, entonces despejando de las ecuaciones \ref{eq:IdPolRe} y \ref{eq:IdPolIm} obtenemos que
    \[
      \inner{v,w} = \frac{1}{4} \norm{v+w}^2 - \frac{1}{4} \norm{v-w}^2 + \frac{i}{4}\norm{v+iw}^2 - \frac{i}{4}\norm{v-iw}^2. \qedhere
    \]
\end{proof}



\subsection{Ángulo entre vectores}

Además de la desigualdad del triángulo, una de las consecuencias de la desigualdad de Cauchy-Schwartz es que para cualesquiera dos vectores $v,w \in V$ no nulos de un espacio euclidiano se cumple que
\[
  \frac{\abs{\inner{v,w}}}{\norm{v}\norm{w}} \leq 1 \implies  -1 \leq \frac{\inner{v,w}}{\norm{v}\norm{w}} \leq 1.
\]
Por las propiedades del coseno, esto implica que existe un único $\theta \in [0,\pi]$ tal que 
\[
  \cos\theta = \frac{\inner{v,w}}{\norm{v}\norm{w}}.
\]
Este número $\theta$ es el que se suele asociar como el ángulo entre los vectores $v$ y $w$.

\begin{defi}
  Sea $V$ un espacio euclidiano y $v,w \in V$ son vectores no nulos, entonces definimos el \emph{ángulo} entre $v$ y $w$ como el único número $\angle(v,w) \in [0,\pi]$ tal que
    \[ \cos\angle(v,w) = \frac{\inner{v,w}}{\norm{v}\norm{w}}. \]
\end{defi}

Notemos que por definición el ángulo no depende de la posición de $v$ y $w$ por lo que $\angle(v,w) = \angle(w,v)$. Además, como se podría suponer, el angulo entre dos vectores cumple muchas de las propiedades geométricas conocidas.

\begin{teor}
  Sea $V$ un espacio euclidiano y $v,w \in V$ son vectores no nulos, se cumplen las siguientes propiedades:
  \begin{enumerate}
    \item Los vectores $v$ y $w$ son linealmente dependientes si y solo si el ángulo entre $v$ y $w$ es $0$ o $\pi$.
    \item Sea $\lambda \in \R-\{0\}$ entonces, los ángulos $\angle(v,w)$ y $\angle(\lambda v,w)$ son iguales si $\lambda > 0$ y son suplementarios si $\lambda < 0$.
    \item (Teorema de cosenos) Se cumple que
      \[ \norm{v \pm w}^2 = \norm{v}^2 + \norm{w}^2 \pm 2\norm{v}\norm{w}\cos\angle(v,w). \]
  \end{enumerate}
\end{teor}

\begin{proof}~
  \begin{enumerate}
    \item Por el corolario \ref{coro:SchwartzEq} tenemos que $v$ y $w$ son linealmente dependientes si y solo si se cumple que $\abs{\inner{v,w}} = \norm{v}\norm{w}$, pero eso se cumple si y solo si $\cos\angle(v,w) = \pm 1$ y por propiedades del coseno se cumple si y solo si $\angle(v,w) = 0$  o $\angle(v,w) = \pi$.
    
    \item Notemos que por propiedades conocidas del producto interno y la norma se cumple que 
      \[
        \cos\angle(\lambda v,w) = \frac{\inner{\lambda v,w}}{\norm{\lambda v} \norm{w}}
          = \frac{\lambda}{\abs{\lambda}}  \frac{\inner{v,w}}{\norm{v} \norm{w}}
          = \frac{\lambda}{\abs{\lambda}} \cos\angle(v,w).
      \]
      Si $\lambda > 1$ entonces $\lambda/\abs{\lambda} = 1$, por lo que $\angle(\lambda v ,w) = \angle(v,w)$, ya que sus cosenos son iguales. Ahora, si $\lambda<0$, entonces $\lambda/\abs{\lambda} = -1$, por lo que $\angle(\lambda v,w) = \pi - \angle(v,w)$ ya que $\cos\angle(\lambda v,w) = -\cos\angle(v,w)$.

      \item Por el teorema \ref{teor:propNorm} tenemos que $\norm{v - w}^2 = \norm{v}^2 \pm 2 \inner{v,w} + \norm{w}^2$ y por definición $\inner{v,w} = \norm{v}\norm{w} \cos\angle(v,w)$, de este modo tenemos que 
        \[ \norm{v \pm w}^2 = \norm{v}^2  + \norm{w}^2 \pm 2 \norm{v}\norm{w} \cos\angle(v,w). \qedhere \]
  \end{enumerate}
\end{proof}

\section{Ortogonalidad y ortonormalidad}

Un concepto de especial interés en la geometría es la perpendicularidad en esta sección generalizaremos más este concepto.

Por lo visto en la sección anterior, tenemos que dos vectores no nulos $v,w \in V$ de un espacio euclidiano cumplen que $\angle(v,w) = \pi/2$ si y solo si $\cos \angle(v,w) = 0$, pero esta ultima condición se cumple si y solo si $\inner{v,w} = 0$.

De este modo, aunque no podamos definir los ángulos en cualquier espacio con producto interno, sí podemos definir un conceptos similar al de la perpendicularidad.

\begin{defi}
  Sea $V$ un espacio con producto interno, decimos que dos vectores $v,w \in V$ son \emph{ortogonales} y lo denotaremos como $v\perp w$, si $\inner{v,w} = 0$. Si $S$ es un conjunto de vectores de $V$, decimos que es un \emph{conjunto ortogonal} si para cualesquiera $v,w \in S$ con $v \neq w$ se cumple que $\inner{v,w} = 0$. Además, decimos que es un conjunto $S$ es un \emph{conjunto ortonormal} si es un conjunto ortogonal y además $\norm{v} = 1$ para todo $v \in S$.
\end{defi}

Notemos que el vector $0$ es ortogonal a cualquier otro vectores, de este modo un conjunto ortogonal puede contener al cero, pero un conjunto ortonormal no.

El principal atractivo de los conjuntos ortogonales y ortonormales es que cualquier vector que sea combinación lineal de este conjunto puede ser expresado en términos del producto interno, de este modo es bastante cómodo trabajar sobre bases que sean ortogonales u ortonormales.

\begin{teor}[Expansión de Fourier] \label{teor:ExoFourier}
  Sea $S = \{v_1, \ldots, v_n\}$ un conjunto ortogonal de vectores no nulos de un espacio con producto interno $V$, si $v \in \inner{S}$, entonces
    \[
      v = \sum_{i=1}^n \frac{\inner{v,v_i}}{\norm{v_i}^2} v_i.
    \]
\end{teor}

\begin{proof}
  Sea $v = \lambda_1 v_1 + \cdot + \lambda_n v_n$, notemos, por las propiedades del producto interno, que para cualquier $i \in \{1,\ldots,n\}$ se cumple que
  \[
    \inner{v,v_i} = \inner{\lambda_1 v_1 + \cdots + \lambda_n v_n,v_i} 
      = \lambda_1 \inner{v_1,v_i} + \cdots + \lambda_n \inner{v_n,v_i}
  \]
  Ahora, dado que $S$ es ortogonal, entonces $\inner{v_j, v_i} = 0$ para todo $j \in \{1,\ldots,n\}$ tal que $j \neq i$, además, dado que $v_i \neq 0$ por hipótesis, entonces $\norm{v_i} \neq 0$ y por tanto tenemos que
  \[
    \inner{v,v_i} = \lambda_i \inner{v_i,v_i} = \lambda_i \norm{v_i}^2 \implies \lambda_i = \frac{\inner{v,v_i}}{\norm{v_i}^2}.
  \]
  Lo que finalmente nos permite concluir que 
    \[
      v = \sum_{i=1}^n \frac{\inner{v,v_i}}{\norm{v_i}^2} v_i. \qedhere
    \]
\end{proof}

Una consecuencia de este teoremas es que si $S = \{v_1, \ldots, v_n\}$ es un conjunto ortonormal, dado que $\norm{v_i} = 1$ para todo $i\in\{1,\ldots,n\}$, entonces para cualquier $v\in \inner{S}$ se cumple que 
\[
  v = \sum_{i=1}^n \inner{v,v_i} v_i.
\]

Otra consecuencia importante es que todo conjunto ortogonal de vectores no nulos siempre es linealmente independientes, por lo que si se tiene un conjunto ortogonal no es necesario comprobar la independencia lineal. Además, los coeficientes dados en el teorema determinan completamente a cualquier combinación lineal. Por ello, a los coeficientes del teorema anterior se les conoce como los \emph{coeficientes de Fourier} de un vector.

\begin{coro} \label{coro:IndepOrtog}
  Sea $S = \{v_1, \ldots, v_n\}$ un conjunto ortogonal de vectores no nulos de un espacio con producto interno $V$, entonces $S$ es linealmente independiente.
\end{coro}
\begin{proof}
  Por la demostración del teorema \ref{teor:ExoFourier} sabeos que si $\lambda_1 v_1 + \cdots + \lambda_n v_n = 0$ entonces para todo $i \in  \{0,\ldots,n\}$ se cumple que
    \[
      \lambda_i = \frac{\inner{0,v_i}}{\norm{v_i}^2} = 0,
    \]
  lo que implica que $S$ es linealmente independiente.
\end{proof}



\subsection{Proceso de ortogonalización de Gram-Schmidt}

Ya vimos que tener una base ortogonal tiene muchas ventajas, entonces es normal querer buscar este tipo de bases. De igual forma es normal cuestionarse si tales bases existen y cómo construirlas en el caso que existan.

La respuesta de la primera cuestión es que siempre existen bases ortogonales para todo espacio vectorial con producto interno. La forma de construirla es usando un método llamado proceso de ortogonalización de Gram-Schmidt. Este proceso funciona de la siguiente forma: supongamos que tenemos un conjunto linealmente independiente $S = \{v_1,\ldots,v_n\}$ de un espacio con producto interno, entonces definimos de manera recursiva los vectores $S' = \{w_1,\ldots,w_n\}$ de la siguiente forma, primero $w_1 = v_1$ y para $i \in \{2,\ldots,n\}$ se define
  \[
    w_i = v_i - \sum_{k=1}^{i-1} \frac{\inner{v_i,w_k}}{\norm{w_k}^2} w_k.
  \]

Estos vectores $w_1,\ldots,w_n$ cumplen que para todo $i \in \{1,\ldots,n\}$ el vector $w_i$ es no nulo, el conjunto $\{w_1,\ldots,w_i\}$ es ortogonal y además $\inner{v_1,\ldots,v_i} = \inner{w_1,\ldots,w_n}$. La demostración de estas afirmaciones se da en el siguiente teorema.

\begin{teor}[Proceso de ortogonalización de Gram-Schmidt]
  Sea $V$ un espacio con producto interno y sea $\{v_1,\ldots,v_n\}$ un conjunto de vectores linealmente independientes de $V$. Entonces, se pueden construir un conjunto ortogonal $\{w_1,\ldots,w_n\}$ tal que para todo $i \in \{1,\ldots,n\}$ el conjunto $\{w_1,\ldots,w_i\}$ es una base de $\inner{v_1,\ldots,v_i}$.
\end{teor}
\begin{proof}
  Es claro que si $n=1$ entonces definiendo $w_1 = v_1$ se cumple con lo pedido, de esta forma supongamos que existe $s>1$ tal que la proposición se cumple para todo $1 \leq r < s$, mostremos que también se cumple para $s$.

  Sea $\{v_1,\ldots,v_s\}$ un conjunto linealmente independiente,  por hipótesis de inducción tenemos que para $\{v_1,\ldots,v_{s-1}\}$ existe un conjunto ortogonal $\{w_1,\ldots,w_{s-1}\}$ tal que cumple con la proposición, de esta forma definamos
    \[
      w_s = v_s - \sum_{k=1}^{s-1} \frac{\inner{v_s,w_k}}{\norm{w_k}^2} w_k.
    \]
  
  En primer lugar, notemos que $w_s$ no es el vector cero ya que en caso contrario se cumpliría que
    \[
      v_s = \sum_{k=1}^{s-1} \frac{\inner{v_s,w_k}}{\norm{w_k}^2} w_k
    \]
  y por tanto $v_s \in \inner{w_1,\ldots,w_{s-1}}=\inner{v_1,\ldots,w_{v-1}}$, pero eso contradice que $\{v_1,\ldots,v_s\}$ es linealmente independiente.

  Ahora, sea $i \in \{1,\ldots,s-1\}$, dado que para todo $j \in  \{1,\ldots,s-1\}$ con $i \neq j$ se cumple que $\inner{w_j,w_i} = 0$, ya que $\{w_1,\ldots,w_{s-1}\}$ es ortogonal, entonces veamos que
    \begin{align*}
      \inner{w_s, w_i}
        &= \Inner{v_s - \sum_{k=1}^{s-1} \frac{\inner{v_s,w_k}}{\norm{w_k}^2} w_k, w_i} 
        = \inner{v_s,w_i} - \sum_{k=1}^{s-1} \frac{\inner{v_s,w_k}}{\norm{w_k}^2} \inner{w_k, w_i} \\
        &= \inner{v_s,w_i} - \frac{\inner{v_s,w_i}}{\norm{w_i}^2} \inner{w_i, w_i} 
        = \inner{v_s,w_i} - \inner{v_s,w_i} \\
        &= 0.
    \end{align*}
  Pero esto implica que $\{w_1,\ldots,w_s\}$ es un conjunto ortogonal y todos sus vectores son distintos de cero, así, por el corolario \ref{coro:IndepOrtog} tenemos que $\{w_1,\ldots,w_s\}$ es linealmente independiente, por lo que para todo $i \in \{1,\ldots,s\}$ se cumple que $\{w_1,\ldots,w_i\}$ es una base de $\inner{v_1,\ldots,v_i}$. Así, por inducción, la proposición se cumple para todo $n \in \N$.
\end{proof}

\begin{coro}
  Sea $V$ un espacio con producto interno de dimensión finita, entonces $V$ tiene una base ortogonal y una base ortonormal.
\end{coro}
\begin{proof}
  Dado que $V$ tiene una base, por el proceso de ortogonalización de Gram-Schmidt podemos construir una base ortogonal $B = \{v_1,\ldots,v_n\}$ de $V$, ahora definamos el conjunto
    \[
      B' = \set{ \frac{1}{\norm{v_1}}v_1, \ldots, \frac{1}{\norm{v_n}}v_n }.
    \]
  Por construcción $B'$ va a ser ortonormal, además dado que cada vector de $B'$ es un múltiplo escalar de uno de $B$, entonces también es una base de $V$.
\end{proof}

\section{Formas sesquilineales}

Así como las transformaciones lineales pueden expresarse como matrices, también es posible realizar un proceso similar con los productos internos, para ello necesitamos algunas definiciones.

\begin{defi}
  Sea $V$ un $\K$-espacio vectorial decimos que $f\colon V\times V \to \K$ es una \emph{forma sesquilineal} si para cualesquiera $v,w,u \in V$ y $\lambda \in \K$ cumple que
    \begin{enumerate}
      \item $f(v+\lambda w, u) = f(v,u) + \lambda f(w,u)$.
      \item $f(u, v+\lambda w) = f(u,v) + \bar\lambda f(u,w)$.
    \end{enumerate}
  Además, decimos que la forma sesquilineal es \emph{hermítica} si $f(v,w) = \overline{f(w,v)}$ para cualesquiera $v,w\in V$.
\end{defi}

En el caso donde $\K=\R$, las formas sesquilineal reciben el nombre de \emph{formas bilineales}, además, en vez de decir que  una forma bilineal es hermítica, decimos que es \emph{simétrica}.

Una observación importante es que que cualquier producto interno es una forma sesquilineal hermítica, pero la vuelta no es cierta, ya que no necesariamente se cumple que es definida positiva, es decir, no necesariamente se cumple que $f(v,v)>0$ para todo $v\neq 0$.

\subsection{La matriz asociada de una forma sesquilineal}

Supongamos que para un $\K$-espacio vectorial $V$ tenemos una base ordenada $B = (v_1, \ldots, v_n)$ y una forma sesquilineal $f\colon V\to V \to\K$, notemos que si $v,w \in V$, donde $[v]_B = (\lambda_1, \ldots,\lambda_n)^t$ y $[w]_B = (\mu_1, \ldots,\mu_n)^t$, entonces por definición se cumple que
\begin{align*}
  f(v,w) &= f\paren{ \sum_{i=1}^n \lambda_i v_i, \sum_{j=1}^n \mu_j v_j } 
     = \sum_{i=1}^n \lambda_i  f\paren{ v_i, \sum_{j=1}^n \mu_j v_j } \\
    &= \sum_{i=1}^n  \sum_{j=1}^n  \lambda_i\overline{\mu_j} f\paren{ v_i, v_j } 
     = \sum_{i=1}^n \lambda_i \sum_{j=1}^n  \overline{\mu_j} f\paren{ v_i, v_j } \\
    &= \begin{pmatrix} \lambda_1 & \lambda_2 & \cdots & \lambda_n \end{pmatrix}
       \begin{pmatrix} \sum_{j=1}^n \overline{\mu_j} f(v_1,v_j) \\
        \sum_{j=1}^n \overline{\mu_j} f(v_2,v_j) \\
        \vdots \\
        \sum_{j=1}^n \overline{\mu_j} f(v_n,v_j) 
       \end{pmatrix} \\
    &= \begin{pmatrix} \lambda_1 & \lambda_2 & \cdots & \lambda_n \end{pmatrix}
       \begin{pmatrix}
        f(v_1, v_1)  & f(v_1, v_2) & \cdots & f(v_1, v_n) \\
        f(v_2, v_1)  & f(v_2, v_2) & \cdots & f(v_2, v_n) \\
        \vdots & \vdots & \ddots & \vdots \\
        f(v_n, v_1)  & f(v_n, v_2) & \cdots & f(v_n, v_n) \\
       \end{pmatrix}
      \begin{pmatrix} \overline{\mu_1} \\ \overline{\mu_2} \\ \vdots \\ \overline{\mu_n} \end{pmatrix}.
\end{align*}
Pero entonces $f(v,w) = [v]_B^t M \overline{[w]_B}$ donde $M = \bigl(f(v_i,v_j)\bigr)$ para cualesquiera $v,w \in V$. La vuelta de esa propiedad también es cierta: si $M \in \M_n(\K)$ entonces la función $f\colon V\times V \to \K$ dada por $f(v,w) = [v]_B^t M \overline{[w]_B}$ es una forma sesquilineal. Estas propiedades las podemos resumir en el siguiente teorema.

\begin{teor}\label{teor:MAsocFS}
  Sea $V$ un $\K$-espacio vectorial, $B = (v_1,\ldots,v_n)$ una base ordenada de $V$ y $f\colon V\times V \to\K$ una forma sesquilineal, entonces existe una única matriz $M_f \in \M_n(\K)$ tal que para cualesquiera $v,w \in V$ se cumple que
    \[
      f(v,w) =  [v]_B^t M_f \overline{[w]_B}.
    \]
    A la matriz $M_f$ se le conocerá como la \emph{matriz asociada} de $f$ bajo la base ordenada $B$ y cumple que $M_f = \bigl(f(v_i,v_j)\bigr)$. Además, existe una biyección entre las formas sesquilineales y el conjunto de matrices de $n\times n$ con entradas en $\K$.
\end{teor}
\begin{proof}
  Ya demostramos la existencia, falta solo ver la unicidad. Así, supongamos que $M \in \M_n(\K)$ es alguna matriz tal que $f(v,w) =  [v]_B^t M \overline{[w]_B}$ para cualesquiera $v,w \in V$. De este modo, notemos que para cualesquiera $i,j \in \{0,\ldots,n\}$ se cumple que
    \[
      (M_f)_{i,j} = f(v_i,v_j) = [v_i]_B^t M \overline{[v_j]_B} = e_i^t M e_j = e_i^t M_{*j} = M_{ij}
    \]
  y por tanto $M_f = M$, lo cual muestra la unicidad.
  
  La biyección también es inmediata, ya que si $S(V)$ es el conjunto de formas sesquilineales de $V$, entonces la función $\Phi_B\colon S(V) \to \M_n(\K)$ dada por $\Phi_B(f) = M_f$ es una biyección, por la unicidad y dado que para cualquier $M\in\M_n(\K)$ la función $f_M\colon V\times V \to \K$ dada por $f(v,w) = [v]_B^t M \overline{[w]_B}$ es una forma sesquilineal.
\end{proof}

\subsection{La matriz asociada a un producto interno}

Ya vimos que, dada una base ordenada, toda forma sesquilineal tiene asociada una única matriz y dado que todo producto interno es una forma sesquilineal, entonces los productos internos también tienen asociada una matriz. De esta forma, buscaremos qué propiedades debe cumplir la matriz asociada de un producto interno.

Para continuar con el estudio de las matrices asociadas a una forma sesquilineal y productos internos, necesitamos una definición previa.

\begin{defi}
  Sea $M \in \M_n(\K)$, decimos que $M$ es \emph{hermítica} si $M = M^*$. En el caso en que $\K = \R$, entonces a una matriz hermítica se le dice que es \emph{simétrica}.
\end{defi}

Con esta definición entonces podemos demostrar la siguiente proposición.

\begin{prop}\label{prop:MAsocFSH}
  Sean $V$ un $\K$-espacio vectorial, $B = (v_1,\ldots,v_n)$ una base ordenada de $V$, $f\colon V\times V \to\K$ una forma sesquilineal y $M_f$ la matriz asociada a $f$ bajo $B$, entonces $f$ es hermítica si y solo si $M_f$ es hermítica.
\end{prop}
\begin{proof}
  Supongamos que $f$ es hermítica, por el teorema \ref{teor:MAsocFS} sabemos que para cualesquiera $i,j \in \{0,\ldots,n\}$ se cumple que $(M_f)_{ij} = f(v_i, v_j)$, de este modo, dado que $f$ es hermítica, entonces tenemos que
  \[
    (M_f)_{ij} = f(v_i, v_j) = \overline{f(v_j, v_i)} = \overline{(M_f)_{ji}} = (M_f^*)_{ij}.
  \]
  Así, tenemos que $M_f = M_f^*$ y por definición, $M_f$ es hermítica.

  Ahora supongamos que $M_f$ es hermítica, sean $v,w \in V$ entonces notemos que por definición y propiedades de la adjunta que
    \[
      \overline{f(w,v)} = \bigl(f(w,v)\bigr)^* = ([w]_B^t M_f \overline{[v]_B})^*
    \]
  Dado que la multiplicación de matrices está compuesta de sumas y multiplicaciones no es difícil mostrar que $(AB)^* = B^* A^*$ para cualesquiera $A \in \M_{n\times k} (\K)$ y $B \in \M_{k\times m} (\K)$, de este modo, dado que $M_f = M_f^*$, entonces se cumple que
    \[
      \overline{f(w,v)} =  (\overline{[v]_B})^* M_f^* ([w]_B^t)^* = [v]_B^tM_f \overline{[w]_B} = f(v,w).
    \]
  Así, $f$ es hermítica por definición.
\end{proof}

Dado que no toda forma sesquilineal hermítica es un producto interno, esta proposición solo nos da el primer paso para encontrar las matrices asociadas a un producto interno. Para ello, recordemos que la única propiedad que le falta a una forma sesquilineal hermítica para ser un producto interno es que sea definida positiva, por lo que tenemos que revisar cuáles matrices nos brindan esta propiedad.

Con eso en mente, consideremos un espacio con producto interno $V$ y una base ordenada $B = (v_1,\ldots,v_n)$ de $V$. Veamos que si $M$ es la matriz asociada al producto interno, por la proposición \ref{prop:MAsocFSH} se debe cumplir que $M$ es hermítica, más aún, por las propiedades del producto interno, para todo $v \in V$ se debe cumplir que 
  \[
    [v]_B^t M \overline{[v]_B} > 0.
  \]
Dado que $[\cdot]_B$ es una isomorfismo y considerando las propiedades de la conjugación, no es difícil deducir que $M$ debe satisfacer que $x^* M x > 0$ para todo $x \in \K^n$. Este tipo de matrices son especiales y reciben su propio nombre.

\begin{defi}
  Sea $M \in \M_n(\K)$ una matriz hermítica, decimos que $M$ es \emph{definida positiva} si para todo $x \in \K^n$ se cumple que $x^* M x > 0$.
\end{defi}

Con las propiedad que tenemos hasta ahora, aun no tenemos una forma simple de determinar cuando una matriz es definida positiva, pero podemos inferir algunas de sus propiedades. En primer lugar si $M$ es definida positiva, entonces $M$ debe ser invertible, ya que en caso contrario existiría $x \neq 0$ tal que $x^* M x = 0$ Otra propiedad interesante es que para todo $i \in \{1,\ldots,n\}$ se cumple que
\[
  M_{ii} = e_i^* M e_i > 0,
\]
es decir, las entradas en la diagonal de $M$ son todas positivas. Por último y como se podría suponer, las matrices definidas positivas son las matrices asociadas a un producto interno.

\begin{teor}
  Sea $V$ un $\K$-espacio vectorial, $B = (v_1,\ldots,v_n)$ una base ordenada de $V$, $f\colon V\times V \to\K$ una forma sesquilineal y $M_f$ la matriz asociada a $f$ bajo $B$, entonces $f$ es un producto interno si y solo si $M_f$ es definida positiva.
\end{teor}
\begin{proof}
  Sea $f$ un producto interno, por la proposición \ref{prop:MAsocFSH} se cumple que $M_f$ es hermítica, de este modo solo falta ver que $x^* M x > 0$ para todo $x \in \K$. Ahora, dado que $[\cdot]_B$ es una isomorfismo, entonces para todo $x \in \K^n-\{\bec 0\}$ existe $v \in V - \{0\}$ tal que $[v]_B = \bar x$, de este modo, dado que $f(v,v)>0$, entonces tenemos que
  \[
    f(v,v) = [v]_B^t M_f \overline{[v]_B} = x^* M_f x > 0.
  \]
  Lo que muestra que $M_f$ es definida positiva.

  Ahora, si $M_f$ es definida positiva entonces es hermítica y por la proposición \ref{prop:MAsocFSH} $f$ también es hermítica. De igual manera, veamos que para todo $v\in V - \{0\}$ se cumple que $\overline{[v]_B} \neq \bec 0$ y por tanto
  \[
    f(v,v) = [v]_B^t M_f \overline{[v]_B} = \overline{[v]_B}^* M_f \overline{[v]_B} > 0.
  \]
  Lo que finalmente muestra que $f$ es un producto interno de $V$.
\end{proof}

\section{Matrices normales y semejanza unitaria}

En esta sección estaremos trabajando con un conjunto muy especial de matrices, así como su relación con sus valores y vectores propios. 

\begin{defi}
  Sea $N \in \N_n(\K)$, decimos que $N$ es \emph{normal} si $NN^* = N^*N$. De igual forma, si $U\in\M_n(\K)$ decimos que $U$ es \emph{unitaria} si $UU^* = I$. En el caso en que $\K = \R$, a las matrices unitarias se les llamará \emph{ortogonales}.
\end{defi}

Una observación interesante es que, por definición, no es difícil ver que las matrices hermíticas y unitarias son normales. Ahora, las matrices unitarias son de especial interés, dado que si las consideramos como matrices de cambio de base, entonces la base será ortonormal bajo el producto interno canónico. Por convención, en lo que resta de la sección, el producto interno a usar será el canónico.

\begin{prop}\label{prop:PropUnitaryMat}
  Sea $U \in \M_n(\K)$, las siguientes afirmaciones son equivalentes:
  \begin{enumerate}
    \item $U$ es unitaria.
    \item $U^*$ es unitaria.
    \item Las columnas de $U$ conforman un conjunto ortonormal.
    \item Los renglones de $U$ conforman un conjunto ortonormal.
    \item $v \bcdot w = (Uv) \bcdot (Uw)$ para cualesquiera $v,w \in \K^n$.
  \end{enumerate}
\end{prop}
\begin{proof}
  En primero lugar, por propiedades de los productos internos y la conjugación, notemos que $w \bcdot v = \overline{v\bcdot w} = v^* w $, de esta forma tenemos que $v^* w = r$ si y solo si $v\bcdot w = r$ cuando $r \in \R$, con esto dicho, pasemos a la demostración del teorema.
  
  \medskip\noindent
  $(1 \Leftrightarrow 2)$ Dado que $UU^* = I$ entonces $U^{-1} = U^*$ por lo que $U^* (U^*)^* = U^* U = UU^* = I$, lo que muestra que $U^*$ es unitaria. Ahora si $U^*$ es unitaria, por la parte anterior se cumple que $(U^*)^* = U$ es unitaria.

  \medskip\noindent
  $(1 \Rightarrow 3)$ Si $U = \bigl( v_1 \mid \cdots \mid v_n \bigr)$, dado que $U$ es unitaria, entonces $ U^* U = UU^* =I$ y por propiedades de la adjunta, tenemos que
  \[
    U^* U = \begin{spmatrix}{c} v_1^* \\\hline v_2^* \\\hline \vdots \\\hline v_n^* \end{spmatrix}
      \begin{spmatrix}{c|c|c|c} v_1 & v_2 & \cdots & v_n \end{spmatrix}
      = \begin{pmatrix}
        v_1^* v_1 & v_1^* v_2 & \cdots & v_1^* v_n \\
        v_2^* v_1 & v_2^* v_2 & \cdots & v_2^* v_n \\
        \vdots & \vdots & \ddots & \vdots \\
        v_n^* v_1 & v_n^* v_2 & \cdots & v_n^* v_n 
      \end{pmatrix} = I.
  \]
  Por igualdad de matrices tenemos que $v_i^* v_j \in \R$ para todos $i,j\in\{1,\ldots,n\}$, pero por lo dicho al principio esto implica que $v_i^* v_j  = v_i \bcdot v_j$ y por la ecuación anterior $v_i \bcdot v_j$ siempre que $i \neq j$ y $v_i \bcdot v_j = 1$ cuando $i=j$, pero por definición, esto muestra que las columnas de $U$ forman un conjunto ortonormal.
    
  \medskip\noindent
  $(3 \Rightarrow 1)$ Si $U = \bigl( v_1 \mid \cdots \mid v_n \bigr)$, dado que sus columnas forman un conjunto ortonormal, entonces tenemos que $v_i \bcdot v_j \in \R$ para cualesquiera $i,j \in \{1,\ldots,n\}$ y por lo mencionado al principio, entonces se cumple que $ v_i^* v_j = v_i \bcdot v_j  = 0$ si $i \neq j$ y $v_i^* v_j = v_i \bcdot v_j  =  1$ si $i = j$. De este modo, realizando un cálculo similar al de inciso anterior, tenemos que
  \[
    U^* U =
      \begin{pmatrix}
        v_1^* v_1 & v_1^* v_2 & \cdots & v_1^* v_n \\
        v_2^* v_1 & v_2^* v_2 & \cdots & v_2^* v_n \\
        \vdots & \vdots & \ddots & \vdots \\
        v_n^* v_1 & v_n^* v_2 & \cdots & v_n^* v_n 
      \end{pmatrix}  =
      \begin{pmatrix}
        v_1 \bcdot v_1 & v_1 \bcdot v_2 & \cdots & v_1 \bcdot v_n \\
        v_2 \bcdot v_1 & v_2 \bcdot v_2 & \cdots & v_2 \bcdot v_n \\
        \vdots & \vdots & \ddots & \vdots \\
        v_n \bcdot v_1 & v_n \bcdot v_2 & \cdots & v_n \bcdot v_n 
      \end{pmatrix} 
    = I.
  \]
  Mostrando, por definición, que $U$ es unitaria.

  \medskip\noindent
  $(1 \Rightarrow 4)$ Si $U$ es unitaria, entonces $UU^* = U^* U = I$, aplicando traspuesta y por propiedades de la adjunta se cumple que $U^t (U^t)^* = (U^* U)^t  = I^t = I$, lo que muestra que $U^t$ es unitaria, pero por incisos anteriores, eso implica que las columnas de $U^t$ forman un conjunto ortonormal. Dado que las columnas de $U^t$ son los renglones de $U$, entonces tenemos que los renglones de $U$ forman un conjunto ortonormal.

  \medskip\noindent
  $(4 \Rightarrow 1)$ Si los renglones de $U$ forman un conjunto ortonormal, entonces las columnas de $U^t$ forman un conjunto ortonormal y por incisos anteriores, eso implica que $U^t$ es unitaria, así $U^t (U^t)^* = I$. Dado que $(U^* U)^t = U^t (U^t)^* $ entonces tenemos que $U^* U = I^t = I$, lo que muestra que $U$ es unitario.

  \medskip\noindent
  $(1 \Rightarrow 5)$ Si $U$ es unitario, entonces $U^* U = I$, aplicando la conjugada obtenemos que $U^t \overline{U} = I$, de esta forma, para cualesquiera $v,w \in \K^n$ se cumple que 
      \[
        (Uv) \bcdot (Uw) = (Uv)^t (\overline{Uw}) = v^t U^t \overline{U}\overline{w} = v^t I \overline{w} = v\bcdot w.
      \]
  
  \medskip\noindent
  $(5 \Rightarrow 1)$ Notemos que si $v\bcdot w  = (Uv) \bcdot (Uw)$, realizando el mismo cálculo que en el inciso anterior, entonces $v\bcdot w = v^t I \overline{w} = v^t (U^t \overline{U}) \overline{w}$, pero eso implica que $U^t \overline{U}$ y $I$ son matrices asociadas al producto interno $\bcdot$ bajo la base canónica, pero por el teorema \ref{teor:MAsocFS} esto implica que $U^t \overline{U} = I$. Aplicando la conjugación finalmente obtenemos que $U^* U = \overline{I} = I$, lo que muestra que $U$ es unitaria.
\end{proof}

Esta proposición nos muestra que las matrices unitarias son simplemente cambios de base donde la base es un conjunto ortonormal. Dado que este tipo de cambios de base son bastante importantes, es útil saber cuando dos matrices están asociadas bajo este tipo de cambio de base. Así podemos justificar la siguiente definición.

\begin{defi}
  Sea $M,N \in \M_n(\K)$, decimos que $M$ es \emph{unitariamente semejante} a $N$ si existe una matriz $U\in \M_n(\K)$ unitaria tal que $M = U^* N U$. A las matrices unitariamente semejante, en el caso en que $\K = \R$, se les dice que son \emph{ortogonalmente semejantes}.
\end{defi}

Al igual que con la semejanza de matrices, la semejanza unitaria será una relación de equivalencia, pero a diferencia de la semejanza, la semejanza unitaria preserva algunas de las propiedades importantes que hemos visto.

\begin{teor}
  Sean $M,N \in \M_n(\K)$ donde $M$ es unitariamente semejante a $N$, entonces:
  \begin{enumerate}
    \item $M$ es normal si y solo si $N$ es normal.
    \item $M$ es hermítica si y solo si $N$ es hermítica.
    \item $M$ es unitaria si y solo si $N$ es unitaria.
    \item $M$ es definida positiva si y solo si $N$ es definida positiva.
  \end{enumerate}
\end{teor}
\begin{proof}
  En primer lugar, por la simetría de la relación de semejanza unitaria, no es necesario probar la vuelta de las proposición, sino que basta con la ida. Así demostremos la ida.
  \begin{enumerate}
    \item Sea $M = U^* N U$, donde $U$ es una matriz unitaria, dado que $M$ es normal y $M^* = U^* N^* U$, entonces veamos que
      \begin{align*}
        MM^* &= M^* M \\
        (U^* N U)(U^* N^* U) &= (U^* N^* U)(U^* N U) \\
        U^* N N^* U &= U^* N^* N U.
      \end{align*}
    Dado que $U$ es invertible, entonces es fácil ver que $NN^* = N^* N$, por lo que $N$ es normal.

    \item Análogamente al caso anterior, si $M = U^* N U$, donde $U$ es una matriz unitaria, dado que $M$ es hermítica y $M^* = U^* N^* U$, entonces veamos que
      \[ M =  U^* N U = U^* N^* U = M^*, \]
    y dado que $U$ es invertible, entonces $N = N^*$, por lo que $N$ es hermítica.

    \item Análogamente, si $M = U^* N U$, donde $U$ es una matriz unitaria, dado que $M$ es unitaria y $M^* = U^* N^* U$, entonces veamos que
    \begin{align*}
      I &= MM^*  \\
      &= (U^* N U)(U^* N^* U)  \\
      &= U^* N N^* U 
    \end{align*}
  Dado que $U$ es invertible, entonces $NN^* = I$, por lo que $N$ es unitaria.

  \item Por ultimo, si $M = U^* N U$, donde $U$ es una matriz unitaria, dado que $M$ es definida positiva entonces es hermítica y por un inciso anterior sabemos que $N$ es hermítica, así, solo faltaría mostrar que $x^* N x > 0$ para todo vector $x \in \K^n - \{\bec 0\}$.
  
  Dado que $U$ es invertible, si $x \neq \bec 0$ entonces $Ux \neq \bec 0$ por lo que $(Ux)^* M (Px)>0$ de esta forma tenemos que
    \[x^* N x = x^* (U^* M U) x = (Ux)^* M (Px) > 0,\]
  lo que finalmente muestra que $N$ es definida positiva. \qedhere
  \end{enumerate}
\end{proof}


\subsection{Matrices unitariamente triangularizables y diagonalizables}

De igual forma con la semejanza, ahora buscamos saber cuando una matriz puede ser diagonalizable o triangularizable, pero esta vez, mediante la semejanza unitaria. 

Ya sabemos que en los complejos, toda matriz es triangularizable. Este resultado sale de aprovecharse de que por la proposición \ref{prop:MComplexHasEV} siempre existe al menos un valor propio en cualquier matriz y construir una base acorde.

Resulta que gracias a que siempre podemos construir bases ortonormales, este proceso se puede replicar de manera exacta, pero esta vez usando bases ortonormales.

\begin{teor}[Lema de Schur]
  Sea $M \in\M_n(\C)$, entonces $M$ es unitariamente semejante a alguna matriz triangular.
\end{teor}
\begin{proof}
  Primero, probemos el teorema para matrices triangulares superiores. Procedamos por inducción, podemos ver que trivialmente se cumple para $n = 1$, de este modo supongamos que se cumple para $k$ y demostremos que se cumple para $k+1$.
  
  Sea $M \in M_{k+1}(\C)$, por la proposición \ref{prop:MComplexHasEV} sabemos que $M$ tiene al menos un valor propio $\lambda$, sea $v$ un vector propio del valor $\lambda$ de tal manera que $\norm{v}=1$. Ahora, por teoremas de álgebra lineal y aplicando el proceso Gram-Schmidt podemos extender a $\{v\}$ para obtener una base $B = \{ v, q_1, \ldots, q_n \}$ ortonormal de $\C^{k+1}$.
  
  Sea $Q = ( v \mid  q_1 \mid \ldots \mid q_n  ) $, notemos que $Q$ es unitario, por la proposición \ref{prop:PropUnitaryMat}, de esta forma si $N = Q^* M Q$, entonces por multiplicación por bloques y recordando que $Q^*v = e_1$ por propiedades de la matriz inversa, veamos que
  \begin{align*}
      N &= Q^* \begin{spmatrix}{c|c|c|c} M v &  M u_1 & \ldots & M u_n  \end{spmatrix} \\
        &= Q^* \begin{spmatrix}{c|c|c|c} \lambda v &  M u_1 & \ldots & M u_n  \end{spmatrix} \\
        &= \begin{spmatrix}{c|c|c|c} \lambda Q^* v  &  Q^* M u_1 & \ldots & Q^* M u_n  \end{spmatrix} \\
        &= \begin{spmatrix}{c|c|c|c} \lambda e_1 &  Q^* M u_1 & \ldots & Q^* M u_n  \end{spmatrix}.
  \end{align*}
  De esta forma para alguna matriz $L \in M_k(\C)$ y vector $u \in \C^k$, podemos ver que $N$ tiene la forma
  \[
  N = \begin{spmatrix}{c|c} \lambda & u^t \\\hline \bec 0 & L \end{spmatrix}.
  \]
  
  Ahora, por hipótesis de inducción $L$ es unitariamente semejante a una matriz triangular superior $U$, en otras palabras existe alguna matriz unitaria $R$ tal que $U = R^* L R$. De esta forma si definimos a $P$ como
  \[ P = \begin{spmatrix}{c|c} 1 & \bec 0^t \\\hline \bec 0 & R \end{spmatrix} \]
  entonces notemos, por propiedades de la multiplicación por bloques, se cumple que
  \[ PP^* = \begin{spmatrix}{c|c} 1 & \bec 0^t \\\hline \bec 0 & R \end{spmatrix} \begin{spmatrix}{c|c} 1 & \bec 0^t \\\hline \bec 0 & R^* \end{spmatrix} = \begin{spmatrix}{c|c} 1 & \bec 0^t \\\hline \bec 0 & RR^* \end{spmatrix} = \begin{spmatrix}{c|c} 1 & \bec 0^t \\\hline \bec 0 & I_k \end{spmatrix} = I_{k+1}, \]
  de esta forma podemos ver que $P$ es unitaria. De esta forma, si definimos a $T = P^* N P$ entonces veamos que
  \begin{align*}
  T &= \begin{spmatrix}{c|c} 1 & \bec 0^t \\\hline \bec 0 & R^* \end{spmatrix}  \begin{spmatrix}{c|c} \lambda & u^t \\\hline \bec 0 & L \end{spmatrix} \begin{spmatrix}{c|c} 1 & \bec 0^t \\\hline \bec 0 & R \end{spmatrix}  \\
    &= \begin{spmatrix}{c|c} 1 & \bec 0^t \\\hline \bec 0 & R^* \end{spmatrix}  \begin{spmatrix}{c|c} \lambda & u^t R \\\hline \bec 0 & LR \end{spmatrix}
    = \begin{spmatrix}{c|c} \lambda & u^t R \\\hline \bec 0 & R^*LR \end{spmatrix} \\
    &= \begin{spmatrix}{c|c} \lambda & u^t R \\\hline \bec 0 & U \end{spmatrix}.
  \end{align*}
  Dado que $U$ es triangular superior, entonces podemos ver que $T$ es triangular superior, pero como $P$ es unitaria, entonces $T$ es unitariamente semejante a $N$ y como $Q$ es unitario, entonces $N$ es unitariamente semejante a $M$, pero por transitividad de esta equivalencia, entonces tenemos que $T$ es unitariamente semejante a $M$. De esta forma, dado que la proposición se cumple para $k+1$, entonces toda matriz $M \in \M_n(\C)$ es unitariamente semejante a una matriz triangular superior.
  
  Ahora demostremos la proposición para matrices triangulares inferiores. Por la parte anterior sabemos que si $M \in \M_n(\C)$ entonces $M^*$ es unitariamente semejante a una matriz triangular superior $T$, es decir, existe alguna matriz unitaria $Q$ tal que $M^* = U^* T U$, ahora notemos que
  \[ M = (M^*)^* = (U^* T U)^* = U^* T^* U,\]
  de aquí podemos ver que $M$ es semejante a $T^*$ la cual es una matriz triangular inferior. Así probamos que si $M \in \M_n(\C)$ entonces $M$ es unitariamente equivalente a una matriz triangular superior y a una inferior.
  \end{proof}

  Ya tenemos que todas las matrices son unitariamente semejantes a alguna matriz triangular, pero antes de poder caracterizar las matrices diagonales necesitamos otro lema.

\begin{lema}\label{lema:TriangNorm}
  Sea $T \in \M_n(\K)$ una matriz triangular, $T$ es normal si y solo si $T$ es diagonal.
\end{lema}
\begin{proof}
  Primero, probemos la ida para matrices triangulares superiores. Por inducción, notemos que la proposición se cumple trivialmente para $n=1$, de esta manera, supongamos que se cumple para $k$ y veamos que se cumple también para $k+1$. Ahora, si $T \in \M_{k+1}(\K)$ es una matriz triangular inferior, notemos entonces que podemos escribirla de la siguiente forma
  \[
    T = \begin{spmatrix}{c|c} \lambda & v^t  \\\hline   \bec 0 & N  \end{spmatrix},
  \]
  donde $\lambda \in \K$, $v \in\K^k$ y $N \in \M_k(\K)$ es una matriz triangular superior. Ahora, por multiplicación por bloques y recordando que $\norm{v}^2 = v\bcdot v = v^t \bar v $, notemos que
  \begin{align*}
    TT^*
      &= \begin{spmatrix}{c|c} \lambda & v^t  \\\hline   \bec 0 & N  \end{spmatrix} \begin{spmatrix}{c|c} \bar\lambda & \bec 0^t  \\\hline   \bar v & N^* \end{spmatrix} &
        T^* T 
          &=  \begin{spmatrix}{c|c} \bar \lambda & \bec 0^t  \\\hline   \bar v & N^* \end{spmatrix}\begin{spmatrix}{c|c} \lambda & v^t  \\\hline   \bec 0 & N  \end{spmatrix} \\
      &= \begin{spmatrix}{c|c} \abs{\lambda}^2 + \norm{v}^2 & v^tN^* \\\hline N\bar v & NN^* \end{spmatrix}, &
          &= \begin{spmatrix}{c|c} \abs{\lambda}^2 & \bar t v^t \\\hline t \bar v & \bar v v^t + N^*N \end{spmatrix}. \tagthis\label{eq:1}
  \end{align*}

  Ahora, notemos que por hipótesis $TT^* = T^* T$, de esta forma, usando la ecuación \eqref{eq:1}, tenemos que
  \[ \abs{\lambda}^2 + \norm{v}^2 = \abs{\lambda}^2 \implies \norm{v}^2 = 0.\]
  Pero por propiedades de la norma, sabemos que $\norm{v}^2 = 0$ si y solo si $v = \bec 0$, pero eso implica que
  \begin{align*}
    TT^*
      &= \begin{spmatrix}{c|c} \abs{\lambda}^2  & \bec 0^t \\\hline \bec 0 & NN^* \end{spmatrix}, &
        T^* T 
          &=  \begin{spmatrix}{c|c} \abs{\lambda}^2 & \bec 0^t \\\hline \bec 0 &  N^*N \end{spmatrix}. \tagthis\label{eq:2}
  \end{align*}
  Pero de igual forma, por normalidad de $T$ y la la ecuación \eqref{eq:2}, esto implica que $NN^* = N^*N$, en otras palabras, $N$ es normal y triangular superior, pero por hipótesis de inducción eso implica que $N$ es diagonal, pero entonces $T$ tiene la forma
  \[
    T = \begin{spmatrix}{c|c} \lambda & \bec 0^t  \\\hline   \bec 0 & N  \end{spmatrix},
  \]
  donde $N$ es una matriz diagonal, pero eso implica finalmente que $T$ es diagonal. Dado que se cumple para $k+1$, por inducción podemos ver que si $T \in \M_n(\K)$ es normal y triangular superior entonces $T$ es diagonal.

  Ahora, supongamos que $T\in \M_n(\K)$ es triangular inferior y normal, entonces $T^*$ es una matriz triangular superior y además es normal dado que
  \[ T^* (T^*)^* = T^*T = TT^* = (T^*)^*T^*, \]
  pero por la parte anterior esto implica que $T^*$ es diagonal, por lo tanto $T$ es diagonal.

  Ahora, probemos la vuelta, sea $T$ una matriz diagonal notemos que por propiedades de las matrices diagonales $TT^*$ y $T^*T$ también son matrices diagonales y cumplen que
  \[ (TT^*)_{jj} = T_{jj} (T^*)_{jj} =  (T^*)_{jj} T_{jj} = (T^*T)_{jj}, \]
  de aquí podemos ver que $TT^* = T^*T$, por lo tanto $T$ es normal. Ahora, como $T$ es diagonal, entonces es triangular superior e inferior, de esta forma demostramos que $T$ es normal y triangular superior o inferior si y solo si $T$ es diagonal.
\end{proof}

Con este lema, finalmente podemos caracterizar a las matrices unitariamente diagonalizables.

\begin{teor}\label{teor:CaracMatNorm}
  $N \in \M_n(\C)$ es normal si y solo si es unitariamente diagonalizable.
\end{teor}
\begin{proof}
  Por el lema de Schur, sabemos que $N$ debe ser unitariamente equivalente a una matriz $T$ triangular superior, pero por propiedades de la semejanza unitaria, tenemos que $T$ debe ser normal, pero si $T$ es normal, aplicando el lema \ref{lema:TriangNorm}, entonces $T$ es una matriz diagonal. De esta forma si $N$ es normal, entonces es unitariamente diagonalizable.

  Ahora, si $N$ es unitariamente diagonalizable, entonces es unitariamente semejante a una matriz diagonal $D$. Ya que $D$ es una matriz diagonal, por el lema \ref{lema:TriangNorm}, tenemos que $D$ es normal, pero como la semejanza unitaria preserva la normalidad, esto implica que $N$ es normal. Así vemos que $N$ es normal si y solo si es unitariamente diagonalizable.
\end{proof}


\subsection{Caracterización de matrices unitarias y hermíticas}

Ya vimos que la normalidad es una condición necesaria y suficiente para que sea unitariamente diagonalizable. Así que, en lo que resta de esta sección, daremos algunas condiciones necesarias y suficientes para las matrices unitarias, hermíticas y definidas positivas.

Todas estas propiedades estarán asociadas a los valores propios de las matrices, dado que la semejanza unitaria también implica la semejanza, entonces gran parte de los teoremas relacionados con la semejanza también aplican para la semejanza unitaria.

El primer tipo de matriz que caracterizaremos serán las matrices unitarias, aunque ya la proposición \ref{prop:PropUnitaryMat} nos da algunas propiedades equivalentes, el siguiente teorema nos dará una visión acerca de qué deben cumplir sus vectores propios.

\begin{teor}
  $U \in \M_n(\C)$ es unitaria si y solo si es unitariamente diagonalizable y sus valores propios tienen módulo 1.
\end{teor}
\begin{proof}
  Sea $U$ unitaria, sabemos que $U$ es normal, de esta forma, por el teorema \ref{teor:CaracMatNorm}, tenemos que es unitariamente semejante a una matriz diagonal $D$. Ahora, dado que la semejanza unitaria conserva la propiedad de ser unitaria, entonces podemos ver que $D$ es también unitaria, de esta forma, cada columna de $D$ debe tener norma uno. De esta forma, si $D = (\lambda_1 e_1 \mid \cdots \mid  \lambda_n e_n)$, por propiedades de la norma, entonces
  \[
  \norm{D_{*i}} = \norm{ \lambda_{i} e_i } = \abs{\lambda_i} \norm{e_i} = \abs{\lambda_i} = 1.
  \]
  Pero recordemos, por el teorema \ref{teor:SemEspectro} y la proposición \ref{prop:EpecTriang}, que los elementos en la diagonal de $D$ corresponden a los valores propios de $U$, de esta forma tenemos si $U$ es unitaria entonces es unitariamente diagonalizable y sus valores propios tienen módulo 1.
  
  Ahora, si $U$ es unitariamente diagonalizable y sus valores propios tienen módulo 1, entonces por el teorema \ref{teor:SemEspectro} y la proposición \ref{prop:EpecTriang}, $U$ es unitariamente semejante a una matriz diagonal $D = (\lambda_1 e_1 \mid \cdots \mid  \lambda_n e_n)$ donde $E(U) = \{\lambda_1, \ldots, \lambda_n\}$. Ahora, notemos que, por propiedades de los complejos, dado que $\lambda_i$ tiene módulo 1, entonces $\overline{\lambda_i} = \lambda_i^{-1}$, de esta forma $D^* =  (\lambda_1^{-1} e_1 \mid \cdots \mid  \lambda_n^{-1} e_n) $ y por propiedades de las matrices diagonales, tenemos que
  \[ DD^* = \begin{spmatrix}{c|c|c} \lambda_1\lambda_1^{-1} e_1 & \cdots &  \lambda_n \lambda_n^{-1} e_n \end{spmatrix} = \begin{spmatrix}{c|c|c}  e_1 & \cdots &  e_n \end{spmatrix} = I_n. \]
  De esta forma tenemos que $D$ es unitaria, como la semejanza unitaria conserva la unitariedad, finalmente tenemos que $U$ es unitaria. Así $U$ es unitaria si y solo si es unitariamente diagonalizable y sus valores propios tienen módulo 1.
\end{proof}

El siguiente tipo de matrices a caracterizar serán las matrices hermíticas. 

\begin{teor}\label{teor:CaracMatHerm}
  $H \in \M_n(\C)$ es hermítica si y solo si es unitariamente diagonalizable y sus valores propios son reales.
\end{teor}
\begin{proof}
  Si $H$ es hermítica, entonces $H$ es normal y, por el teorema \ref{teor:CaracMatNorm}, es unitariamente semejante a una matriz diagonal $D$. Ahora, dado que la semejanza unitaria conserva la hermeticidad, entonces podemos ver que $D$ es también hermítica.
 
 De esta forma, si $D = (\lambda_1 e_1 \mid \cdots \mid  \lambda_n e_n)$ entonces $D^* = (\overline{\lambda_1} e_1 \mid \cdots \mid  \overline{\lambda_n} e_n)$, pero como $D$ es hermítica, entonces $D = D^*$, pero esto implica que $\lambda_i = \overline{\lambda_i}$, pero esto se cumple si y solo si $\lambda_i \in \R$. Dado que, por el teorema \ref{teor:SemEspectro} y la proposición \ref{prop:EpecTriang}, los elementos en la columna de $D$ corresponden a los valores propios de $H$, entonces tenemos si $H$ es hermítica entonces es unitariamente diagonalizable y sus valores propios son reales.
 
 Ahora, si $H$ es unitariamente diagonalizable y sus valores propios son reales, entonces por el teorema \ref{teor:SemEspectro} y la proposición \ref{prop:EpecTriang}, $H$ es unitariamente semejante a una matriz diagonal $D = (\lambda_1 e_1 \mid \cdots \mid  \lambda_n e_n)$ donde $E(H) = \{\lambda_1, \ldots \lambda_n\}$. Ahora, dado que $\lambda_i$ es real, entonces, por propiedades de los complejos $\overline{\lambda_i} = \lambda$, de esta forma
 \[ D^* =  \begin{spmatrix}{c|c|c} \overline{\lambda_1} e_1 & \cdots &  \overline{\lambda_n} e_n \end{spmatrix} = \begin{spmatrix}{c|c|c} \lambda_1 e_1 & \cdots &  \lambda_n e_n \end{spmatrix} = D.\]
 Por tanto, tenemos que $D$ es hermítica, como la semejanza unitaria conserva la hermeticidad, finalmente tenemos que $H$ es hermítica. Así $H$ es hermítica si y solo si es unitariamente diagonalizable y sus valores propios son reales.
\end{proof}

Finalmente, con todas estas propiedades podemos dar la condición que nos faltaba para las matrices asociadas a los productos internos.

\begin{teor}
  $M \in \M_n(\C)$ es definida positiva si y solo si es unitariamente diagonalizable y sus valores propios son positivos.
\end{teor}
\begin{proof}
  Si $M$ es definida positiva entonces es hermítica, entonces por el teorema \ref{teor:CaracMatHerm} tenemos que $M$ es unitariamente semejante a una matriz diagonal $D$ y sus valores propios son reales. Ahora, si $D = (\lambda_1 e_1 \mid \cdots \mid  \lambda_n e_n)$, dado que la semejanza unitaria preserva la definición positiva, entonces $D$ es definida positiva y por tanto, para todo $i\in\{1,\ldots,n\}$, se cumple que
    \[
      e_i^* D e_i = e_i^t (\lambda_i e_i) = \lambda_i > 0.
    \]
  Dado que $E(M) = \{\lambda_1,\ldots,\lambda_n\}$, el teorema \ref{teor:SemEspectro} y la proposición \ref{prop:EpecTriang}, entonces tenemos que si $M$ es definida positiva, entonces es unitariamente diagonalizable y sus valores propios son positivos.

  Ahora, si $M$ es unitariamente diagonalizable y sus valores propios son positivos, entonces por el teorema \ref{teor:SemEspectro} y la proposición \ref{prop:EpecTriang}, $M$ es unitariamente semejante a una matriz diagonal $D = (\lambda_1 e_1 \mid \cdots \mid  \lambda_n e_n)$ donde $E(M) = \{\lambda_1, \ldots \lambda_n\}$.

  Notemos que $D$ es hermítica, por la proposición \ref{prop:EpecTriang} y el teorema \ref{teor:CaracMatHerm}, así veamos que $x^* D x > 0$ para todo $x \in \C^n - \{\bec 0\}$. Dado que $\lambda_i > 0$ para todo $i \in \{1,\ldots,n\}$ si $x \neq \bec 0$ dado por $x = (x_1, \ldots, x_n)^t$ entonces veamos que
  \[
    x^* D x 
      = \begin{pmatrix} \overline{x_1} & \overline{x_2} & \cdots & \overline{x_n} \end{pmatrix} 
        \begin{pmatrix} \lambda_1 x_1 \\ \lambda_2 x_2 \\ \vdots \\ \lambda_n x_n \end{pmatrix}
      = \lambda_1 \abs{x_1} + \lambda_2 \abs{x_2} + \cdots + \lambda_n \abs{x_n} > 0.
  \]
  De esta forma, $D$ es definida positiva y dado que la semejanza unitaria conserva la definición positiva, entonces tenemos que $M$ es definida positiva.
\end{proof}

\end{document}
